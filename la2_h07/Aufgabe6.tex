\documentclass[main.tex]{subfiles}

\begin{document}

\section{Aufgabe 6}
Gilt die Beziehung
$$
    \det (A+B) = \det(A) + \det(B) \text{?}
$$
Beweisen Sie ihre Aussage.

\subsection{Lösung 6}
Wir nehmen an, dass $A,B \in K^{n\times n}$ gilt. Zwar ist nach Definition 5.1 Satz 3 die Determinante additiv, jedoch nur dann, wenn nru eine Spalte/Zeile unterschiedlich ist.\\

Als Gegenbeispiel für den allgemeinen Fall wählen wir $A = \eqmatrix{cc}{1 & 0\\ 0 & 0}$ und $B = \eqmatrix{cc}{0 & 0\\ 0 & 1}$. Es gilt $A+B = \mathcal{E}$ und $\det(\mathcal{E}) = 1$, jedoch ist $\det (A) + \det (B) = 0$ und damit ist gezeigt, dass $\det(A+B) \neq \det(A) + \det(B)$. $\checkmark$

\end{document}
