\documentclass[main.tex]{subfiles}

\begin{document}

\section{Aufgabe 7}
Gegeben ist das lineare Gleichungssystem $Ax=b$, mit
$$
    A = \eqmatrix{ccc}{
        3 & 2 & 1 \\
        2 & 3 & 1 \\
        1 & 2 & 3
    } \quad \text{und} \quad
    b=\vektor{39\\ 34\\ 26}.
$$

\begin{enumerate}
    \item Bringen Sie die Matrix $A$ auf die obere Dreiecksgestalt, indem Sie sie von links mit den für die Zeilenumformungen erforderlichen Elementarmatrizen multiplizieren.
    \item Multiplizieren Sie auch die rechte Seite $b$ von links mit diesen Elementarmatrizen.
    \item Lösen Sie anschließend das Gleichungssystem durch Rückwärtssubstitution.
\end{enumerate}

\subsection{Lösung 7}

% {{1,0,0},{0,1,0},{0,0,15/36}}
% {{1,0,0},{0,1,0},{0,-4/3,1}}
% {{1,0,0},{0,1,0},{-1,0,1}}
% {{1,0,0},{0,3/5,0},{0,0,1}}
% {{1,0,0},{-2,1,0},{0,0,1}}
% {{1/3,0,0},{0,1,0},{0,0,1}}
% A = {{3,2,1},{2,3,1},{1,2,3}}
% b = {{39},{34},{26}}
% C's = {{1/3,0,0},{-2/5,3/5,0},{1/12,-1/3,5/12}}
% A* =  {{1,2/3,1/3},{0,1,1/5},{0,0,1}}
% b* = {{13},{24/5},{11/4}}

\begin{align*}
    C3_{\lambda_{3,3}=\frac{15}{36}} \cdot
    C1_{\lambda_{3,2}=-\frac{4}{3}} \cdot
    C1_{\lambda_{3,1}=-1} \cdot
    C3_{\lambda_{2,2}=\frac{3}{5}} \cdot
    C1_{\lambda_{2,1}=-2} \cdot
    C3_{\lambda_{1,1}=\frac{1}{3}} \cdot
    A
    &=
    \eqmatrix{ccc}{
        1 & \frac{2}{3} & \frac{1}{3} \\
        0 &           1 & \frac{1}{5} \\
        0 &           0 & 1
    }\\[2mm]
    C3_{\lambda_{3,3}=\frac{15}{36}} \cdot
    C1_{\lambda_{3,2}=-\frac{4}{3}} \cdot
    C1_{\lambda_{3,1}=-1} \cdot
    C3_{\lambda_{2,2}=\frac{3}{5}} \cdot
    C1_{\lambda_{2,1}=-2} \cdot
    C3_{\lambda_{1,1}=\frac{1}{3}}
    &=
    \eqmatrix{ccc}{
        \frac{1}{3}  &            0 & 0 \\
        -\frac{2}{5} & \frac{3}{5}  & 0 \\
        \frac{1}{12} & -\frac{1}{3} & \frac{5}{12}
    }\\[2mm]
    \eqmatrix{ccc}{
        \frac{1}{3}  &            0 & 0 \\
        -\frac{2}{5} & \frac{3}{5}  & 0 \\
        \frac{1}{12} & -\frac{1}{3} & \frac{5}{12}
    } \cdot b
    &= \vektor{13\\ \nicefrac{24}{5} \\ \nicefrac{11}{4}}\\
\end{align*}

Durch Rückwärtssubstitution ergibt sich $x_3 = \frac{11}{4}$,
\begin{align*}
                      & x_2 + \frac{1}{5} x_3 = \frac{24}{5} \\
    \Leftrightarrow\  & x_2 = \frac{24}{5} - \frac{11}{20}\\
    \Leftrightarrow\  & x_2 = \frac{17}{4}\\
\end{align*}
und für $x_1$
\begin{align*}
                      & x_1 + \frac{2}{3} x_2 + \frac{1}{3} x_3 = 13 \\
    \Leftrightarrow\  & x_1 = 13 - \frac{45}{12} \\
    \Leftrightarrow\  & x_1 = \frac{37}{4}.
\end{align*}

% {{37/4},{17/4},{11/4}}
Somit erhalten wir $x = \frac{1}{4} \cdot \vektor{37\\ 17\\ 11}$.

\end{document}
