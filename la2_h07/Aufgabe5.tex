\documentclass[main.tex]{subfiles}

\begin{document}

\section{Aufgabe 5}
Sei $0 \neq v \in \mathbb{R}^n$ gegeben. Wir betrachten die Projektion $p$ von $x\in \mathbb{R}^n$ in Richtung $v$. Es gilt $p=\frac{\scalarprod{v,x}}{\norm{v}^2} \cdot v$, wobei $\scalarprod{\cdot, \cdot}$ das Standardskalarprodukt und $\norm{\cdot}$ die euklidische Norm bezeichne.

\begin{enumerate}
    \item Zeigen Sie, dass die Abbildung $P(x)=p$ linear ist.
    \item Bestimmen Sie die zugehörige Abbildungsmatrix $A$.
    \item Berechnen Sie $\Bild(P)$ und geben Sie eine Basis des Bildes an. Wie lautet $\rg(A)$?
    \item Bestimmen Sie $\ker(P)$ und deuten Sie $\ker(P)$ geometrisch.
    \item Geben Sie eine Basis von $\ker(P)$ an.
    \item Zeigen Sie, dass $P$ keine Umkehrabbildung besitzt.
\end{enumerate}

\subsection{Lösung 5a}
$$
    P: \begin{cases}
        \mathbb{R}^n \to \mathbb{R}^n \\
        x \mapsto \frac{\scalarprod{v,x}}{\norm{v}^2} \cdot v\\
    \end{cases}
$$

Die Abbildung $P(x)=p$ ist linear, genau dann wenn sie additiv und homogen ist.
Zeige Additivität mit $x,y\in \mathbb{R}^n$:
\begin{align*}
    P(x+y) &= \frac{\scalarprod{v,x+y}}{\norm{v}^2} \cdot v \\
    &= \frac{\scalarprod{v,x}+\scalarprod{v,y}}{\norm{v}^2} \cdot v \\
    &= \left(
        \frac{\scalarprod{v,x}}{\norm{v}^2} + \frac{\scalarprod{v,y}}{\norm{v}^2}
    \right) \cdot v \\
    &= \frac{\scalarprod{v,x}}{\norm{v}^2}\cdot v + \frac{\scalarprod{v,y}}{\norm{v}^2}\cdot v \\
    &= P(x) + P(y) \quad \checkmark
\end{align*}
Zeige Homogenität mit $\lambda \in \mathbb{R}$:
\begin{align*}
    P(\lambda \cdot x) &= \frac{\scalarprod{v,\lambda x}}{\norm{v}^2} \cdot v \\
                 &= \lambda \cdot \frac{\scalarprod{v,x}}{\norm{v}^2} \cdot v \\
                 &= \lambda \cdot P(x)  \quad \checkmark
\end{align*}

\subsection{Lösung 5b}
Es gilt
\begin{align*}
    P(x)
    &= \frac{\scalarprod{v,x}}{\norm{v}^2} \cdot v \\[2mm]
    &= \frac{v_1 x_1 + \dots + v_n x_n}{v_1^2 + \dots + v_n^2} \cdot v
\end{align*}

Um die Abbildungsmatrix $A$ bzgl. der kanonischen Einheitsvektoren zu erhalten, setzen wir diese der Reihe nach ein.

\begin{align*}
    A &= \eqmatrix{cccc}{
        P(e_1) & P(e_2) & \cdots & P(e_n)
    } \\[2mm]
    &=
    \eqmatrix{cccc}{
        \frac{v_1}{\norm{v}^2} \cdot v & \frac{v_2}{\norm{v}^2} \cdot v & \cdots & \frac{v_n}{\norm{v}^2} \cdot v
    } \\[2mm]
    &=
    \frac{1}{\norm{v}^2} \cdot
    \eqmatrix{cccc}{
        v_1 \cdot v & v_2 \cdot v & \cdots & v_n \cdot v
    } \\
    &=
    \frac{1}{\norm{v}^2} \cdot
    \eqmatrix{cccc}{
        v_1 v_1 & v_2 v_1 & \cdots & v_n v_1 \\
        v_1 v_2 & v_2 v_2 & \cdots & v_n v_2 \\
               \vdots &        \vdots & \ddots & \vdots \\
        v_1 v_n & v_2 v_n & \cdots & v_n v_n \\
    }
\end{align*}

\subsection{Lösung 5c}

Es gilt $\Bild (P) = \mathcal{L}\left( P(e_1), P(e_2), \dots, P(e_n) \right) = \mathcal{L} (A) = \mathcal{L} (v)$ und somit ist $(v)$ eine Basis von $\Bild (P)$ und $\rg (P) = \dim(\Bild(P)) = 1$.

\subsection{Lösung 5d}
$\ker (P) = \left\{x \middle| \scalarprod{x, v} = 0\right\}$ da die Projektion eines Vektors $x$ auf einen anderen Vektor $v$ nur dann $0$ ist, wenn $x\bot v$ also $\scalarprod{x,v} = 0$.

\subsection{Lösung 5e}
Für die Normalform gilt $\scalarprod{x,v}=\scalarprod{p,v}$.

\subsection{Lösung 5f}
Aus $\det(A) \neq 0$  folgt, dass $A$ nicht invertierbar ist.


\end{document}
