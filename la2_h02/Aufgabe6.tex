\documentclass[main.tex]{subfiles}

\begin{document}

\section{Aufgabe 6}
Welche der folgenden Abbildungen von $\mathbb{R}^2 \to \mathbb{R}^3$ sind linear? Geben Sie ggf. die zugehörige Matrix an.
Bestimmen Sie jeweils den Kern (auch für die nicht-linearen Abbildungen).

\begin{enumerate}
    \item $f_1(x_1, x_2) = \vektor{ -x_2 \\ -x_1 \\ 5x_1 - 7x_2 }$
    \item $f_2(x_1, x_2) = \vektor{ x_1 + 1 \\ x_2 - 1 \\ 3 }$
\end{enumerate}

\subsection{Lösung 6a}
Additivität, $x, y \in \R^2$
\begin{align*}
    f_1(x + y) &= \vektor{ -(x_2 +y_2) \\ -(x_1 + y_1) \\ 5(x_1+y_1) - 7(x_2 + y_2) } \\[2mm]
    &= \vektor{ -x_2 -y_2 \\ -x_1 -y_1 \\ 5x_1 - 7x_2 + 5y_1 - 7y_2 } \\[2mm]
    &= \vektor{ -x_2 \\ -x_1 \\ 5x_1 - 7x_2 } + \vektor{ -y_2 \\ -y_1 \\[2mm] 5y_1 - 7y_2 } \\[2mm]
    &= f_1(x) + f_1(y) \quad \checkmark
\end{align*}

Homogenität, $x \in \R^2, \lambda \in \R$
\begin{align*}
    \lambda \cdot f_1(x) & = \lambda \cdot \vektor{ -x_2 \\ -x_1 \\ 5x_1 - 7x_2 } \\[2mm]
    & = \vektor{ -\lambda x_2 \\ -\lambda x_1 \\ \lambda (5x_1 - 7x_2) } \\[2mm]
    &= \vektor{-\lambda x_2 \\ -\lambda x_1 \\ 5\lambda x_1 - 7\lambda x_2}\\[2mm]
    &= f_1(\lambda \cdot x) \quad \checkmark
\end{align*}

Die Spalten der Abbildungsmatrix einer Funktion sind die Bilder der kanonischen Einheitsvektoren.
\arraycolsep=0.8em\def\arraystretch{1}
\begin{align*}
    A_{f_1} &= \left(
        f_1(e_1) \quad f_1(e_2)
    \right) \\
    &= \left(
        f_1\left(\vektor{1 \\ 0}\right) \quad f_1\left(\vektor{0 \\ 1}\right)
    \right) \\
    &= \eqmatrix{rr}{
         0 & -1 \\
        -1 &  0 \\
         5 & -7
    } \\
\end{align*}

Der Kern der Abbildung ist $\ker (f_1) = \{\vec{0}\}$, da lediglich der Nullvektor aus $\R^2$ auf den Nullvektor im Bild der Abbildung zeigt.

\subsection{Lösung 6b}
Untersuche Additivität, $x, y \in \R^2$
\begin{align*}
    f_2(x + y) &= \vektor{ x_1 +y_1 + 1 \\ x_2 + y_2 - 1 \\ 3 } \\
    &\neq \vektor{ x_1 + y_2 + 2 \\ x_2 + y_2 - 2 \\ 6}\\
    &= \vektor{ x_1 + 1 \\ x_2 - 1 \\ 3 } + \vektor{ y_1 + 1 \\ y_2 - 1 \\ 3 }\\[2mm]
    &= f_2(x) + f_2(y)
\end{align*}
Die Abbildung $f_2$ ist nicht additiv, also handelt es sich nicht um eine lineare Abbildung, also existiert keine auch Abbildungsmatrix.
Der Kern der Abbildung ist die leere Menge, da der Nullvektor nicht im Bild der Abbildung liegt. 

\end{document}
