\documentclass[main.tex]{subfiles}

\begin{document}

\section{Aufgabe 7}
Es sei $\lambda \in \mathbb{R}$ und $x \in \mathbb{R}^n$. Der Ausdruck $\lambda x$ kann als lineare Abbildung interpretiert werden:
\begin{enumerate}
    \item $\mathbb{R}^n \to \mathbb{R}^n : x \mapsto \lambda x$
    \item $\mathbb{R}^1 \to \mathbb{R}^n : \lambda \mapsto \lambda x$
\end{enumerate}

Wie lauten in jedem Fall die Matrizen der zugehörigen Abbildungen? Im Fall a) kann damit die Multiplikation eines Vektors mit einem Faktor als Matrixmultiplikation interpretiert werden.

\subsection{Lösung 7a}
Für die Abbildung 
$$
    f_a: \begin{cases}
        \R^n \to R^n\\
        x \mapsto \lambda x & \lambda \in \R
    \end{cases}
$$
ist die Abbildungsmatrix $A_{f_a} = \lambda E_n$, also das $\lambda$-fache der $n\times n$-Einheitsmatrix.
\arraycolsep=0.6em\def\arraystretch{1}
$$
A_{f_a} = \eqmatrix{cccc}{
    \lambda &      0 & \cdots & 0 \\
          0 & \ddots &      0 & \vdots \\
     \vdots &      0 & \ddots & 0 \\
          0 & \cdots & 0 & \lambda}
$$

\subsection{Lösung 7b}
\textit{Hinweis:} Bei $\lambda$ handelt es sich um eine Variable, während $x$ hier ein Parameter ist. \\
Für die Abbildung 
$$
    f_b: \begin{cases}
        \R \to \R^n\\
        \lambda \mapsto \lambda x & x \in \R^n
    \end{cases}
$$
hat die Abbildungsmatrix die Gestalt $A_{f_b} \in K^n\times 1$. Es handelt sich somit um einen Vektor, welcher durch Multiplikation mit $\lambda$, $\lambda$ in $f_b$ abbildet.
$$
    A_{f_b} = \vec{x} = \vektor{x_1 \\ \vdots \\ x_n}
$$

\end{document}
