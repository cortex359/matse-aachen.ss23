\documentclass[main.tex]{subfiles}

\begin{document}

\section{Aufgabe 8}
Gegeben sei die Abbildung $f : \mathbb{R}^3 \to \mathbb{R}^2$ mit $f(x_1, x_2, x_3) = (x_1 - 2x_3, 4x_2)$.

\begin{enumerate}
    \item Zeigen Sie, dass $f$ linear ist.
    \item Bestimmen Sie den $\ker(f)$ und geben Sie die $\dim (\ker (f))$ an.
    \item Berechnen Sie die $\dim(\Bild(f))$ bzw. $\rang(f)$ und bestimmen Sie das $\Bild(f)$.
    \item Ist die Abbildung $f$ injektiv oder surjektiv?
\end{enumerate}

\subsection{Lösung 8a}
Additivität:
\begin{equation*}
    \begin{array}{ r r l }
    & f(x_1, x_2, x_3) & = (x_1 - 2x_3, 4x_2)\\
    & f(x_1 + y_1, x_2 + y_2, x_3 + y_3)  & = (x_1 + y_1 - 2(x_3 + y_3), 4(x_2 + y_2)) \\
    & f(x_1, x_2, x_3) + f(y_1, y_2, y_3) & = (x_1 - 2x_3, 4x_2) + (y_1 - 2y_3, 4y_2) \\
    & & = (x_1 - 2x_3 + y_1 - 2y_3, 4x_2 + 4y_2) \\
    & & = (x_1 + y_1 - 2(x_3 + y_3 ), 4(x_2 + y_2)) \\
    \Rightarrow  & f(x_1, x_2, x_3) + f(y_1, y_2, y_3) & = f(x_1 + y_1, x_2 + y_2, x_3 + y_3) \quad \checked
    \end{array}
\end{equation*}

Homogenität:
\begin{equation*}
    \begin{array}{ r r l }
    & f(\lambda x_1, \lambda x_2, \lambda x_3) & = (\lambda \cdot x_1 - 2\cdot \lambda \cdot x_3, 4 \cdot \lambda \cdot x_2)\\
    & & = (\lambda \cdot (x_1 - 2 \cdot x_3), 4 \cdot \lambda \cdot x_2)\\
    & \lambda f(x_1, x_2, x_3) & = \lambda \cdot (x_1 - 2x_3, 4x_2)\\
    & & = (\lambda \cdot (x_1 - 2x_3), 4 \cdot \lambda \cdot x_2)\\
    \Rightarrow  & f( \lambda x_{1} ,\lambda x_{2}) & =\lambda f( x_{1} ,x_{2}) \quad \checked
    \end{array}
\end{equation*}
Die Abbildung ist linear.

\subsection{Lösung 8b}
Die Menge der Vektoren aus dem Definitionsbereich, welche auf den Nullvektor im Wertebereich abbilden, ermitteln wir durch Betrachtung der Abbildung. Es fällt auf, dass $x_2 = 0$ sein muss und dass $x_1-2x_3 = 0$, also $x_1 = 2x_3$ sein muss. Wählen $x_3=\lambda \in \R$ beliebig:
$$
    \ker(f) = \left\{ \lambda \cdot \vektor{2 \\ 0 \\ 1} \middle| \lambda \in \R \right\}
$$
Die Dimension ist die Anzahl der Vektoren in einem minimal Erzeugendensystem, welche wir aus der obrigen Darstellung einfach abzählen können. $\dim(\ker(f)) = 1$.

\subsection{Lösung 8c}
Aus der Rangformel $\text{def}(f) + \text{rg}(f) = \dim(V)$ können wir mit dem zuvor ermittelten $\text{def}(f) + \dim(\ker(f)) = 1$ leicht erkennen, dass $\dim(\Bild(f)) = \text{rg}(f) = 2$ ist.

Außerdem ist $\Bild(f) = \left\{y\in \R^2 \middle| y=\alpha \vektor{1\\0} + \beta \vektor{0\\1},\; \alpha,\beta \in \R \right\} = \R^2$

\subsection{Lösung 8d}
Die Abbildung $f$ ist surjektiv, wie in Aufgabenteil c gezeigt, aber nicht injektv, wie in Teil b gezeigt.

\end{document}
