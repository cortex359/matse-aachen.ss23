\documentclass[main.tex]{subfiles}

\begin{document}

\section{Aufgabe 5}
Gegeben seien
$$
    A = \eqmatrix{rrr}{
        1 & 0 & 1 \\
        c & 1 & -1 \\
        0 & 2 & -1}
    \qquad
    b = \vektor{-1 \\ d \\ 1}.
$$

Bestimmen Sie die Werte für $c$ und $d$, für die das lineare Gleichungssystem $Ax=b$
\begin{enumerate}
    \item genau eine Lösung
    \item keine Lösung
    \item unendlich viele Lösungen
\end{enumerate}
hat.

\subsection{Lösung 5a}
\begin{align*}
		\begin{array}{rrl}
			& |A| & = \begin{vmatrix}
				1 & 0 & 1\\
				c & 1 & -1\\
				0 & 2 & -1
			\end{vmatrix}\\[7mm]
			& & = -1+2c+2\\
			& & = 2c+1\\
			\Rightarrow & c & = -\frac{1}{2}
		\end{array}
	\end{align*}
	$\det(A)=0$ für $c=-\frac{1}{2}\Rightarrow \rg(A)=2$.\vspace{5mm}\\
	 Genau eine Lösung gibt es somit für $c\neq -\frac{1}{2}$ und $d \in \R$, da dann $|A| \neq 0$ gilt und $b$ somit für beliebiges $d$ als Linearkombination der Spaltenvektoren von $A$ darstellbar ist.

\subsection{Lösung 5b}
$|A,b|$ für $c=-\frac{1}{2}$ :
\begin{align*}
    \begin{array}{rrl}
        & |A,b| & = \begin{vmatrix}
            1 & 0 & 1 & -1\\
            -\frac{1}{2} & 1 & -1 & d\\
            0 & 2 & -1 & 1
        \end{vmatrix}\\[7mm]
        & & = \begin{vmatrix}
            0 & 1 & -1\\
            1 & -1 & d\\
            2 & -1 & 1
        \end{vmatrix}\\[7mm]
        & & = 1 + 2d - 1 - 2\\
        & & = 2d - 2\\
        \Rightarrow & |A,b| & = 0 \text{ für } d = 1 \text{ und } c = -\frac{1}{2}
    \end{array}
\end{align*}
Keine Lösung gibt es somit für $c = -\frac{1}{2}$ und $d \neq 1$, da $\rg(A) = 2 \neq 3 = \rg(A,b)$

\subsection{Lösung 5c}
Unendliche viele Lösungen gibt es somit für $c = -\frac{1}{2}$ und $d = 1$, da dann \\
	$\rg(A) = 2 = \rg(A,b)$ und $|A| = 0$ gilt.


\end{document}
