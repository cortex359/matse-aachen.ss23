\documentclass[main.tex]{subfiles}

\begin{document}

\section{Aufgabe 7}
Untersuchen Sie die Lösungsmenge des linearen Gleichungssystems $a\times x = b$ für gegebene Vektoren $a$ und $b$ des $\mathbb{R}^3$. Gehen Sie dabei wie folgt vor:
\begin{enumerate}
    \item Stellen Sie die zugehörige Abbildungsmatrix $A$ auf.
    \item Bestimmen Sie alle Lösungen $z$ mit $Az=0$.
    \item Bestimmen Sie den Wert für $c$, für den das LGS $a\times x =b$ mit
          $$
            a = \vektor{4 \\ 3 \\ 1} \qquad \text{und} \qquad b = \vektor{8 \\ c \\ -2}
          $$
          lösbar ist.
    \item Bestimmen Sie die Lösungsmenge für diesen konkreten Fall.
\end{enumerate}

\subsection{Lösung 7}
Das lineare Gleichungssystem $a\times x = b$ entspricht dem Vektorprodukt
\begin{align*}
	\begin{pmatrix}
		a_2x_3-a_3x_2\\
		a_3x_1-a_1x_3\\
		a_1x_2-a_2x_1
	\end{pmatrix} = b
\end{align*}

\subsubsection{Lösung 7a}
\begin{align*}
		A = \begin{pmatrix}
			0 & -a_3 & a_2\\
			a_3 & 0 & -a_1\\
			-a_2 & a_1 & 0
		\end{pmatrix}
	\end{align*}
\subsubsection{Lösung 7b}

Wie man anhand der Abbildungsmatrix erkennen kann, sowie an der Ausformulierung des Vektorprodukts, muss $x$ bei beliebigen $a$ dem Nullvektor entsprechen. Somit ist $\ker(a\times x)=0$, bzw. $z = 0$ für $Az=0$

\subsubsection{Lösung 7c}
\begin{align*}
		\begin{array}{rl}
			&\begin{vmatrix}
				0 & -1 & 3 & 8\\
				1 & 0 & -4 & c\\
				-3 & 4 & 0 & -2
			\end{vmatrix}\\[7mm]
			\equiv & \begin{vmatrix}
				0 & -1 & 3 & 8\\
				1 & 0 & -4 & c\\
				0 & 4 & -12 & -2+3c
			\end{vmatrix}\\[7mm]
			\equiv & \begin{vmatrix}
				0 & -4 & 12 & 32\\
				1 & 0 & -4 & c\\
				0 & -4 & 12 & 2-3c
			\end{vmatrix}\\[7mm]
			\equiv & \begin{vmatrix}
				0 & -1 & 3 & 8\\
				1 & 0 & -4 & c\\
				0 & 0 & 0 & 30-3c
			\end{vmatrix}\\[7mm]
		\end{array}
	\end{align*}
	$\Rightarrow$ Für $c=-10$ ist das Gleichungssystem lösbar und besitzt unendlich Lösungen mit einem Parameter.

\subsubsection{Lösung 7d}
  \begin{align*}
		\begin{array}{rrl}
			& x_3 & = \frac{1}{2}x_1+\frac{5}{2}\\
			\Rightarrow & x_2 & = \frac{3}{2}x_1 - \frac{1}{2}
		\end{array}
	\end{align*}
	Somit lautet die Lösungsmenge:
\begin{align*}
	L &=\left\{x \in \R^3 \middle| x = \frac{1}{2} x_1\cdot
	\begin{pmatrix}
		2\\
		3\\
		1
	\end{pmatrix} + \frac{1}{2} \cdot
	\begin{pmatrix}
		0\\
		-1\\
		5
	\end{pmatrix}\right\}
\end{align*}

\end{document}
