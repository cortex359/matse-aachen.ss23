\documentclass[main.tex]{subfiles}

\begin{document}

\section{Aufgabe 8}
Ein:e Unternehmer:in legt folgende Umsatzermittlungstabelle dem Finanzamt vor:

\begin{center}        
    \begin{tabular}{ p{2cm}|c|c|c }
        & Produkt 1 & Produkt 2 & Umsatz [Euro] \\
        \hline
        Januar & 10 & 20 & 70 \\
        Februar & 20 & 10 & 80 \\
        März & 15 & 15 & 50 \\
    \end{tabular}
\end{center}

\begin{enumerate}
    \item Begründen Sie, warum der Beamte/die Beamtin an der Korrektheit der Zahlen zweifelt.
    \item Korrigieren Sie den Umsatz im März so, dass das Finanzamt nicht misstrauisch wird. 
\end{enumerate}

\subsection{Lösung 8}
Aus den Angaben von Januar und Februar lässt sich ein LGS aufstellen, dessen Lösung anzeigt, dass der Handel mit Produkt 1 zu einem Umsatz von 3 Euro pro Stück und der Handel mit Produkt 2 zu einem Umsatz von 2 Euro pro Stück führt. \\

Überträgt man dies auf die Angaben vom März, so wäre ein Umsatz von $3\cdot 15 + 2\cdot 30 = 75$ Euro zu erwarten. Da der gemeldete Umsatz gerade mal zwei Drittel der erwarteten Zahl beträgt, bleibt der/die Unternehmende eine Erklärung dafür schuldig, warum der erzielte Umsatz pro gehandeltem Produkt sich zwischen Februar und März derart signifikat verändert hat.\\

Statt den Umsatz im März in den Bereich von 75 Euro (abzüglich einer betriebswirtschaftlich glaubhaft erklärbaren Schwankung der Umsatzrate) zu korrigieren, wäre es natürlich sinnvoller gewesen die Umsätze bereits im Januar mit entsprechender "Korrektur" glaubhaft zu machen.

\end{document}
