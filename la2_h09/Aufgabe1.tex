\documentclass[main.tex]{subfiles}

\begin{document}

\section{Aufgabe 1}
Weisen Sie folgende Gleichung nach:

$$
    \begin{vmatrix}
        1 & a & a^2 & a^3 \\
        1 & b & b^2 & b^3 \\
        1 & c & c^2 & c^3 \\
        1 & d & d^2 & d^3 \\
    \end{vmatrix}
    = (d-a)(c-a)(b-a)(d-b)(c-b)(d-c)
$$

\subsection{Lösung 1}
Lösung mit Laplace, Spalte 1:

\begin{align*}
    \begin{vmatrix}
        1 & a & a^2 & a^3 \\
        1 & b & b^2 & b^3 \\
        1 & c & c^2 & c^3 \\
        1 & d & d^2 & d^3 \\
    \end{vmatrix} 
    =&
    \begin{vmatrix}
        b & b^2 & b^3 \\
        c & c^2 & c^3 \\
        d & d^2 & d^3 \\
    \end{vmatrix}
    -
    \begin{vmatrix}
        a & a^2 & a^3 \\
        c & c^2 & c^3 \\
        d & d^2 & d^3 \\
    \end{vmatrix}
    +
    \begin{vmatrix}
        a & a^2 & a^3 \\
        b & b^2 & b^3 \\
        d & d^2 & d^3 \\
    \end{vmatrix}
    -
    \begin{vmatrix}
        a & a^2 & a^3 \\
        b & b^2 & b^3 \\
        c & c^2 & c^3 \\
    \end{vmatrix}\\
% 2x2 Matrizen:
    =&\ 
    b \cdot \textcolor{OliveGreen}{\begin{vmatrix}
        c^2 & c^3 \\
        d^2 & d^3 \\
    \end{vmatrix}}
    - c \cdot
    \textcolor{Cerulean}{\begin{vmatrix}
        b^2 & b^3 \\
        d^2 & d^3 \\
    \end{vmatrix}}
    + d \cdot
    \textcolor{BrickRed}{\begin{vmatrix}
        b^2 & b^3 \\
        c^2 & c^3 \\
    \end{vmatrix}}\\
%
    -&\ 
    a \cdot \textcolor{OliveGreen}{\begin{vmatrix}
        c^2 & c^3 \\
        d^2 & d^3 \\
    \end{vmatrix}}
    - c \cdot
    \textcolor{Goldenrod}{\begin{vmatrix}
        a^2 & a^3 \\
        d^2 & d^3 \\
    \end{vmatrix}}
    + d \cdot
    \textcolor{Melon}{\begin{vmatrix}
        a^2 & a^3 \\
        c^2 & c^3 \\
    \end{vmatrix}}\\
%
    +&\ 
    a \cdot \textcolor{Cerulean}{\begin{vmatrix}
        b^2 & b^3 \\
        d^2 & d^3 \\
    \end{vmatrix}}
    - b \cdot
    \textcolor{Goldenrod}{\begin{vmatrix}
        a^2 & a^3 \\
        d^2 & d^3 \\
    \end{vmatrix}}
    + d \cdot
    \begin{vmatrix}
        a^2 & a^3 \\
        b^2 & b^3 \\
    \end{vmatrix}\\
%
    -&\ 
    a \cdot \textcolor{BrickRed}{\begin{vmatrix}
        b^2 & b^3 \\
        c^2 & c^3 \\
    \end{vmatrix}}
    - b \cdot
    \textcolor{Melon}{\begin{vmatrix}
        a^2 & a^3 \\
        c^2 & c^3 \\
    \end{vmatrix}}
    + c \cdot
    \begin{vmatrix}
        a^2 & a^3 \\
        b^2 & b^3 \\
    \end{vmatrix}\\
% Matrizen ausklammern:
    =& 
    \left( b - a \right) \cdot \textcolor{OliveGreen}{\begin{vmatrix}
        c^2 & c^3 \\
        d^2 & d^3 \\
    \end{vmatrix}} 
    + \left( a - c \right) \cdot
    \textcolor{Cerulean}{\begin{vmatrix}
        b^2 & b^3 \\
        d^2 & d^3 \\
    \end{vmatrix}}
    + \left( d - a \right) \cdot
    \textcolor{BrickRed}{\begin{vmatrix}
        b^2 & b^3 \\
        c^2 & c^3 \\
    \end{vmatrix}}\\
    -& \left( b + c \right) \cdot
    \textcolor{Goldenrod}{\begin{vmatrix}
        a^2 & a^3 \\
        d^2 & d^3 \\
    \end{vmatrix}}
    + \left( d - b \right) \cdot
    \textcolor{Melon}{\begin{vmatrix}
        a^2 & a^3 \\
        c^2 & c^3 \\
    \end{vmatrix}}
    + \left( d + c \right) \cdot
    \begin{vmatrix}
        a^2 & a^3 \\
        b^2 & b^3 \\
    \end{vmatrix}\\
    \\
% Quadrate rausziehen:
    =& 
    \left( b - a \right) \cdot c^2 d^2 \cdot \textcolor{OliveGreen}{\begin{vmatrix}
        1 & c \\
        1 & d \\
    \end{vmatrix}} 
    + \left( a - c \right) \cdot b^2 d^2 \cdot
    \textcolor{Cerulean}{\begin{vmatrix}
        1 & b \\
        1 & d \\
    \end{vmatrix}}
    + \left( d - a \right) \cdot b^2 c^2 \cdot
    \textcolor{BrickRed}{\begin{vmatrix}
        1 & b \\
        1 & c \\
    \end{vmatrix}}\\
    -& \left( b + c \right) \cdot a^2 d^2 \cdot
    \textcolor{Goldenrod}{\begin{vmatrix}
        1 & a \\
        1 & d \\
    \end{vmatrix}}
    + \left( d - b \right) \cdot a^2 c^2 \cdot
    \textcolor{Melon}{\begin{vmatrix}
        1 & a \\
        1 & c \\
    \end{vmatrix}}
    + \left( d + c \right) \cdot a^2 b^2 \cdot
    \begin{vmatrix}
        1 & a \\
        1 & b \\
    \end{vmatrix}\\
    \\
% Triviale Determinanten lösen:
    =& 
    \left( b - a \right) \cdot c^2 d^2 \cdot 
    \textcolor{OliveGreen}{(d - c)} 
    + \left( a - c \right) \cdot b^2 d^2 \cdot
    \textcolor{Cerulean}{(d-b)}
    + \left( d - a \right) \cdot b^2 c^2 \cdot
    \textcolor{BrickRed}{(c-b)}\\
    -& \left( b + c \right) \cdot a^2 d^2 \cdot
    \textcolor{Goldenrod}{(d-a)}
    + \left( d - b \right) \cdot a^2 c^2 \cdot
    \textcolor{Melon}{(c-a)}
    + \left( d + c \right) \cdot a^2 b^2 \cdot
    (b-a)\\
    \\
%
%
=&\  b c^2 d^3 + b^2 c^3 d + b^3 c d^2 - b^3 c^2 d - b c^3 d^2 - b^2 c d^3 \\
     & - \left( a c^2 d^3 + a^2 c^3 d + a^3 c d^2 - a^3 c^2 d - a c^3 d^2 - a^2 c d^3 \right) \\
     & + a b^2 d^3 + a^2 b^3 d + a^3 b d^2 - a^3 b^2 d - a b^3 d^2 - a^2 b d^3 \\
     & - \left( a b^2 c^3 + a^2 b^3 c + a^3 b c^2 - a^3 b^2 c - a b^3 c^2 - a^2 b c^3 \right) \\
    =&\  b c^2 d^3 + b^2 c^3 d + b^3 c d^2 - b^3 c^2 d - b c^3 d^2 - b^2 c d^3 \\
     & - a c^2 d^3 - a^2 c^3 d - a^3 c d^2 + a^3 c^2 d + a c^3 d^2 + a^2 c d^3 \\
     & + a b^2 d^3 + a^2 b^3 d + a^3 b d^2 - a^3 b^2 d - a b^3 d^2 - a^2 b d^3 \\
     & - a b^2 c^3 - a^2 b^3 c - a^3 b c^2 + a^3 b^2 c + a b^3 c^2 + a^2 b c^3 \\
    %=&
    % bcd \cdot \left(  c d^2 + b c^2 + b^2 d - b^2 c - c^2 d - b d^2 \right) \\
    % acd \cdot \left(- c d^2 - a c^2 - a^2 d + a^2 c + c^2 d + a d^2 \right) \\
    % abd \cdot \left(  b d^2 + a b^2 + a^2 d - a^2 b - b^2 d - a d^2 \right) \\
    % abc \cdot \left(- b c^2 - a b^2 - a^2 c + a^2 b + b^2 c + a c^2 \right) \\
    =& \textcolor{OliveGreen}{\left( dc -da -ac +a^2 \right)} 
    \textcolor{Cerulean}{\left( bd -b^2 -ad +ab \right)} 
    \textcolor{BrickRed}{\left( cd - c^2 -bd +bc \right)} 
    \\
    =& \textcolor{OliveGreen}{(d-a)(c-a)}
    \textcolor{Cerulean}{(b-a)(d-b)}
    \textcolor{BrickRed}{(c-b)(d-c)} \\
\end{align*}


\subsection{Lösung 1 Variante 2}
Da $\det (A) = \det (A^T)$ betachten wir $A^T$ und ziehen das $a^{(i-1)}$-fache der ersten Zeile von den übrigen Zeilen ab und lösen mit Laplace:
\begin{align*}
    A^T &=
    \begin{vmatrix}
        1 & 1 & 1 & 1 \\
        a & b & c & d \\
        a^2 & b^2 & c^2 & d^2 \\
        a^3 & b^3 & c^3 & d^3 \\
    \end{vmatrix} \\[2mm]
    &=
    \begin{vmatrix}
        1 & 1 & 1 & 1 \\
        0 & b-a & c-a & d-a \\
        0 & b^2-a^2 & c^2-a^2 & d^2-a^2 \\
        0 & b^3-a^3 & c^3-a^3 & d^3-a^3 \\
    \end{vmatrix} \\[2mm]
    &=
    1 \cdot 
    \begin{vmatrix}
        b-a & c-a & d-a \\
        b^2-a^2 & c^2-a^2 & d^2-a^2 \\
        b^3-a^3 & c^3-a^3 & d^3-a^3 \\
    \end{vmatrix} \\[2mm]
    &= % wir ziehen (b-a) aus der ersten Spalte raus und erhalten mit binomischen formeln:
    (b-a) \cdot 
    \begin{vmatrix}
        1 & c-a & d-a \\
        a+b & c^2-a^2 & d^2-a^2 \\
        (a+b)^2 & c^3-a^3 & d^3-a^3 \\
    \end{vmatrix} \\[2mm]
    &= \underbrace{\textcolor{Cerulean}{
    {(b-a) (c-a) (d-a)}}}_{:=p_1} \cdot 
    \begin{vmatrix}
              1 &       1 & 1 \\
            a+b &     a+c & a+d \\
        a^2+ab+b^2 & a^2+ac+c^2 & a^2+ad +d^2 \\
    \end{vmatrix} \\[2mm]
    &= p_1 \cdot 
        \begin{vmatrix}
            1 &               1 & 1 \\
            0 &       a+c-(a+b) & a+d-(a+b) \\
            0 & (a+c)^2-(a+b)^2 & (a+d)^2-(a+b)^2 \\
        \end{vmatrix} \\[2mm]
    &= p_1 \cdot 
        \begin{vmatrix}
            1 &               1 & 1 \\
            0 &             c-b & d-b \\
            0 & (a+c)^2-(a+b)^2 & (a+d)^2-(a+b)^2 \\
        \end{vmatrix} \\[2mm]
    &= p_1 \cdot 
        \begin{vmatrix}
                        c-b & d-b \\
            (a+c)^2-(a+b)^2 & (a+d)^2-(a+b)^2 \\
        \end{vmatrix} \\[2mm]
    &= p_1 \cdot 
    \left( (c-b)\left((a+d)^2-(a+b)^2\right) - (d-b)\left((a+c)^2-(a+b)^2\right) \right)\\
    &= p_1 \cdot 
    \left( (c-b)\left(
        2ad +d^2  -2ab -b^2
    \right) - (d-b)\left(
        2ac +c^2  -2ab -b^2
    \right) \right)\\
    &= p_1 \cdot 
    \left( (c-b)\left(
        2a\cdot (d-b) + \textcolor{Cerulean}{d^2 - b^2}
    \right) - (d-b)\left(
        2a \cdot(c-b) + \textcolor{ForestGreen}{c^2 - b^2}
    \right) \right)\\
    &= % Binomische Formeln
    p_1 \cdot 
    \left( (c-b)\left(
        2a\cdot (d-b) + \textcolor{Cerulean}{(d-b)(d+b)}
    \right) - (d-b)\left(
        2a \cdot(c-b) + \textcolor{ForestGreen}{(c-b)(c+b)}
    \right) \right)\\
    &= % Ausklammern
    p_1 \cdot 
    \left( (c-b)
        (d-b) \cdot (2a + (d+b))
    - (d-b)
        (c-b) \cdot (2a + (c+b))
    \right)\\
    &= % (d-b) ausklammern
    p_1 \cdot 
    \left( (c-b)(d-b) \left(
         (2a+b+d) - (2a+b+c)
    \right) \right)\\
    &= % (d-b) ausklammern
    p_1 \cdot (c-b)(d-b)(d - c) \\
    &= \textcolor{Cerulean}{(d-a)(c-a)(b-a)} \cdot (d-b)(c-b)(d-c)
\end{align*}


\end{document}
