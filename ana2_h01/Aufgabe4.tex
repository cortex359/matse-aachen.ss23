\documentclass[main.tex]{subfiles}

\begin{document}

\section{Aufgabe 4}
Durch welche der folgenden Funktionen werden Skalarprodukte auf $\mathbb{R}^{2}$ definiert?

\begin{enumerate}
    \item $\left(\vec{x} ,\vec{y}\right) =x_{1}^{2} +x_{2} \cdotp y_{1} \cdotp y_{2}$
    \item $\left(\vec{x} ,\vec{y}\right) =x_{1} \cdotp y_{1} +x_{1} \cdotp y_{2} +x_{2} \cdot y_{1} +x_{2} \cdot y_{2}$
\end{enumerate}

\subsection{Lösung 4a}
$\left(\vec{x} ,\vec{y}\right)$ definiert kein Skalarprodukt, da $\left(\vec{x} ,\vec{y}\right) \neq \left(\vec{y} ,\vec{x}\right)$:
\begin{equation*}
    x_{1}^{2} +x_{2} \cdotp y_{1} \cdotp y_{2} \neq y_{1}^{2} +y_{2} \cdotp x_{1} \cdotp x_{2}
\end{equation*}

\subsection{Lösung 4b}
$\left(\vec{x} ,\vec{y}\right)$ definiert kein Skalarprodukt, die Bedingung 5 $\left(\vec{x},\vec{x}\right) = 0 \Leftrightarrow \vec{x} = \vec{0}$ verletzt ist:
\begin{equation*}
    \begin{array}{ c r l c }
    & \left(\vec{x} ,\vec{x}\right) & =x_{1} \cdotp x_{1} +x_{1} \cdotp x_{2} +x_{2} \cdot x_{1} +x_{2} \cdot x_{2} & \\
    &  & =x_{1}^{2} +2x_{1} x_{2} +x_{2}^{2} & \\
    &  & =( x_{1} +x_{2})^{2} & \\
    & \left(\vec{x} ,\vec{x}\right) & \eqdef 0 & \\
    \Leftrightarrow  & 0 & =( x_{1} +x_{2})^{2} & \\
    \Rightarrow  &  & x_{1} =-x_{2} \ \lor \ x_{2} =-x_{1} &
    \end{array}
\end{equation*}
$\left(\vec{x} ,\vec{y}\right)$ definiert kein Skalarprodukt, da z.B. für $\vec{a} =(1,1)^{T}$ mit $\left(\vec{a} ,\vec{a}\right) =4 \neq 0$ \ gilt.

\end{document}
