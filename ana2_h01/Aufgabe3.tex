\documentclass[main.tex]{subfiles}

\begin{document}

\section{Aufgabe 3}

Sei der Vektorraum $V$ und $d:V\times V\rightarrow \mathbb{R}$ gegeben durch
\begin{equation*}
    d(\vec{x} ,\vec{y}) =\begin{cases}
    0 & \text{für} \ x=y\\
    1 & \text{für} \ x\neq y
    \end{cases}
\end{equation*}

Zeige, dass $d$ eine Metrik in $V$ ist.

\subsection{Lösung 3}
Die Abbildung $d$ ist eine Metrik, da sowohl
\begin{equation*}
    d(\vec{x} ,\vec{y}) =0\Leftrightarrow \vec{x} =\vec{y} \ \checked
\end{equation*}
gilt, als auch die Dreiecksungleichung
\begin{equation*}
    d(\vec{x} ,\vec{y}) \leq d(\vec{x} ,\vec{z}) +d(\vec{y} ,\vec{z}) \ \ \ \forall \vec{x} ,\vec{y} ,\vec{z} \in V.
\end{equation*}



\end{document}
