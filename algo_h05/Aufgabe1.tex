\documentclass[main.tex]{subfiles}

\begin{document}

\section{Aufgabe 1}
Minimieren Sie den nachfolgenden DEA unter Verwendung eines in der Vorlesung vorgestellten Verfahrens:

\usetikzlibrary{arrows}
\usetikzlibrary{automata}
\usetikzlibrary{positioning}

\begin{center}
\begin{tikzpicture}[->, >=stealth', shorten >=1pt,
    node distance = 2 cm ]
\node [state, initial] (q0) {$q_0$};
\node [state] (q1) [right= of q0 ] {$q_1$};
\node [state] (q2) [right= of q1 ] {$q_2$};
\node [state] (q4) [below=10mm of q2] {$q_4$};

\node [state] (q6) [above left=15mm and 10mm of q1] {$q_6$};
\node [state] (q7) [above right=15mm and 10mm of q1] {$q_7$};
\node [state] (q8) [above=35mm of q1] {$q_8$};


\node [state, accepting ] (q3) [right= of q4] {$q_3$};
\node [state, accepting ] (q5) [above right=12mm and 12mm of q3] {$q_5$};

\path (q0) edge[below] node {0} (q1)
           edge[bend right=45, below] node {1} (q3)
      (q1) edge[left] node {0} (q6)
           edge[right] node {1} (q7)
      (q2) edge[below] node {0} (q1) 
           edge[right] node {1} (q4)
      (q3) edge[below right] node {0, 1} (q5)
      (q4) edge[above] node {0, 1} (q3)
      (q5) edge[loop above, above] node {0, 1} ()
      (q6) edge[bend left, above] node {0} (q7)
           edge[left] node {1} (q8)
      (q7) edge[bend left, below] node {0} (q6)
           edge[right] node {1} (q8)
      (q8) edge[bend left=45, above right] node {0} (q3)
           edge[bend right=45, left] node {1} (q0);
\end{tikzpicture}
\end{center}

\subsection{Lösung 1}

\textbf{Schritt 1} Streichen von unerreichbaren Zuständen.

Der Zustand $q_2$ ist nicht erreichbar und kann damit gestrichen werden.\\

\textbf{Schritt 2} Wir fassen die nicht-akzeptierenden Zustände in die mögliche Äquivalenzklasse [1] zusammen und die akzeptierenden Zustände in die mögliche Äquivalenzklasse [2].\\

\begin{table}[ht]
\centering
\begin{tabular}{r|r|r|r}
     Eq & Q & $\vdash 0$ & $\vdash 1$ \\
     \hline
                         & \cellcolor{yellow!20} $q_0$ & \cellcolor{yellow!20} [1] & \cellcolor{yellow!20} [2] \\
                         & \cellcolor{blue!20} $q_1$ & \cellcolor{blue!20} [1] & \cellcolor{blue!20} [1] \\    
                         & \cellcolor{orange!20} $q_4$ & \cellcolor{orange!20} [2] & \cellcolor{orange!20} [2] \\    
                         & \cellcolor{blue!20} $q_6$ & \cellcolor{blue!20} [1] & \cellcolor{blue!20} [1] \\    
                         & \cellcolor{blue!20} $q_7$ & \cellcolor{blue!20} [1] & \cellcolor{blue!20} [1] \\    
     \multirow{-6}{*}{[1]} & \cellcolor{red!20} $q_8$ & \cellcolor{red!20} [2] & \cellcolor{red!20} [1] \\
     \hline
     \multirow{2}{*}{[2]}
                         & $q_3$ & [2] & [2] \\    
                         & $q_5$ & [2] & [2] \\    
\end{tabular}\\
\end{table}

\textbf{Schritt 3} Nun Suchen wir nach Zuständen, die sich nicht gleich verhalten, also nicht in der selben Äquivalenzklasse liegen können (farbliche Markierungen in der vorherigen Tabelle) und unterteilen diese in neue Äquivalenzklassen.

\begin{table}[ht]
\centering
\begin{tabular}{r|r|r|r}
    Eq & Q & $\vdash 0$ & $\vdash 1$ \\
    \hline
    \cellcolor{yellow!20} [1] & $q_0$ & [2] & [5] \\
    \hline
    \cellcolor{blue!20}
    & $q_1$ & [2] & [2] \\    
    \cellcolor{blue!20}
    & $q_6$ & [2] & [4] \\    
    \cellcolor{blue!20}
    \multirow{-3}{*}{[2]} & $q_7$ & [2] & [4] \\    
    \hline
    \cellcolor{orange!20}
    [3] & $q_4$ & [5] & [5] \\    
    \hline
    \cellcolor{red!20}
    [4] & $q_8$ & [5] & [1] \\
    \hline
    \multirow{2}{*}{[5]}
                         & $q_3$ & [5] & [5] \\    
                         & $q_5$ & [5] & [5] \\    
\end{tabular}
\end{table}

\textbf{Schritt 4} Dies wiederholen wir solange, bis wir in einer Äquivalenzklasse nur noch Zustände habe, die sich gleich verhalten (die also äquivalent sind).

\begin{table}[ht]
\centering
\begin{tabular}{r|r|r|r}
    Eq & Q & $\vdash 0$ & $\vdash 1$ \\
    \hline
    [1] & $q_0$ & [2] & [6] \\
    \hline
    [2] & $q_1$ & [3] & [3] \\
    \hline
    \multirow{2}{*}{[3]}
    & $q_6$ & [3] & [5] \\    
    & $q_7$ & [3] & [5] \\    
    \hline
    \rowcolor{blue!20} $[4]$ & $q_4$ & $[6]$ & $[6]$ \\    
    \hline
    [5] & $q_8$ & [6] & [1] \\
    \hline
    \multirow{2}{*}{[6]}
                         & $q_3$ & [6] & [6] \\    
                         & $q_5$ & [6] & [6] \\    
\end{tabular}
\end{table}

Wenn dabei weitere Zustände auftreten, die nicht erreicht werden können, sollten diese ebenfalls gestrichen werden. Hier ist das \colorbox{blue!20}{$q_4$}.\\

\textbf{Schritt 5} Die so erzeugte Transitionstabelle ergibt den minimalen DEA.

\begin{center}
\begin{tikzpicture}[->, >=stealth', shorten >=1pt,
     node distance = 15mm]
 \node [state, initial] (q0) {$q_0$};
 \node [state] (q1) [right= of q0 ] {$q_1$};
 \node [state] (q67) [above=10mm of q1] {$q_{6,7}$};
 \node [state] (q8) [above=10mm of q67] {$q_8$};
 
 \node [state, accepting ] (q35) [right= of q1] {$q_{3,5}$};
 
 \path (q0) edge[below] node {0} (q1)
            edge[bend right=45, below] node {1} (q35)
       (q1) edge[left] node {0, 1} (q67)
       (q35) edge[loop right, right] node {0, 1} ()
       (q67) edge[left] node {1} (q8)
             edge[loop right] node {0} ()
       (q8) edge[bend left=45, above right] node {0} (q35)
            edge[bend right=45, above left] node {1} (q0);
\end{tikzpicture}
\end{center}

\end{document}
