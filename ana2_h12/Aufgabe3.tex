\documentclass[main.tex]{subfiles}

\begin{document}

\section{Aufgabe 3}
\textit{Analysis 2-Klausur vom 24.09.2020, A04}\\
Lösen Sie die Differenzialgleichung
\begin{equation*}
	y'' = 4y' - 4y + 8x + 4
\end{equation*}

\subsection{Lösung 3}
Gegeben ist eine lineare DGL 2. Ordnung mit polynomialer Störfunktion.
\begin{equiveqs}[clcl]
	   & y'' 			&=& 4y' - 4y + 8x + 4 \\
\equiv & y'' - 4y' + 4y &=& 8x + 4
\end{equiveqs}

Die charakteristische Gleichung der homogenen DGL lautet $\lambda^2-4\lambda+4=0 \equiv \lambda_{1,2} = 2$, also liegt eine doppelte Nullstelle vor und die Lösung lautet:
\[
	y_h = c_0 \e^{2x} + c_1 x \e^{2x}
\]

Zur Bestimmung der partikulären Lösung wählen wir den Ansatz
\begin{align*}
	y_p   &= ax^2 + bx + c \\
	y'_p  &= 2ax + b\\
	y''_p &= 2a
\end{align*}
und setzen ein:
\begin{equiveqs}[clcl]
	   & y_p'' - 4y_p' + 4y_p &=& 8x + 4 \\
\equiv & 2a - 8ax - 4b + 4ax^2 + 4bx + 4c  &=& 8x + 4  \\
\equiv & \underbrace{4a}_{\exeq 0}\cdot x^2 + \underbrace{(4b - 8a)}_{\exeq 8} \cdot x \underbrace{+2a -4b + 4c}_{\exeq 4}  &=& 8x + 4  \\
\end{equiveqs}

Mit $b=2$ und $c=3$ erhalten wir als partikuläre Lösung
\[
	y_p = 2x + 3
\]
und somit als allgemeine Lösung der DGL
\[
	y(x) = c_0 \e^{2x} + c_1 x \e^{2x} + 2x + 3.
\]
	
\end{document}
