\documentclass[main.tex]{subfiles}

\begin{document}

\section{Aufgabe 1}
\textit{Analysis 2-Klausur vom 16.07.2021, A05}\\
Lösen Sie die Differentialgleichung
\begin{equation*}
	y' - y = y^2 \cdot \sin(x)
\end{equation*}
Hinweis: Bernoulli-DGL

\subsection{Lösung 1}
Es liegt eine Differentialgleichung der Form
$$
	y'(x) + f(x)\cdot y = g(x)\cdot y^\alpha
$$
mit $\alpha = 2, g(x) = \sin(x), f(x) = -1$ vor. 

Wir substituieren mit
\begin{align*}
	z &:= y^{1-\alpha} = \frac{y}{y^\alpha} \;\equiv\; y=z^{\frac{1}{1-\alpha}}\\[4mm]
\intertext{also}
	y  &= z^{-1} \\
	y' &= -z^{-2}\cdot z'
\end{align*}

und setzen in die DGL:
\begin{equation*}
\begin{array}{crl}
				& y' - y &= y^2 \cdot \sin(x) \\[2mm]
\overset{z(x) \neq 0}{\equiv} & -z^{-2}\cdot z' - z^{-1} &= z^{-2}\cdot \sin(x) \\[2mm]
\equiv	& -z' - z &= \sin(x)
\end{array}
\end{equation*}

Wir bestimmen die Lösung der linearen inhomogen DGL $-z' - z = \sin(x)$, indem wir zunächst die Lösung der homogenen DGL $-z_h' - z_h = 0$ bestimmen.
Diese lautet
$$
	z_h = c_0 \cdot \e^{-x}. 
$$
Die Rücksubstitution ergibt
\begin{align*}
	y_h &= \frac{1}{z_h} \\
	    &= c \cdot \ct{e}^{x}
\end{align*}

Für die partikuläre Lösung nutzen wir den trigonometrischen Ansatz 
\begin{align*}
    z_p(x)   &= c_0 \sin(\alpha x) + c_1 \cos(\alpha x) \\
    z_p'(x)  &= \alpha c_0 \cos(\alpha x) - \alpha c_1 \sin(\alpha x)
\end{align*}
und setzen ein:
\begin{align*}
\begin{array}{crl}
				& \sin(x) &= -z_p' - z_p \\
\equiv & \sin(x) &= - \alpha c_0 \cos(\alpha x) + \alpha c_1 \sin(\alpha x)
							 - c_0 \sin(\alpha x) - c_1 \cos(\alpha x) \\
\overset{\alpha = 1}{\equiv} & \sin(x) &=
					  \underbrace{(c_1 - c_0)}_{= 1}\cdot \sin(x)
					- \underbrace{(c_0 + c_1)}_{= 0}\cdot \cos(x) \\
\end{array}
\end{align*}

Der Koeffizientenvergleich liefert uns mit $c_0 = -\sfrac{1}{2}$ und $c_1 = \sfrac{1}{2}$ die Funktion
\[
	z_p(x) = -\frac{1}{2} \sin(x) + \frac{1}{2}\cos(x).
\]

Die Rücksubstitution ergibt:
\begin{align*}
	y_p &= \frac{1}{z_p} \\
	    &= \frac{2}{\cos(x) - \sin(x)}
\end{align*}

Somit können wir die allgemeine Lösung der inhomogenen DGL wie folgt angeben:
\begin{align*}
	y(x) &= c \cdot \ct{e}^{x} + \frac{2}{\cos(x) - \sin(x)}
\end{align*}



\end{document}