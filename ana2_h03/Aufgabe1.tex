\documentclass[main.tex]{subfiles}

\begin{document}

\section{Aufgabe 1}
Gegeben sei das Vektorfeld $\vec{f}$ mit:

$$
    \vec{f} = \begin{pmatrix}[1]
        x^2 + 5a \cdot y + 3y\cdot z \\
        5x + 3a\cdot x\cdot z - 2 \\
        2x\cdot y + a\cdot x \cdot y - 4z
    \end{pmatrix}
$$
\begin{enumerate}
    \item Berechnen Sie die Rotation.
    \item Für welche Werte von $a$ ist das Feld wirbelfrei?
\end{enumerate}

\subsection{Lösung 1a}

% Das Kreuzprodukt vom Nabla-Operator mit einem Vektorfeld nennt man Rotation.
% Die Rotation eines Vektorfeldes ist ein Maß für Drehbewegungen bzw. für die Wirbel des Vektorfeldes.
\arraycolsep=1pt\def\arraystretch{1} % streach array

$$
    \operatorname{rot} \vec{f} = \vec{\nabla} \times \vec{f} = \begin{pmatrix}
        \frac{\partial f_z}{\partial y} - \frac{\partial f_y}{\partial z} \\
        \frac{\partial f_x}{\partial z} - \frac{\partial f_z}{\partial x} \\
        \frac{\partial f_y}{\partial x} - \frac{\partial f_x}{\partial y}
    \end{pmatrix} = \begin{pmatrix}
        2x + a\cdot x - 3a\cdot x\\
        3y - (2y + a\cdot y) \\
        5 + 3a\cdot z - (5a + 3z)
    \end{pmatrix} = \begin{pmatrix}
        2x \cdot (1 - a)\\
        y \cdot (1 - a) \\
        5\cdot (1-a) + 3z\cdot (a-1)
    \end{pmatrix}
$$

\subsection{Lösung 1b}

Das Feld ist wirbelfrei, wenn $\operatorname{rot} \vec{f} = \vec{0}$ ist, also für $a = 1$.












\end{document}
