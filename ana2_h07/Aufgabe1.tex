\documentclass[main.tex]{subfiles}

\begin{document}

\section{Aufgabe 1}
Berechnen Sie mittels Polarkoordinaten das uneigentliche mehrdimensionale Integral
\[
\int \int \limits_{\mathbb{R}^2} \ct{e}^{-(x^2 + y^2)} \ dx \ dy
= \int \limits_{y = -\infty}^{\infty} \int \limits_{x=-\infty}^{\infty} 
		\ct{e}^{-(x^2 + y^2)} \ dx \ dy
\]

\subsection{Lösung 1}

Uneigentliche Integrale können mithilfe von Polarkoordinaten wie folgt gelöst werden:
\begin{align*}
    \iint f(x,y) \dx{y,x} = \int_{\phi=0}^{2\pi} \int_{r=0}^{\inf} f(r\cdot \cos \phi, r\cdot \sin \phi) \cdot r \dx{(r,\phi)} \\
\end{align*} 

Es gilt also
\begin{align*}
    \iint \ct{e}^{-(x^2 + y^2)} \dx{(x,y)}
    &= \int_{\phi=0}^{2\pi} \int_{r=0}^{\infty} \ct{e}^{-\left((r\cdot \cos \phi)^2 + (r\cdot \sin \phi)^2\right)} \cdot r \dx{(r,\phi)} \\
    &= \int_{\phi=0}^{2\pi} \int_{r=0}^{\infty} \ct{e}^{-r^2} \cdot r \dx{(r,\phi)} \\
    &= \int_{\phi=0}^{2\pi} \int_{r=0}^{\infty} r\cdot \ct{e}^{u} \frac{1}{-2r} \dx{(u,\phi)} \\
    &= \int_{\phi=0}^{2\pi} \frac{-1}{2} \cdot \int_{r=0}^{\infty} \ct{e}^{u} \dx{(u,\phi)} \\
    &= \int_{\phi=0}^{2\pi} \frac{-1}{2} \cdot \left[ \ct{e}^{-r^2} \right]_{r=0}^{\infty} \dx{\phi} \\
    &= \int_{\phi=0}^{2\pi} \frac{1}{2} \dx{\phi} \\
    &= \left[ \frac{1}{2}\phi \right]_{\phi=0}^{2\pi} \\
    &= \pi \\
\end{align*}

\end{document}
