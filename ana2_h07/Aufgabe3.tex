\documentclass[main.tex]{subfiles}

\begin{document}

\section{Aufgabe 3}
Gegeben sei die Halbkugel
\[
    H = \{ (x,y,z) \in \mathbb{R}^3 \ | \ x^2 + y^2 + z^2 \leq 9, \ z \geq 0 \}
\]
mit der Dichte $\rho(x, y, z) = 1$.
\begin{enumerate}
    \item Berechnen Sie die Masse der Halbkugel.
    \item Berechnen Sie anschließend den Schwerpunkt der Halbkugel.
\end{enumerate}

\subsection{Lösung 3}
Für das Rechnen mit Kugelkoordinaten, mit $\phi$ \textbf{auf} der $x$-$y$-Ebene und 
dem Winkel $\theta$ von der $z$-Achse auf die $x$-$y$-Ebene gilt:
\begin{align*}
    \iiint f(x,y,z) \dx{(z,y,x)} &= 
        \iiint f(r\cos\phi\sin\theta, r\sin\phi\sin\theta, r\cos\theta )\cdot |r^2\sin\theta| \dx{(\theta, \phi, r)}
\end{align*}

Um die Halbkugel als Integral über das Gebiet $H$ der Funktion $\rho(x, y, z)$ zu beschreiben, überlegen wir uns die Orientierung
der Halbkugel im Koordinatensystem und die Darstellung in Kugelkoordinaten.

Um den Betrag im Integral zu behandeln, teilen wir das Integral in die Summer zweier Integrale auf.

\begin{align*}
    \iiint \limits_H \rho(x, y, z) \dx{H} 
    &=
    \int_{r=0}^{3}
    \int_{\phi=0}^{\frac{\pi}{2}}
    \int_{\theta=0}^{2\pi}
        1 \cdot \abs{r^2\sin\theta}
    \dx{(\theta, \phi, r)}\\
    &=
    \int_{r=0}^{3}
    \int_{\phi=0}^{\frac{\pi}{2}}
    \left(
        \int_{\theta=0}^{\pi}
            r^2\sin\theta
        \dx{\theta}+ (-1) \cdot
        \int_{\theta=\pi}^{2\pi}
            r^2\sin\theta
        \dx{\theta}
    \right)
    \dx{(\phi, r)}\\
    &=
    \int_{r=0}^{3}
    \int_{\phi=0}^{\frac{\pi}{2}}
    \left(
        \left[
            - r^2 \cos\theta
        \right]_{\theta=0}^{\pi} 
        + (-1) \cdot
        \left[
            - r^2 \cos\theta
        \right]_{\theta=\pi}^{2\pi}
    \right)
    \dx{(\phi, r)}\\
    &=
    \int_{r=0}^{3}
    \int_{\phi=0}^{\frac{\pi}{2}}
    \left(
        \left(
            - r^2 \cos (\pi)
            + r^2 \cos (0)
        \right) 
        -
        \left(
            - r^2 \cos (2\pi)
            + r^2 \cos (\pi)
        \right)
    \right)
    \dx{(\phi, r)}\\
    &=
    \int_{r=0}^{3}
    \int_{\phi=0}^{\frac{\pi}{2}}
        4 r^2
    \dx{(\phi, r)}\\
    &=
    \int_{r=0}^{3}
    \left[
        4 r^2 \cdot \phi
    \right]_{\phi=0}^{\frac{\pi}{2}}
    \dx{r}\\
    &=
    \int_{r=0}^{3}
        2 r^2 \cdot \pi
    \dx{r}\\
    &=
    \left[
        \frac{2}{3} \pi r^3
    \right]_{r=0}^{3}\\
    &= \frac{2}{3} \pi\cdot 3^3 \\
    &= 18\pi \\
\end{align*}

Zur Probe rechnen wir das Volumen einer Kugel mit dem Radius $r=3$ aus
$V = \frac{4}{3} \pi r^3 = 36 \pi$, halbieren dies und erhalten ebenfalls $18\pi$. 

\end{document}
