\documentclass[main.tex]{subfiles}

\begin{document}

\section{Aufgabe 8}
Bei welchen Werten $a,\; b$ hat die Matrix
$$
    A = \eqmatrix{rr}{a & -1 \\ 1 & b}
$$
\begin{enumerate}
    \item zwei verschiedene reelle Eigenwerte?
    \item einen (doppelten) reellen Eigenwert?
    \item keine reellen Eigenwert?
\end{enumerate}

\subsection{Lösung 8}
Wir bestimmen die allgemeine charakteristische Gleichung und formen in einer Weise um, die uns den Koeffizientenvergleich mit der $p$-$q$-Formel erlaubt.

\begin{align*}
    0 &= \det (A-\lambda E) \\
    &= \begin{vmatrix}
        a -\lambda & -1 \\ 1 & b-\lambda
    \end{vmatrix} \\[1mm]
    &= (a-\lambda)(b-\lambda)+1 \\[1mm]
    &= ab -a\lambda -b\lambda + \lambda^2 + 1 \\[1mm]
    &= \lambda^2 \underbrace{-(a+b)}_{:= p}\cdot \lambda + \underbrace{ab +1}_{:= q}
\end{align*}

Mit $\lambda = \frac{p}{2} \pm \sqrt{\left(\frac{p}{2}\right)^2 -q}$ können wir nun durch Betrachtung der Diskriminante erkennen, dass
\begin{enumerate}
    \item zwei verschiedene reelle Eigenwerte genau dann vorliegen, wenn $\left(\frac{p}{2}\right)^2 > q$, also $(a-b)^2 > 4$
    \item ein doppelter reeller Eigenwert genau dann vorliegt, wenn $\left(\frac{p}{2}\right)^2 = q$, also $(a-b)^2 = 4$
    \item keine reellen Eigenwerte vorliegen, genau dann wenn $\left(\frac{p}{2}\right)^2 < q$, also $(a-b)^2 < 4$.
\end{enumerate}

\end{document}
