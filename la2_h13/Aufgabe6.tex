\documentclass[main.tex]{subfiles}

\begin{document}

\section{Aufgabe 6}
Bestimmen Sie die Eigenwerte und Eigenvektoren der Matrizen
\begin{align*}
    \text{a) }\; A &=\eqmatrix{rr}{2 & 4 \\ 1 & 2}
    & \text{b) }\; B &=\eqmatrix{rr}{3 & -1 \\ 2 & 5}
\end{align*}
Bestimmen  Sie auch eventuelle komplexe Eigenwerte und Eigenvektoren. Ein komplexes Gleichungssystem können Sie wie ein reelles Gleichungssystem mit dem Gauß-Verfahren lösen.

\subsection{Lösung 6a}
Wir ermitteln die Eigenwerte durch lösen der charakteristischen Gleichung.
\begin{equiveqs}[crcl]
       & 0 &=& \det(A-\lambda E) \\[2mm]
\equiv & 0 &=& \begin{vmatrix}
        2-\lambda & 4 \\
        1 & 2 - \lambda
    \end{vmatrix} \\
\equiv & 0 &=& (2-\lambda)^2 - 4 \\
\equiv & 0 &=& \lambda^2 - 4\lambda \\
\end{equiveqs}

Somit erhalten wir die Eigenwerte $\lambda_1 = 4$ und $\lambda_2 = 0$.
Um die zugehörigen Eigenvektoren zu erhalten, setzen wir lösen wir die Gleichung $(A-\lambda E)\cdot v = 0$ nach $v$.
\begin{equiveqs}[crcl]
       & 0 &=& (A-\lambda_1 E)\cdot v\\[2mm]
\equiv & 0 &=& \eqmatrix{rr}{
    2-4 & 4 \\
    1 & 2-4 } \cdot v \\
\equiv & 0 &=& \eqmatrix{rr}{
    -2 & 4 \\
    1 & -2 } \cdot v \\
\end{equiveqs}
Wir addieren zwei mal die zweite Zeile zur ersten Zeilen und bestimmen $x_1 = t$ als freie Variable. Damit erhalten wir $x_2 = \sfrac{t}{2}$ und den zu $\lambda_1 = 4$ gehörigen Eigenvektor $v_1 = \vektor{1 \\ 0,5}$.

Für $\lambda_2 = 0$ verfahren wir ebenso und erhalten $v_2 = \vektor{-2 \\ 1}$.

\subsection{Lösung 6b}
Wir bestimmen $\abs{B - \lambda E} = 0$ also $(3-\lambda)(5-\lambda)+2 = 0 \equiv \lambda^2 -8\lambda + 17 = 0$ und erhalten die komplexen Eigenwerte $\lambda_1 = 4+\i$ und $\lambda_2 = 4-\i$.

Für $\lambda_1$ bestimmen wir $v_1$.
\begin{equiveqs}[crcl]
    & 0 &=& (B-\lambda_1 E)\cdot v\\[2mm]
\equiv & 0 &=& \eqmatrix{rr}{
 -1 -\i & -1 \\
 2 & 1 -\i } \cdot v \\
\end{equiveqs}
\[
    \eqmatrix{rr}{
     -1 -\i & -1 \\
    2 & 1 -\i }
    \leadsto \eqmatrix{rr}{
    1 & (1-\i)/2 \\
    2 & 1 -\i }
    \leadsto \eqmatrix{rr}{
    1 & (1-\i)/2 \\
    0 & 0 }
\]
Mit $x_2 = t$ erhalten wir $x_1 = t \cdot (-\frac{1}{2} + \frac{\i}{2})$ und den zu $\lambda_1 = 4+\i$ gehörigen Eigenvektor $v_1 = \vektor{-1 + \i \\ 2}$.

Für $\lambda_2 = 4-\i$ kommt man entsprechend auf den Eigenvektor $v_2 = \vektor{-1 - \i \\ 2}$.

\end{document}
