\documentclass[main.tex]{subfiles}

\begin{document}

\section{Aufgabe 7}
Eine Abbildung im $\R^2$ ist wie folgt definiert:
Ein gegebener Vektor $x$ wird zuerst um $90^\circ$ im Uhrzeigersinn gedreht und anschließend an der $y$-Achse gespiegelt.
\begin{enumerate}
    \item Das Vielfache welchen Vektors wird auf sich selbst abgebildet?
    \item Die Abbildung welchen Vektors (bzw. des Vielfachen diesen Vektors) zeigt genau in die entgegengesetzte Richtung?
    \item Stellen Sie die zugehörige Abbildungsmatrix $A$ auf.
    \item Bestimmen Sie die Eigenwerte von $A$. Passen diese zu den Beobachtungen aus a) und b)?
\end{enumerate}

\subsection{Lösung 7}

\end{document}
