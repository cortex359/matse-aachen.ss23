\documentclass[main.tex]{subfiles}

\begin{document}

\section{Aufgabe 7}
Eine Abbildung im $\R^2$ ist wie folgt definiert:
Ein gegebener Vektor $x$ wird zuerst um $90^\circ$ im Uhrzeigersinn gedreht und anschließend an der $y$-Achse gespiegelt.
\begin{enumerate}
    \item Das Vielfache welchen Vektors wird auf sich selbst abgebildet?
    \item Die Abbildung welchen Vektors (bzw. des Vielfachen diesen Vektors) zeigt genau in die entgegengesetzte Richtung?
    \item Stellen Sie die zugehörige Abbildungsmatrix $A$ auf.
    \item Bestimmen Sie die Eigenwerte von $A$. Passen diese zu den Beobachtungen aus a) und b)?
\end{enumerate}

\subsection{Lösung 7a}
Dreht man bspw. den Vektor $(1, -1)^T$ um $90^\circ$ im Uhrzeigersinn $(-1, -1)^T$ und spiegelt ihn an der $y$-Achse, so erhalt man wieder den Ursprungsvektor $(1, -1)^T$. Dies gilt auch alle Vielfachen des Vektors.

\subsection{Lösung 7b}
In die entgegengesetzte Richtung nach Anwendung der Transformationen zeigen die Vektoren $(1, 1)^T$, $(-1, -1)^T$, sowie weitere Vielfache.

\subsection{Lösung 7c}
Die Abbildungsmatrix der Transformationen ergibt sich als Produkt der Rotationsmatirx
\[
    R_\alpha = \eqmatrix{rr}{\cos\alpha & -\sin\alpha \\ \sin\alpha & \cos\alpha}
\]
sowie der Spiegelmatirx
\[
    S_y = \eqmatrix{rr}{1 & 0\\ 0 & -1}
\]
zu
\begin{align*}
    A &= S_y \cdot R_{\frac{\pi}{2}}\\
    &= \eqmatrix{rr}{1 & 0\\ 0 & -1} \cdot \eqmatrix{rr}{0 & -1\\ 1 & 0} \\
    &= \eqmatrix{rr}{0 & -1 \\ -1 & 0}
\end{align*}

\subsection{Lösung 7d}
Die charakteristische Gleichung der Abbildungsmatrix $A$ lautet $\lambda^2 -1 = 0$ und die Eigenwerte sind somit $\lambda_1 = 1$ und $\lambda_2 = -1$.

Zu $\lambda_1$ gehört der Eigenvektor $v_1 = \vektor{-1 \\ 1}$ und zu $\lambda_2$ der Eigenvektor $v_2 = \vektor{1 \\ 1}$.

Der Vektor $v_1$ passt zu unserem Ergebnis aus 7a) mit $(-1)\cdot v_1 = (1, -1)^T$. Der Vektor $v_2$ passt zu dem Ergebnis aus 7b).


\end{document}
