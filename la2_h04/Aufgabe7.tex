\documentclass[main.tex]{subfiles}

\begin{document}

\section{Aufgabe 7}
$A$ sei eine $3 \times 3$-Matrix.
\begin{enumerate}
    \item Welche Beziehung ($=,\neq, \subseteq, \subset, \supset, \supseteq$) besteht zwischen dem Kern von $A$ und dem Kern von $A^2$ (und dem von $A^3$)?
    \item Verifizieren Sie ihr Ergebnis aus (a) mit der Matrix $\begin{pmatrix}
        0 & 1 & 0 \\
        0 & 0 & 1 \\
        0 & 0 & 0
    \end{pmatrix}$.
\end{enumerate}

\subsection{Lösung 7a}
Der Kern einer Matrix $A$ ist die Menge aller Vektoren $x \in \mathbb{R}^3$, die von $A$ auf den Nullvektor abgebildet werden, d.h.
\begin{align*}
    \ker \left(A\right) = \left\{ x\in \mathbb{R}^3 \middle| Ax=0 \right\}.
\end{align*}

Für $A^2$ gilt entsprechend
\begin{align*}
    \ker \left(A^2\right) = \left\{ x\in \mathbb{R}^3 \middle| A^2x=0 \right\}.
\end{align*}

Da $Ax$ ein Vektor in $\mathbb{R}^3$ ist, bedeutet $Ax \in \operatorname{ker}(A)$, dass $A^2x = A(Ax) = 0$.
Daher gilt
\begin{align*}
    \ker \left(A^2\right) &= \left\{ x\in \mathbb{R}^3 \middle| A(Ax)=0 \right\} \\
                          &= \left\{ x\in \mathbb{R}^3 \middle| Ax \in \ker \left(A\right) \right\} \\
                          &\supseteq \ker \left(A\right)
\end{align*}

Das bedeutet, dass $\ker \left(A^n\right) \subseteq \ker \left(A^{n+1}\right)\ \forall n \in \mathbb{N}_0$.

\subsection{Lösung 7b}

Sei
$$
    A = \begin{pmatrix}
        0 & 1 & 0 \\
        0 & 0 & 1 \\
        0 & 0 & 0
        \end{pmatrix}.
$$
Dann ist
$$
    A^2 = \begin{pmatrix}
    0 & 0 & 1 \\
    0 & 0 & 0 \\
    0 & 0 & 0
    \end{pmatrix}.
$$

Wir wollen nun den Kern von $A$ und $A^2$ bestimmen. Der Kern von $A$ ist der Lösungsraum des homogenen Gleichungssystems $Ax = 0$:
\begin{align*}
    \begin{pmatrix}
        0 & 1 & 0 \\
        0 & 0 & 1 \\
        0 & 0 & 0
    \end{pmatrix}
    \begin{pmatrix} x_1 \\ x_2 \\ x_3 \end{pmatrix}
    = \begin{pmatrix} 0 \\ 0 \\ 0 \end{pmatrix}
    \Rightarrow x_2 = 0,\ x_3 = 0
\end{align*}

Das bedeutet, dass der Kern von $A$ durch den Vektor $(1, 0, 0)^T$ aufgespannt wird.

Der Kern von $A^2$ ist der Lösungsraum des homogenen Gleichungssystems $A^2x = 0$:
\begin{align*}
    \begin{pmatrix}
        0 & 0 & 1 \\
        0 & 0 & 0 \\
        0 & 0 & 0
    \end{pmatrix}
    \begin{pmatrix} x_1 \\ x_2 \\ x_3 \end{pmatrix}
    = \begin{pmatrix} 0 \\ 0 \\ 0 \end{pmatrix}
    \Rightarrow x_3 = 0
\end{align*}

Das bedeutet, dass der Kern von $A^2$ durch die Vektoren $(1, 0, 0)^T$ und $(0, 1, 0)^T$ aufgespannt wird.

Wie erwartet ist der Kern von $A$ ein Unterraum des Kerns von $A^2$.

\end{document}
