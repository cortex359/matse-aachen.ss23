\documentclass[main.tex]{subfiles}

\begin{document}

\section{Aufgabe 6}
Gegeben sei die folgende Matrix:

\begin{equation*}
    A = \begin{pmatrix}
        2 & 0 & 0 \\
        0 & 1 & 0 \\
        1 & 0 & 1
    \end{pmatrix}
\end{equation*}

\begin{enumerate}
    \item Berechnen Sie $A^k$ für $k=1 \ldots 3$.
    \item Stellen Sie eine Vermutung auf für $A^n$ und beweisen Sie diese.
\end{enumerate}

\subsection{Lösung 6a}

\begin{equation*}
    \begin{aligned}
        A^1 & = \begin{pmatrix}
            2 & 0 & 0 \\
            0 & 1 & 0 \\
            1 & 0 & 1
        \end{pmatrix}\\
        A^2 & = \begin{pmatrix}
            2 & 0 & 0 \\
            0 & 1 & 0 \\
            1 & 0 & 1
        \end{pmatrix} \times \begin{pmatrix}
            2 & 0 & 0 \\
            0 & 1 & 0 \\
            1 & 0 & 1
        \end{pmatrix} = \begin{pmatrix}
            4 & 0 & 0 \\
            0 & 1 & 0 \\
            3 & 0 & 1
        \end{pmatrix}\\
        A^3 & = \begin{pmatrix}
            4 & 0 & 0 \\
            0 & 1 & 0 \\
            3 & 0 & 1
        \end{pmatrix} \times \begin{pmatrix}
            2 & 0 & 0 \\
            0 & 1 & 0 \\
            1 & 0 & 1
        \end{pmatrix} = \begin{pmatrix}
            8 & 0 & 0 \\
            0 & 1 & 0 \\
            7 & 0 & 1
        \end{pmatrix}\\
    \end{aligned}
\end{equation*}

\subsection{Lösung 6b}
Zu zeigen:
\begin{equation*}
    \begin{aligned}
        A^n & = \begin{pmatrix}
            2^n & 0 & 0 \\
            0 & 1 & 0 \\
            2^n -1 & 0 & 1
        \end{pmatrix}\\
    \end{aligned}
\end{equation*}

Induktionsanfang mit $n = 1$:
\begin{equation*}
    \begin{aligned}
        A^1 & = \begin{pmatrix}
            2^1 & 0 & 0 \\
            0 & 1 & 0 \\
            2^1 -1 & 0 & 1
        \end{pmatrix}\\
    \end{aligned} = \begin{pmatrix}
        2 & 0 & 0 \\
        0 & 1 & 0 \\
        1 & 0 & 1
    \end{pmatrix}\ \checkmark
\end{equation*}

Induktionsvoraussetzung: Es sei 
\begin{equation*}
    \begin{aligned}
        A^n & = \begin{pmatrix}
            2^n & 0 & 0 \\
            0 & 1 & 0 \\
            2^n -1 & 0 & 1
        \end{pmatrix}
    \end{aligned}
\end{equation*}
für ein festes, aber beliebiges $n \in \mathbb{N}$.\\

Induktionsbehauptung: Dann gilt für $n+1$ ebenso
\begin{equation*}
    \begin{aligned}
        A^{n+1} & = \begin{pmatrix}
            2^{n+1} & 0 & 0 \\
            0 & 1 & 0 \\
            2^{n+1} -1 & 0 & 1
        \end{pmatrix}.
    \end{aligned}
\end{equation*}

Induktionsschluss: 
\begin{equation*}
    \begin{aligned}
        A^{n+1} = A^n \times A & = \begin{pmatrix}
            2^n & 0 & 0 \\
            0 & 1 & 0 \\
            2^n -1 & 0 & 1
        \end{pmatrix} \times \begin{pmatrix}
            2 & 0 & 0 \\
            0 & 1 & 0 \\
            1 & 0 & 1
        \end{pmatrix} \\
        & = \begin{pmatrix}
            2^n\cdot 2 & 0 & 0 \\
            0 & 1 & 0 \\
            (2^n -1)\cdot 2 + 1 & 0 & 1
        \end{pmatrix}\\
        & = \begin{pmatrix}
            2^{n+1} & 0 & 0 \\
            0 & 1 & 0 \\
            2^{n+1} -2 + 1 & 0 & 1
        \end{pmatrix}\\
        & = \begin{pmatrix}
            2^{n+1} & 0 & 0 \\
            0 & 1 & 0 \\
            2^{n+1} -1 & 0 & 1
        \end{pmatrix}\ \checkmark
    \end{aligned}
\end{equation*}

\end{document}
