\documentclass[main.tex]{subfiles}

\begin{document}

\section{Aufgabe 8}
Die Spur einer quadratischen Matrix $A=(a_{ij})$ ist definiert durch
\begin{equation} \label{eq:spur}
    \Spur (A) = \sum_{i=1}^{n} a_{ii}.
\end{equation}

\begin{enumerate}
    \item Zeigen Sie, dass die Spur eine lineare Abbildung darstellt.
    \item Sei $A = \begin{pmatrix}
        1 & 2 & 3 \\
        0 & 1 & 0
    \end{pmatrix}$ und $B = A^T$. Verifizieren Sie $\Spur (AB)= \Spur (BA)$.
    \item Zeigen Sie, dass $\Spur (AB) = \Spur (BA)$, wobei $A \in \mathbb{R}^{m\times n}$ und $B\in \mathbb{R}^{n\times m}$.
    \item Zeigen Sie, dass $\Spur (A^T A) = 0$ genau dann, wenn $A=(0)$.
    \item Man zeige weiter: $\Spur (ABC) = \Spur (BCA)$, aber i.a. $\Spur (ABC) \neq \Spur (BAC)$.
\end{enumerate}

\subsection{Lösung 8a}
Die Abbildung
$$
    f : \begin{cases}
        \mathbb{R}^{n\times n} \to \mathbb{R}\\
        A \mapsto \Spur (A)
    \end{cases}
$$
ist genau dann lineare, wenn sie homogen und additiv ist.\\

Zeige Homogenität mit $\lambda \in \mathbb{R}$:
\begin{equation*}
    \begin{aligned}
                        \Spur (\lambda \cdot A)       & = \lambda \cdot \Spur (A)\\
        \Leftrightarrow \sum_{i=1}^{n} \lambda a_{ii} & = \lambda \cdot \sum_{i=1}^{n} a_{ii}\ \checkmark
    \end{aligned}
\end{equation*}

Zeige Additivität mit $B = (b_{ij}) \in \mathbb{R}^{n\times n}$:
\begin{equation*}
    \begin{aligned}
        \Spur (A + B) & = \Spur (B) + \Spur (A) \\
        \Leftrightarrow \sum_{i=1}^{n} ( a_{ii} + b_{ii} ) & = \sum_{i=1}^{n} a_{ii} + \sum_{i=1}^{n} b_{ii} \ \checkmark
    \end{aligned}
\end{equation*}

\subsection{Lösung 8b}
\begin{align*}
    A\times B = \begin{pmatrix}
        1 & 2 & 3 \\
        0 & 1 & 0 \\
    \end{pmatrix} \times \begin{pmatrix}
        1 & 0 \\
        2 & 1 \\
        3 & 0 \\
    \end{pmatrix} &= \begin{pmatrix}
        14 & 2 \\
        2 & 1 \\
    \end{pmatrix}\\
    B\times A = \begin{pmatrix}
        1 & 0 \\
        2 & 1 \\
        3 & 0 \\
    \end{pmatrix} \times \begin{pmatrix}
        1 & 2 & 3 \\
        0 & 1 & 0 \\
    \end{pmatrix} &= \begin{pmatrix}
            1 & 2 & 3 \\
            2 & 5 & 6 \\
            3 & 6 & 9 \\
    \end{pmatrix}
\end{align*}

Es ergibt sich somit für $\Spur (A\times B) = 14 + 1 = 15$ und für $\Spur (B\times A) = 1 + 5 + 9 = 15$.

\subsection*{Lösung 8c}
Das Produkt der Matrizen $A = (a_{ij}) \in \mathbb{R}^{m\times n}$ und $B = (b_{ij}) \in \mathbb{R}^{n\times m}$
ist von der Form $AB = C \in \mathbb{R}^{m\times m}$ bzw. $BA = D \in \mathbb{R}^{n \times n}$.\\

Nach Definition 4.56 gilt für die Koordinaten $(c_{ik})$ des Produkts der Matrizen $A$ und $B$,
sowie entsprechend für die Koordinaten $(d_{ik})$
\begin{align*}
    c_{ik} &= \sum_{j=1}^{n} a_{ij} b_{jk}\\
    d_{ik} &= \sum_{j=1}^{m} b_{ij} a_{jk}.
\end{align*}

Aus der Definition der Spur \ref{eq:spur} ergibt sich
\begin{align*}
    \Spur (C) &= \sum_{i=1}^{m} c_{ii} = \sum_{i=1}^{m} \sum_{j=1}^{n} a_{ij} b_{ji}\\
    \Spur (D) &= \sum_{i=1}^{n} d_{ii} = \sum_{i=1}^{n} \sum_{j=1}^{m} b_{ij} a_{ji}.
\end{align*}

Durch umordnung der Summenzeichen zeigt sich, dass die beiden Darstellung isomorph (gleich bis auf Umbenennung der Indizes $i$ und $j$) zueinander sind.
\begin{align*}
    \sum_{i=1}^{n} \sum_{j=1}^{m} b_{ij} a_{ji} = \sum_{j=1}^{m} \sum_{i=1}^{n} b_{ij} a_{ji} = \sum_{j=1}^{m} \sum_{i=1}^{n} a_{ji} b_{ij} \simeq \sum_{i=1}^{m} \sum_{j=1}^{n} a_{ij} b_{ji}
\end{align*}

Somit ist gezeigt, dass $\Spur (AB) = \Spur (BA). \ \checkmark$

\subsection*{Lösung 8d}
Für eine Matrix $A = (a_{ij}) \in \mathbb{R}^{m\times n}$ ist bekannt, dass $A^T = (a_{ji}) \in \mathbb{R}^{n\times m}$.

Für das Produkt $B \in \mathbb{R}^{n\times n}$ mit $B = (b_{ik}) = A^T \times A$ gilt
\begin{align*}
    b_{ik} &= \sum_{j=1}^{n} a_{ij} a_{kj}.
\end{align*}

Entsprechend der Definition der Spur \ref{eq:spur} ergibt sich
\begin{align*}
    \Spur (B) = \sum_{i=1}^{n} b_{ii}
    = \sum_{i=1}^{m} \sum_{j=1}^{n} a_{ij} a_{ij}
    = \sum_{i=1}^{m} \sum_{j=1}^{n} a_{ij}^2.
\end{align*}

Es gilt $\Spur (A^T \times A) = 0$ genau dann wenn $a_{ij} = 0\ \forall i\in [1;m], j\in [1;n].\ \checkmark$

\subsection*{Lösung 8e}

Durch Anwendung des Assoziativgesetzes der Matrixmultiplikation (Satz 4.59 Abs. 1) und der Kommutativität aus 8c lässt sich
für $A\in \mathbb{R}^{m\times n}$ und $BC\in \mathbb{R}^{n\times m}$ zeigen
\begin{align*}
    \Spur (ABC) = \Spur \left(A(BC)\right) = \Spur \left((BC)A\right) = \Spur (BCA).
\end{align*}


% {{a_11,a_12,a_13},{a_21,a_22,a_23},{a_31,a_32,a_33}}

% {x_1,x_2,x_3}

% {{b_11,b_12,b_13},{b_21,b_22,b_23},{b_31,b_32,b_33}}
% {{c_11,c_12,c_13},{c_21,c_22,c_23},{c_31,c_32,c_33}}

\end{document}
