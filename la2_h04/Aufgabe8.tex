\documentclass[main.tex]{subfiles}

\begin{document}

\section{Aufgabe 8}
Die Spur einer quadratischen Matrix $A=(a_{ij})$ ist definiert durch
\[
    \Spur (A) = \sum_{i=1}^{n} a_{ii}.
\]

\begin{enumerate}
    \item Zeigen Sie, dass die Spur eine lineare Abbildung darstellt.
    \item Sei $A = \begin{pmatrix}
        1 & 2 & 3 \\
        0 & 1 & 0
    \end{pmatrix}$ und $B = A^T$. Verifizieren Sie $\Spur (AB)= \Spur (BA)$.
    \item Zeigen Sie, dass $\Spur (AB) = \Spur (BA)$, wobei $A \in \mathbb{R}^{m\times n}$ und $B\in \mathbb{R}^{n\times m}$.
    \item Zeigen Sie, dass $\Spur (A^T A) = 0$ genau dann, wenn $A=(0)$.
    \item Man zeige weiter: $\Spur (ABC) = \Spur (BCA)$, aber i.a. $\Spur (ABC) \neq \Spur (BAC)$.
\end{enumerate}

\subsection{Lösung 8a}
Die Abbildung 
$$
    f : \begin{cases}
        \mathbb{R}^{n\times n} \to \mathbb{R}\\
        A \mapsto \Spur (A)
    \end{cases}
$$
ist genau dann lineare, wenn sie homogen und additiv ist.\\

Zeige Homogenität mit $\lambda \in \mathbb{R}$:
\begin{equation}
    \begin{aligned}
                        \Spur (\lambda \cdot A)       & = \lambda \cdot \Spur (A)\\
        \Leftrightarrow \sum_{i=1}^{n} \lambda a_{ii} & = \lambda \cdot \sum_{i=1}^{n} a_{ii}\ \checkmark
    \end{aligned}
\end{equation}

Zeige Additivität mit $B = (b_{ij}) \in \mathbb{R}^{n\times n}$:
\begin{equation}
    \begin{aligned}
        \Spur (A + B) & = \Spur (B) + \Spur (A) \\
        \Leftrightarrow \sum_{i=1}^{n} ( a_{ii} + b_{ii} ) & = \sum_{i=1}^{n} a_{ii} + \sum_{i=1}^{n} b_{ii} \ \checkmark
    \end{aligned}
\end{equation}

\subsection{Lösung 8b}
\begin{align}
    A\times B = \begin{pmatrix}
        1 & 2 & 3 \\
        0 & 1 & 0 \\
    \end{pmatrix} \times \begin{pmatrix}
        1 & 0 \\
        2 & 1 \\
        3 & 0 \\
    \end{pmatrix} &= \begin{pmatrix}
        14 & 2 \\
        2 & 1 \\
    \end{pmatrix}\\
    B\times A = \begin{pmatrix}
        1 & 0 \\
        2 & 1 \\
        3 & 0 \\
    \end{pmatrix} \times \begin{pmatrix}
        1 & 2 & 3 \\
        0 & 1 & 0 \\
    \end{pmatrix} &= \begin{pmatrix}
            1 & 2 & 3 \\
            2 & 5 & 6 \\
            3 & 6 & 9 \\
    \end{pmatrix}
\end{align}

Es ergibt sich somit für $\Spur (A\times B) = 14 + 1 = 15$ und für $\Spur (B\times A) = 1 + 5 + 9 = 15$.

\subsection*{Lösung 8c}

\end{document}
