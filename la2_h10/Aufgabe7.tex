\documentclass[main.tex]{subfiles}

\begin{document}

\section{Aufgabe 7}
Gegeben seien die folgenden Vektoren
$$
    a_1 = \vektor{1\\0\\1\\1}, \;
    a_2 = \vektor{2\\0\\1\\2}, \;
    a_3 = \vektor{1\\1\\1\\1}, \;
    b   = \vektor{1\\0\\0\\ \lambda}, \;
    \lambda \in \mathbb{R}
$$
sowie die zugehörigen LGS
$$
    \text{(1) } (a_1, a_2, a_3)x = b
    \quad \text{und} \quad
    \text{(2) } (a_1, a_2, a_3, b)x = 0.
$$
\begin{enumerate}
    \item Berechnen Sie $\det (a_1, a_2, a_3, b)$ in Abhängigkeit von $\lambda$ und bestimmen Sie den Wert $\lambda^*$, für den die Determinante Null wird.
    \item Bestimmen Sie die Lösbarkeit der LGS (1) und (2) für den Fall $\lambda = \lambda^*$.
    \item Bestimmen Sie die Lösbarkeit der LGS (1) und (2) für den Fall $\lambda \neq \lambda^*$.
    \item Geben Sie die Lösungsmengen für b) und c) an, sollten unendlich viele Lösungen existieren.
\end{enumerate}


% 1 & 2 & 1 & 1
% 0 & 0 & 1 & 0
% 1 & 1 & 1 & 0
% 1 & 2 & 1 & \lambda

% {{1,2,1,1},{0,0,1,0},{1,1,1,0},{1,2,1,a}}

\subsection{Lösung 7a}
Zur Berechnung der Determinanten bietet sich die Anwendung des Laplaceschen Entwicklungssatzes mit der zweite Zeile an.
\begin{align*}
    \det  (a_1, a_2, a_3, b) &= \det \eqmatrix{cccc}{
        1 & 2 & 1 & 1 \\
        0 & 0 & 1 & 0 \\
        1 & 1 & 1 & 0 \\
        1 & 2 & 1 & \lambda \\
    }\\[2mm]
    &= 1 \cdot \det \eqmatrix{ccc}{
        1 & 2 & 1 \\
        1 & 1 & 0 \\
        1 & 2 & \lambda \\
    } \\[2mm]
    &= \lambda + 0 + 2 -1 -0 -2\lambda \\
    &= 1-\lambda
\end{align*}

Somit ist $\lambda^* = 1$.

\subsection{Lösung 7b}
Wir untersuchen das LGS (1) mit $\lambda = \lambda^* = 1$. Sei $A=(a_1, a_2, a_3)$, so bestimmen wir zunächst $\rg (A) = 3 = \rg (A, b)$. 
Da die Anzahl der Spalten der Matrix $A$ gleich ihrem Rang ist $\rg(A) = n$, existiert genau eine Lösung, nämlich die Triviallösung $x=0$.\\

Da für das LGS (2) die Determinante, wie in 7a) gezeigt, $\det (a_1, a_2, a_3, b) = 0$ ist, betrachten wir den Rang $\rg(A) = 3 = \rg(A, \vec{0})$.
Daher existieren unendlich viele Lösungen mit $n-\rg(A) = 1$ Parametern.

\subsection{Lösung 7c}
Für $\lambda \neq \lambda^*$ ist das LGS (1) mit $\rg (A) = 3 \neq 4 = \rg (A,b)$ nicht lösbar. Es existiert also keine Lösung.\\

Da die Determinante für das LGS (2) $\det (a_1, a_2, a_3, b) \neq 0$ ist, existiert genau eine Lösung, nämlich die Triviallösung $x=0$.

\subsection{Lösung 7d}
Bestimme die Lösungsmenge für das LGS (2) $(a_1, a_2, a_3, b) x = 0$ mit $\lambda = \lambda^* = 1$.

% \begin{align*}
%     \eqmatrix{cccc|c}{
%         1 & 2 & 1 & 1 & 0 \\
%         0 & 0 & 1 & 0 & 0 \\
%         1 & 1 & 1 & 0 & 0 \\
%         1 & 2 & 1 & 1 & 0 \\
%     }
% \end{align*}

%$$
%    \mathcal{L} = \left\{ \mu \cdot \vektor{0 \\ -\nicefrac{1}{2} \\ 0 \\ 1} \middle| \mu \in \mathbb{R} \right\}
%$$

$$
    \mathcal{L} = \left\{ \mu \cdot \eqmatrix{r}{1 \\ -1 \\ 0 \\ 1} \middle| \mu \in \mathbb{R} \right\}
$$

\end{document}
