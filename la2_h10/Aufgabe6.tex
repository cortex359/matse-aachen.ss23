\documentclass[main.tex]{subfiles}

\begin{document}

\section{Aufgabe 6}
Welchen Rang haben die Matrizen $A\in \mathbb{R}^{4\times 3}$ und $B\in \mathbb{R}^{n\times 2}$ mit $n\geq 2$?

\begin{enumerate}
    \item $A=\eqmatrix{rrr}{
        4 & 3 & 6 \\
        -2 & 1 & -8 \\
        1 & 5 & -7 \\
        4 & 2 & -1 \\
    }$ \\
    \item $b_{ij} = \begin{cases}
        -i & i\  \text{gerade} \\
         i & i\  \text{ungerade} \\
         n & i=n,\  j=1 \\
         n-1 & i=n,\  j=2 \\
    \end{cases}$
\end{enumerate}

\subsection{Lösung 6a}
Durch Anwendung des Gauß-Jordan-Verfahrens lässt sich eine Nullzeile wegstreichen und es bleiben drei lineare unabhängige Zeilen übrigen,
sodass $\rg(A) = 3$ ist.

\begin{align*}
    \eqmatrix{rrr}{
        4 & 3 & 6 \\
        -2 & 1 & -8 \\
        1 & 5 & -7 \\
        4 & 2 & -1 \\
    } \leadsto
    \eqmatrix{rrr}{
        4 & 3 & 6 \\
        0 & 11 & -22 \\
        1 & 5 & -7 \\
        0 & -1 & -7 \\
    } \leadsto
    \eqmatrix{rrr}{
        0 & -17 & -22 \\
        0 & 0 & -99 \\
        1 & 5 & -7 \\
        0 & -1 & 0 \\
    } \leadsto
    \eqmatrix{rrr}{
        0 & 0 & 0 \\
        0 & 0 & 1 \\
        1 & 0 & 0 \\
        0 & 1 & 0 \\
    }
\end{align*}

\subsection{Lösung 6b}

Wir betrachten die Matrizen $B_n$ mit $n\in \{2, 3, 4\}$
% Zeile x Spalte 
\begin{align*}
    B_2 &= \eqmatrix{rr}{
        1 & 1 \\
        2 & 1 
    } \\
    B_3 &= \eqmatrix{rr}{
         1 &  1 \\
        -2 & -2 \\
         3 &  2
    } \\
    B_4 &= \eqmatrix{rr}{
         1 &  1 \\
        -2 & -2 \\
         3 &  3 \\
         4 &  3
    }
\end{align*}

sowie eine allgemeine Form für $n > 4$
\begin{align*}
    B_n &= \eqmatrix{rr}{
        1 &  1 \\
       -2 & -2 \\
        3 &  3 \\
       -4 & -4 \\
       \vdots & \vdots \\
       n &  n-1 
   }
\end{align*}

Es ist leicht erkennbar das die Zeilen $i \in \{1, \ldots, n-1\}$ linear abhängig voneinander sind und auf die erste Zeile reduziert werden können. 
Somit ist 
\begin{align*}
    \rg (B) &= \rg \eqmatrix{cc}{
        1 &  1 \\
        n &  n-1 
    } = 2.
\end{align*}

Alternativ lässt isch auch mit Satz 6.24 argumentieren, nämlich dass $\rg(A) = \rg_S(A) \leq 2$ und durch die erste und letzte Zeile $\rg_S(A) \geq 2$ sein muss, 
also $\rg(A) = 2$ ist.

\end{document}
