\documentclass[main.tex]{subfiles}

\begin{document}

\section{Aufgabe 5}
Lösen Sie das Gleichungssystem
$$
    \eqmatrix{rcrcrcr}{
        x & + & 3y & + & 3z & = & -2 \\
        x & + & 2y & + & 4z & = &  3 \\
        x & + &  y & + &  z & = &  0 \\
    }
$$
\begin{enumerate}
    \item nach dem Gauß-Verfahren
    \item nach der Cramerschen Regel und
    \item durch Invertierung von der Abbildungsmatrix.
\end{enumerate}

\subsection{Lösung 5a}
\begin{align*}
    \eqmatrix{rrr|r}{
        1 & 3 & 3 & -2 \\
        1 & 2 & 4 &  3 \\
        1 & 1 & 1 &  0 \\
    } &\leadsto 
    \eqmatrix{rrr|r}{
        1 &  3 &  3 & -2 \\
        0 & -1 &  1 &  5 \\
        0 & -2 & -2 &  2 \\
    } \leadsto 
    \eqmatrix{rrr|r}{
        1 &  3 &  3 & -2 \\
        0 &  1 & -1 & -5 \\
        0 &  0 & -4 & -8 \\
    } \\[2mm]
    &\leadsto
    \eqmatrix{rrr|r}{
        1 & 0 & 3 &  7 \\
        0 & 1 & 0 & -3 \\
        0 & 0 & 1 &  2 \\
    } \leadsto
    \eqmatrix{rrr|r}{
        1 & 0 & 0 &  1 \\
        0 & 1 & 0 & -3 \\
        0 & 0 & 1 &  2 \\
    } \\[2mm]
    &\Rightarrow x=1,\quad y=-3,\quad z=2
\end{align*}

\subsection{Lösung 5b}
Sei 
$$
    A = \eqmatrix{rrr}{
        1 & 3 & 3\\
        1 & 2 & 4\\
        1 & 1 & 1\\
    }, \quad
    b = \eqmatrix{r}{-2 \\ 3 \\ 0}, \quad
    x = \vektor{x_1 \\ x_2 \\ x_3}
$$
sowie $Ax=b$. 

Um die Cramersche Regel anzuwenden berechnen wir zunächst $\abs{A}$ mit der Regel von Sarrus
$$
    \abs{A} = 2 + 12 + 3 -6 -4 -3 = 4.
$$

Betrachten wir nun die Spaltenvektoren von $A$ mit $A=(a_1, a_2, a_3) \in K^{3\times 3}$ und bestimmen $A_i$ mit $i\in \left\{1, 2, 3\right\}$ nach Satz 6.28.

\begin{align*}
    A_1 &= (  b, \; a_2, \; a_3) \\
    A_2 &= (a_1, \;   b, \; a_3) \\
    A_3 &= (a_1, \; a_2, \; b) \\
\end{align*}

Durch Anwendung der Regel von Sarrus bestimmen wir die Determinanten wie folgt:

% 1 & 3 & 3 | -2 \\
% 1 & 2 & 4 |  3 \\
% 1 & 1 & 1 |  0 \\

\begin{align*}
    \abs{A_1} &= \det (  b, \; a_2, \; a_3) \\
              &= \det \eqmatrix{rrr}{
                -2 & 3 & 3 \\
                 3 & 2 & 4 \\
                 0 & 1 & 1 \\
              } \\
              &= -4 + 9 -0 - (-8) -9 \\
              &= 4 \\[5mm]
    \abs{A_2} &= \det (a_1, \;   b, \; a_3) \\
              &= \det \eqmatrix{rrr}{
                1 & -2 & 3 \\
                1 &  3 & 4 \\
                1 &  0 & 1 \\
              } \\
              &= 3 + (-8) + 0 -9 -0 -(-2) \\
              &= -12 \\[5mm]
    \abs{A_3} &= \det (a_1, \; a_2, \; b) \\
              &= \det \eqmatrix{rrr}{
                1 & 3 & -2 \\
                1 & 2 &  3 \\
                1 & 1 &  0 \\
              } \\
              &= 0 + 9 + (-2) - (-4) -3 -0 \\
              &= 8
\end{align*}

Mit $x_i = \frac{\det A_i}{\det A}$ ergibt sich:

\begin{align*}
    x_1 &= \frac{\abs{A_1}}{\abs{A}} = \frac{4}{4} = 1 \\
    x_2 &= \frac{\abs{A_2}}{\abs{A}} = \frac{-12}{4} = -3 \\
    x_3 &= \frac{\abs{A_3}}{\abs{A}} = \frac{8}{4} = 2 \\
\end{align*}

\subsection{Lösung 5c}
\renewcommand{\equiv}{\Leftrightarrow}

Das Gleichungssystem $Ax=b$ löst sich durch Multiplikation von links mit dem Inversen von $A$ zu $x=A^{-1}b$ auf:

\begin{equation*}
\begin{array}{rrl}
        & Ax &= b \\
\equiv  & A^{-1} A x &= A^{-1} b \\
\equiv  & E x &= A^{-1} b \\
\equiv  & x &= A^{-1} b \\
\end{array}
\end{equation*}

Wir bestimmen $A^{-1}$.

\begin{align*}
    \eqmatrix{rrr|rrr}{
        1 & 3 & 3 & 1 & 0 & 0 \\
        1 & 2 & 4 & 0 & 1 & 0 \\
        1 & 1 & 1 & 0 & 0 & 1 \\
    } &\leadsto 
    \eqmatrix{rrr|rrr}{
        1 &  3 &  3 &  1 & 0 & 0 \\
        0 & -1 &  1 & -1 & 1 & 0 \\
        0 & -2 & -2 & -1 & 0 & 1 \\
    } \\
    &\leadsto 
    \eqmatrix{rrr|rrr}{
        1 &  3 &  3 &  1 &  0 & 0 \\
        0 &  1 & -1 &  1 & -1 & 0 \\
        0 &  0 & -4 &  1 & -2 & 1 \\
    } \\
    &\leadsto 
    \eqmatrix{rrr|rrr}{
        1 &  3 &  3 &                 1 &                0 & 0 \\
        0 &  1 &  0 &   \nicefrac{3}{4} & -\nicefrac{1}{2} & -\nicefrac{1}{4} \\
        0 &  0 &  1 &  -\nicefrac{1}{4} &  \nicefrac{1}{2} & -\nicefrac{1}{4} \\
    } \\
    &\leadsto 
    \eqmatrix{rrr|rrr}{
        1 &  0 &  3 &  -\nicefrac{5}{4} &  \nicefrac{3}{2} &  \nicefrac{3}{4} \\
        0 &  1 &  0 &   \nicefrac{3}{4} & -\nicefrac{1}{2} & -\nicefrac{1}{4} \\
        0 &  0 &  1 &  -\nicefrac{1}{4} &  \nicefrac{1}{2} & -\nicefrac{1}{4} \\
    } \\
    &\leadsto 
    \eqmatrix{rrr|rrr}{
        1 &  0 &  0 &  -\nicefrac{1}{2} &                0 &  \nicefrac{3}{2} \\
        0 &  1 &  0 &   \nicefrac{3}{4} & -\nicefrac{1}{2} & -\nicefrac{1}{4} \\
        0 &  0 &  1 &  -\nicefrac{1}{4} &  \nicefrac{1}{2} & -\nicefrac{1}{4} \\
    }
\end{align*}

$$
    A^{-1} = \frac{1}{4} \eqmatrix{rrr}{
        -2 &  0 &  6 \\
         3 & -2 & -1 \\
        -1 &  2 & -1 \\
    }
$$

Durch Anwendung der zuvor bestimmten Gleichung erhalten wir den Lösungsvektor.
\begin{align*}
    x &= A^{-1}b\\
    &= \frac{1}{4} \eqmatrix{rrr}{
        -2 &  0 &  6 \\
         3 & -2 & -1 \\
        -1 &  2 & -1 \\
    } \eqmatrix{r}{-2 \\ 3 \\ 0}\\
    &=  \frac{1}{4} \eqmatrix{r}{
        4 \\ -12 \\ 8
    }\\
    &= \eqmatrix{r}{
        1 \\ -3 \\ 2
    }\\
\end{align*}

\end{document}
