\documentclass[main.tex]{subfiles}

\begin{document}

\section{Aufgabe 2}
% Kugelkoordinaten
Gegeben sei eine Halbkugel mit dem Radius $R$, deren Schnittfläche in einem kartesischen Koordinatensystem auf der $xy$-Ebene liegt. Berechnen Sie den Schwerpunkt dieser Halbkugel.

\subsection{Lösung 2}
Für das Rechnen mit Kugelkoordinaten, mit $\phi$ \textbf{auf} der $x$-$y$-Ebene und dem Winkel $\theta$ von der $z$-Achse auf die $x$-$y$-Ebene gilt:
\begin{align*}
    \iiint f(x,y,z) \dx{(z,y,x)} &=
        \iiint f(r\cos\phi\sin\theta, r\sin\phi\sin\theta, r\cos\theta )\cdot |r^2\sin\theta| \dx{(\theta, \phi, r)}
\end{align*}

Wir integrieren über das Gebiet
\[
    H = \left\{
        r \in \interval*{0; R} \land
        \theta \in \interval*{0; \frac{\pi}{2}} \land
        \phi \in \interval*{0; 2\pi}
    \right\}.
\]

Die $x_S$ und $y_S$ Koordinaten des Schwerpunkts liegen im Mittelpunkt der Kugel, also im Koordinatenursprung.

Für die Koordinate in Richtung der $z$-Achse gilt:
\begin{equiveqs}[rl]
    z_S &= \frac{
        \iiint \limits_H z \dx{(z,y,x)}
    }{
        \iiint \limits_H 1 \dx{(z,y,x)}
    }
\end{equiveqs}

Wir berechnen den Zähler.
\begin{align*}
\iiint \limits_H z \dx{(z,y,x)}
    &=
    \int_{r=0}^{R}
    \int_{\phi=0}^{2\pi}
    \int_{\theta=0}^{\frac{\pi}{2}}
        r\cdot \cos\theta
        \cdot \abs*{r^2\sin\theta}
    \dx{(\theta, \phi, r)} \\
    &=
    \int_{r=0}^{R}
    \int_{\phi=0}^{2\pi}
    \left[
        \frac{r^3}{2} \cdot \sin(\theta) \cdot \abs*{\sin(\theta)}
    \right]_{\theta=0}^{\frac{\pi}{2}}
    \dx{(\phi, r)} \\
    &=
    \int_{r=0}^{R}
    \int_{\phi=0}^{2\pi}
        \frac{r^3}{2}
    \dx{(\phi, r)} \\
    &=
    \int_{r=0}^{R}
    \left[
        \frac{r^3}{2} \phi
    \right]_{\phi=0}^{2\pi}
    \dx{r} \\
    &=
    \int_{r=0}^{R}
        r^3 \cdot \pi
    \dx{r} \\
    &=
    \left[
        \frac{r^4}{4} \cdot \pi
    \right]_{r=0}^{R} \\
    &=
        \frac{1}{4} R^4 \cdot \pi
\end{align*}

Wir berechnen den Nenner:
\begin{align*}
\iiint \limits_H 1 \dx{(z,y,x)}
    &=
    \int_{r=0}^{R}
    \int_{\phi=0}^{2\pi}
    \int_{\theta=0}^{\frac{\pi}{2}}
        1 \cdot \abs*{r^2\sin\theta}
    \dx{(\theta, \phi, r)} \\
    &=
    \int_{r=0}^{R}
    \int_{\phi=0}^{2\pi}
    \int_{\theta=0}^{\frac{\pi}{2}}
        r^2 \cdot \sin\theta
    \dx{(\theta, \phi, r)} \\
    &=
    \int_{r=0}^{R}
    \int_{\phi=0}^{2\pi}
    \left[
        - r^2 \cdot \cos\theta
    \right]_{\theta=0}^{\frac{\pi}{2}}
    \dx{(\phi, r)} \\
    &=
    \int_{r=0}^{R}
    \int_{\phi=0}^{2\pi}
        r^2
    \dx{(\phi, r)} \\
    &=
    \int_{r=0}^{R}
    \left[
        r^2 \cdot \phi
    \right]_{\phi=0}^{2\pi}
    \dx{r} \\
    &=
    \int_{r=0}^{R}
        r^2 \cdot 2\pi
    \dx{r} \\
    &=
    \left[
        \frac{2}{3} r^3 \cdot \pi
    \right]_{r=0}^{R} \\
    &=
        \frac{2}{3} R^3 \cdot \pi
\end{align*}

Somit erhalten wir für
\begin{equiveqs}[rcl]
    z_S &=& \frac{
        \iiint \limits_H z \dx{(z,y,x)}
    }{
        \iiint \limits_H 1 \dx{(z,y,x)}
    } \\[8mm]
    &=& \frac{
        \frac{1}{4} R^{\cancel{4}} \cdot \cancel{\pi}
    }{
        \frac{2}{3} \cancel{R^3} \cdot \cancel{\pi}
    } \\[6mm]
    &=& \frac{3}{8} R.
\end{equiveqs}

Der Schwerpunkt der Halbkugel $S = (x_S|y_S|z_S)$ liegt also bei $\left(0 \middle| 0 \middle| \frac{3}{8} R\right)$.

\end{document}
