\documentclass[main.tex]{subfiles}

\begin{document}

\section{Aufgabe 5}
Gegeben sei das Vektorfeld/ Kraftfeld
\begin{equation*}
\vec{F}(x,y)
= \left( \begin{array}{c}
	x^2 + 2x \cdot y - y^2 \\ 
	x^2 - 2x \cdot y - y^2
\end{array} \right) 
\end{equation*}
\begin{enumerate}
\item Überprüfen Sie, ob Kurvenintegrale in $\vec{F}$ wegunabhängig sind.
\item Ermitteln Sie gegebenenfalls eine Potentialfunktion.
\item Berechnen Sie die Arbeit zwischen den Punkten $A = (0, 1)$ zu $E = (1, 0)$ über die Potentialfunktion oder als Wert des Kurvenintegrals über ein Geradenstück von $A$ nach $E$.
\end{enumerate}

\subsection{Lösung 5a}
Ein Gradientenfeld ist konservativ, wenn es eine Funktion $f(x, y) : D \to \R$ gibt, mit der $F=\nabla f$.

Wir bestimmen $\nabla f$ und prüfen die Integrabilitätsbedingung:
\begin{align*}
    \nabla f &= \vektor{f_x \\ f_y} \\
    \frac{\partial \vec{f_x}}{\partial y} &=
        2x -2y \\
    \frac{\partial \vec{f_y}}{\partial x} &=
        2x -2y
\end{align*}

Die Integrabilitätsbedingung ist erfüllt und das Vektorfeld damit konservativ, also Kurvenintegrale wegunabhängig.

\subsection{Lösung 5b}
\begin{align*}
    \begin{array}{rrl}
        & V(x,y) & = \int \vec{F_1} \dx{x}\\[2mm]
        && = \frac{1}{3}x^3+x^2y-xy^2 + c(y)\\[2mm]
        \Rightarrow & \frac{\dx{V}}{\dx{y}} & = x^2-2xy+c'(y)\\[2mm]
        & = \vec{F_2} & = x^2-2xy-y^2\\
        \Rightarrow & c'(y) & = -y^2\\
        \Rightarrow & c(y) & = -\frac{1}{3}y^3+c\\[2mm]
        \Rightarrow & V(x,y) & = \frac{1}{3}x^3+x^2y-xy^2-\frac{1}{3}y^3+c
    \end{array}
\end{align*}

\subsection{Lösung 5c}
\begin{align*}
    \begin{array}{rl}
        W & =V(E)-V(A)\\
        & = V(1, 0) - V(0, 1)\\[2mm]
        & = \frac{1}{3} - \frac{1}{3}\\[2mm]
        & = 0
    \end{array}
\end{align*}

\end{document}
