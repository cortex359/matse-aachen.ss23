\documentclass[main.tex]{subfiles}

\begin{document}

\section{Aufgabe 4}
% Analysis 2-Klausur vom 25.03.2021, A09
Der Graph von $y = \sin(x)$ beschreibt eine Kurve $K$ in der $x$-$y$-Ebene: 
\[
\vec{X}(t)
= \left( \begin{array}{cc}
	t \\ 
	\sin(t)
\end{array} \right)
%
\qquad \qquad
%
t \in \left[ 0,\frac{\pi}{2}\right] 
\]
%
Man berechne das Kurvenintegral $\int_K\vec{v} \ d\vec{X}$ für 
$\vec{v}(x,y)
= \left( \begin{array}{cc}
	y \cdot \cos \left( x \right) + y \\ 
	\sin \left( x \right) +x+2
\end{array} \right)$
mit Hilfe einer Potentialfunktion (im Falle der Existenz).

\subsection{Lösung 4}
Potenzialfunktion existiert, da
\[
	\frac{\dx{\vec{v_1}}}{\dx{y}} = \cos(x)+1 = \frac{\dx{\vec{v_2}}}{\dx{x}}
\]
Bestimmung der Potenzialfunktion:
\begin{align*}
	\begin{array}{rrl}
		& V(x,y) & =  \int \vec{v_1} dx\\
		&& = y\cdot \sin(x) + xy+c(y)\\
		\Rightarrow & \frac{\dx{V}}{\dx{y}} & = \sin(x) + x + c'(y)\\
		&& = \vec{v_2} = sin(x) + x + 2\\
		\Rightarrow & c'(y) & = 2\\
		\Rightarrow & c(y) & = 2y+c\\
		\Rightarrow & V & = y\cdot \sin(x) + xy + 2y+c
	\end{array}
\end{align*}
Bestimmung des Anfangs- und Endpunktes:
\begin{align*}
	\begin{array}{rl}
		A & = \vec{X}(0) = \vektor{0\\0}\\
		E & = \vec{X}\left(\frac{\pi}{2}\right) = \vektor{\sfrac{\pi}{2}\\1}
	\end{array}
\end{align*}
Einsetzen in $\int_K \vec{v} \dx{\vec{X}} = V(E)-V(A)$:
\begin{align*}
	\begin{array}{rl}
		\int_K \vec{v} \dx{\vec{X}} & = V\left(\frac{\pi}{2}, 1\right) - V(0,0)\\[2mm]
		& = 1 + \frac{\pi}{2} + 2\\[2mm]
		& = \frac{6+\pi}{2}
	\end{array}
\end{align*}

\end{document}
