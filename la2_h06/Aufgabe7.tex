\documentclass[main.tex]{subfiles}

\begin{document}

\section{Aufgabe 7}
Es sei $S(x)$ eine Spiegelung an der Ebene $\scalarprod{n,x} = 0$ mit $\norm{n} = 1$.\\
Dann gilt $S:\mathbb{R}^3 \to \mathbb{R}^3, S(x) = x-2 \scalarprod{x,n} n$.

\begin{enumerate}
    \item Bezüglich der Einheitsmatrix im Definitions- und Wertebereich, zeigen Sie:
    $S(x) = M\cdot x$ mit $M = E-2\cdot n \cdot n^T$. $E$ bezeichnet die Einheitsmatrix. 
    % Hinweis: Allgemein gilt für x ∈ R³ : E · x = x.
    \item Bestimmen Sie für $n = \frac{1}{\sqrt{6}} (1, 2, 1)^T$ die zu $M$ inverse Matrix $M^{-1}$.
    \item Bestimmen Sie die Matirx $B^{-1}$, mit welcher Koordinaten bezüglich der Standardbasis in Koordinaten bezüglich der Basis
    $\mathcal{B} = \left( \eqmatrix{c}{1 \\ 0 \\ 0}, \eqmatrix{c}{1 \\ 2 \\ 0}, \eqmatrix{c}{1 \\ 2 \\ 3} \right)$ umgerechnet werden.
    \item Bestimmen Sie nun die Matrix $M^{-1}$ bezüglich der Basis $\mathcal{B}$ im Definitions- und Wertebereich.
\end{enumerate}

\subsection{Lösung 7}

\end{document}
