\documentclass[main.tex]{subfiles}

\begin{document}

\section{Aufgabe 7}
Es sei $S(x)$ eine Spiegelung an der Ebene $\scalarprod{n,x} = 0$ mit $\norm{n} = 1$.\\
Dann gilt $S:\mathbb{R}^3 \to \mathbb{R}^3, S(x) = x-2 \scalarprod{x,n} n$.

\begin{enumerate}
    \item Bezüglich der Einheitsmatrix im Definitions- und Wertebereich, zeigen Sie:
    $S(x) = M\cdot x$ mit $M = E-2\cdot n \cdot n^T$. $E$ bezeichnet die Einheitsmatrix.
    % Hinweis: Allgemein gilt für x ∈ R³ : E · x = x.
    \item Bestimmen Sie für $n = \frac{1}{\sqrt{6}} (1, 2, 1)^T$ die zu $M$ inverse Matrix $M^{-1}$.
    \item Bestimmen Sie die Matirx $B^{-1}$, mit welcher Koordinaten bezüglich der Standardbasis in Koordinaten bezüglich der Basis
    $\mathcal{B} = \left( \eqmatrix{c}{1 \\ 0 \\ 0}, \eqmatrix{c}{1 \\ 2 \\ 0}, \eqmatrix{c}{1 \\ 2 \\ 3} \right)$ umgerechnet werden.
    \item Bestimmen Sie nun die Matrix $M^{-1}$ bezüglich der Basis $\mathcal{B}$ im Definitions- und Wertebereich.
\end{enumerate}

\subsection{Lösung 7a}

Es gilt $n\in \mathbb{R}^3$ und $S(x) = x - 2\cdot \left(x_1n_1 + x_2n_2 + x_3n_3\right) \cdot n$. Durch ausmultiplizieren ergibt sich:

\begin{align*}
    S(x) &= \vektor{x_1 \\ x_2 \\ x_3} - 2\cdot \vektor{
        \left(x_1n_1 + x_2n_2 + x_3n_3\right) \cdot n_1 \\
        \left(x_1n_1 + x_2n_2 + x_3n_3\right) \cdot n_2 \\
        \left(x_1n_1 + x_2n_2 + x_3n_3\right) \cdot n_3} \\
         &= \vektor{x_1 \\ x_2 \\ x_3} - 2\cdot \vektor{
            x_1n_1n_1 + x_2n_2n_1 + x_3n_3n_1 \\
            x_1n_1n_2 + x_2n_2n_2 + x_3n_3n_2 \\
            x_1n_1n_3 + x_2n_2n_3 + x_3n_3n_3} \\
        &= \vektor{x_1 \\ x_2 \\ x_3} - 2\cdot \vektor{
            x_1n_1^2  + x_2n_2n_1 + x_3n_3n_1 \\
            x_1n_1n_2 + x_2n_2^2  + x_3n_3n_2 \\
            x_1n_1n_3 + x_2n_2n_3 + x_3n_3^2} \\
\end{align*}


Für die Abbildungsmatrix $M$ mit $M=E -2\cdot n\cdot n^T$ ergibt sich Schrittweise
\begin{align*}
    n\cdot n^T &= \vektor{n_1 \\ n_2 \\ n_3} \cdot \vektor{n_1 & n_2 & n_3} \\
               &= \eqmatrix{ccc}{
                    n_1^2   & n_1 n_2 & n_1 n_3 \\
                    n_2 n_1 & n_2^2   & n_2 n_3 \\
                    n_3 n_1 & n_3 n_2 & n_3^2
                }\\
    M = E -2\cdot n\cdot n^T &= \eqmatrix{ccc}{1 & 0 & 0 \\ 0 & 1 & 0 \\ 0 & 0 & 1} - 2\cdot \eqmatrix{ccc}{
        n_1^2   & n_1 n_2 & n_1 n_3 \\
        n_2 n_1 & n_2^2   & n_2 n_3 \\
        n_3 n_1 & n_3 n_2 & n_3^2
    }\\
        &= \eqmatrix{ccc}{
            1-2 n_1^2   & -2 n_1 n_2 & -2 n_1 n_3 \\
            -2 n_2 n_1 & 1-2 n_2^2   & -2 n_2 n_3 \\
            -2 n_3 n_1 & -2 n_3 n_2 & 1-2 n_3^2
        }\\
    S(x)=M\cdot x &= \eqmatrix{ccc}{
        1-2 n_1^2   & -2 n_1 n_2 & -2 n_1 n_3 \\
        -2 n_2 n_1 & 1-2 n_2^2   & -2 n_2 n_3 \\
        -2 n_3 n_1 & -2 n_3 n_2 & 1-2 n_3^2
    } \cdot \vektor{x_1 \\ x_2 \\ x_3}\\
     &= \vektor{
        (\textcolor{gruen}{1}-2 n_1^2) \textcolor{gruen}{x_1} - 2 n_1 n_2  x_2 - 2 n_1 n_3  x_3\\
        -2 n_2 n_1  x_1 + (\textcolor{gruen}{1}-2 n_2^2) \textcolor{gruen}{x_2} - 2 n_2 n_3  x_3\\
        -2 n_3 n_1  x_1 - 2 n_3 n_2  x_2 + (\textcolor{gruen}{1}-2 n_3^2) \textcolor{gruen}{x_3}
    }\\
    &= \vektor{\textcolor{gruen}{x_1} \\ \textcolor{gruen}{x_2} \\ \textcolor{gruen}{x_3}} + \vektor{
        -2 n_1^2    x_1 -2 n_1 n_2  x_2 -2 n_1 n_3  x_3\\
        -2 n_2 n_1  x_1 -2 n_2^2    x_2 -2 n_2 n_3  x_3\\
        -2 n_3 n_1  x_1 -2 n_3 n_2  x_2 -2 n_3^2    x_3
    }\\
    &= \vektor{x_1 \\ x_2 \\ x_3} - 2\cdot \vektor{
        n_1^2    x_1 + n_1 n_2  x_2 + n_1 n_3  x_3\\
        n_2 n_1  x_1 + n_2^2    x_2 + n_2 n_3  x_3\\
        n_3 n_1  x_1 + n_3 n_2  x_2 + n_3^2    x_3
    }
\end{align*}

wodruch gezeigt ist, dass $S(x) = M\cdot x = x-2 \scalarprod{x,n} n$. $\checkmark$


\subsection*{Lösung 7b}

Für $n = \frac{1}{\sqrt{6}} (1, 2, 1)^T$ ist nach dem zuvor gezeigten

\begin{align*}
    M &= E -2 \cdot \frac{1}{\sqrt{6}}\cdot \vektor{1 \\ 2 \\ 1} \cdot \frac{1}{\sqrt{6}}\cdot \vektor{1 & 2 & 1} \\
      &= E - \frac{1}{3} \cdot \eqmatrix{rrr}{
        1 & 2 & 1 \\
        2 & 4 & 2 \\
        1 & 2 & 1} \\
      &= \frac{1}{3} \cdot \eqmatrix{rrr}{
        2 & -2 & -1 \\
        -2 & -1 & -2 \\
        -1 & -2 & 2}.
\end{align*}

Diese Matrix ist ihre eigene Inverse, es gilt: $M = M^{-1}$.


\subsection*{Lösung 7c}
Mit der Standardbasis $A = \mathcal{E}$ gilt:
$$
    T^{\mathcal{E}}_{\mathcal{B}} = B^{-1} A.
$$

\begin{align*}
    & \eqmatrix{ccc|ccc}{
        1 & 1 & 1    &    1 & 0 & 0 \\
        0 & 2 & 2    &    0 & 1 & 0 \\
        0 & 0 & 3    &    0 & 0 & 1
    }\\
    \leadsto & \eqmatrix{ccc|ccc}{
        1 & 1 & 1    &    1 & 0   & 0 \\
        0 & 1 & 0    &    0 & \nicefrac{1}{2} & \nicefrac{-1}{3} \\
        0 & 0 & 1    &    0 & 0   & \nicefrac{1}{3}
    }\\
    \leadsto & \eqmatrix{ccc|ccc}{
        1 & 0 & 1    &    1 & \nicefrac{-1}{2} & \nicefrac{1}{3} \\
        0 & 1 & 0    &    0 & \nicefrac{1}{2}  & \nicefrac{-1}{3} \\
        0 & 0 & 1    &    0 & 0    & \nicefrac{1}{3}
    }\\
    \leadsto & \eqmatrix{ccc|ccc}{
        1 & 0 & 0    &    1 & \nicefrac{-1}{2} & 0 \\
        0 & 1 & 0    &    0 & \nicefrac{1}{2}  & \nicefrac{-1}{3} \\
        0 & 0 & 1    &    0 & 0    & \nicefrac{1}{3}
    }\\
    \Rightarrow B^{-1} = & \eqmatrix{ccc}{
        1 & \nicefrac{-1}{2} & 0 \\
        0 & \nicefrac{1}{2} & \nicefrac{-1}{3} \\
        0 & 0 & \nicefrac{1}{3}
    }\\
\end{align*}

\subsection*{Lösung 7d}
Aus Satz 4.81 ergibt sich, dass $M^{-1} = B^{-1} M^{-1} B$, also

\begin{align*}
    M^{-1} &= \eqmatrix{ccc}{
        1 & \nicefrac{-1}{2} & 0 \\
        0 & \nicefrac{1}{2} & \nicefrac{-1}{3} \\
        0 & 0 & \nicefrac{1}{3}
    } \eqmatrix{ccc}{
        1-2 n_1^2   & -2 n_1 n_2 & -2 n_1 n_3 \\
        -2 n_2 n_1 & 1-2 n_2^2   & -2 n_2 n_3 \\
        -2 n_3 n_1 & -2 n_3 n_2 & 1-2 n_3^2
    }^{-1} \eqmatrix{ccc}{
        1 & 1 & 1 \\
        0 & 2 & 2 \\
        0 & 0 & 3
    }\\
    &= \eqmatrix{lll}{
        \frac{-2 n_2^2 - 2 n_3^2 + 1}{-2 n_1^2 - 2 n_2^2 - 2 n_3^2 + 1} &               \frac{-n_1 n_2}{-2 n_1^2 - 2 n_2^2 - 2 n_3^2 + 1} & 0  \\
                                                                      0 & \frac{-2 n_1^2 - 2 n_3^2 + 1}{-2 n_1^2 - 2 n_2^2 - 2 n_3^2 + 1} & \frac{-4 n_2 n_3}{3 (-2 n_1^2 - 2 n_2^2 - 2 n_3^2 + 1)} \\
                                                                      0 &                                                           0 & \frac{-2 n_1^2 - 2 n_2^2 + 1}{-2 n_1^2 - 2 n_2^2 - 2 n_3^2 + 1} \\
    }
%    &= \frac{1}{3} \eqmatrix{lll}{
%        -3 (2 n_1^2 - n_2 n_1 - 1) & -3 (2 n_1^2 + 3 n_2 n_1 - 2 n_2^2)               & -3 (2 n_1^2 + 3 n_2 n_1 + 6 n_3 n_1 - 2 n_2^2 - 3 n_2 n_3) \\
%                2 n_1 n_3 - 3 n_1 n_2 & -6 n_2^2 - 3 n_1 n_2 + 4 n_3 n_2 + 2 n_1 n_3 + 3 & -6 n_2^2 - 3 n_1 n_2 - 5 n_3 n_2 + 6 n_3^2 + 2 n_1 n_3 \\
%                        -2 n_1 n_3 & -2 (n_1 n_3 + 2 n_2 n_3)                         & -6 n_3^2 - 2 n_1 n_3 - 4 n_2 n_3 + 3
%    }
\end{align*}

%\footnotesize
%$M^{-1} =\frac{1}{3} \eqmatrix{lll}{
%        -3 (2 n_1^2 - n_2 n_1 - 1) & -3 (2 n_1^2 + 3 n_2 n_1 - 2 n_2^2)               & -3 (2 n_1^2 + 3 n_2 n_1 + 6 n_3 n_1 - 2 n_2^2 - 3 n_2 n_3) \\
%                2 n_1 n_3 - 3 n_1 n_2 & -6 n_2^2 - 3 n_1 n_2 + 4 n_3 n_2 + 2 n_1 n_3 + 3 & -6 n_2^2 - 3 n_1 n_2 - 5 n_3 n_2 + 6 n_3^2 + 2 n_1 n_3 \\
%                        -2 n_1 n_3 & -2 (n_1 n_3 + 2 n_2 n_3)                         & -6 n_3^2 - 2 n_1 n_3 - 4 n_2 n_3 + 3}$

Für $n = \frac{1}{\sqrt{6}} (1, 2, 1)^T$ wie in 7b würde sich die folgende Matrix $M_n^{-1}$ ergeben:

$$
    M_n^{-1} = \frac{1}{3} \eqmatrix{ccc}{
        2 & 1  & 0 \\
        0 & -1 & \nicefrac{4}{3} \\
        0 & 0  & 2 \\
    }
$$

\end{document}
