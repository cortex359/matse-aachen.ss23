\documentclass[main.tex]{subfiles}

\begin{document}

\section{Präsenz}

% Aufgabe 1 bis 3: Lineare DGL 1. Ordnung mit konst. Koeffizienten.
% 
% Lineare inhomogene DGL
% $$
% y'+f(x)\cdot y = g(x) 
% $$
% 
% 
% Lineare inhomogene DGL mit kostanten Koeffizent
% $$
%     y'+a\cdot y = g(x) 
% $$

%---

% 1. Lösen homog. DGL $(y' + a \cdot y = 0) => y_h(x)$\\
% 2. Bestimmen partikuläre Lösung $y_p(x)$ mit 
%    - Auswahl vom Typ der rechten Seite 
%    a) $y_P$ und $y'_p$ in DGL einsetzen
%    b) Koeffizientenvergleich
% 3. Gesamtlösung $y(x) = y_h(x) + y_p(x)$

\subsection{Aufgabe 1}
Lösen Sie das Anfangswertproblem 
$$
    y' = 3y + 15x^2 - 5x \qquad y(0) = 3
$$

\subsection{Lösung 1}
Es handelt sich um ein Anfangswertproblem mit einer DGL 1. Ordnung mit konstanten Koeffizienten.

Bestimmte die homogene DGL $y'_h(x) - 3y_h(x) = 0$ durch Trennung der Variablen zu $y_h(x) = c \cdot \ct{e}^3x$.\\

Bestimme die partikuläre Lösung der Form $y_p(x) = c_2 x^2 + c_1 x + c_0$ und $y'_p(x) = 2c_2 x + c_1$.

\begin{equation*}
\begin{array}{rrl}       
        & y'_p(x) - 3y_p(x) &= 15x^2 - 5x \\
\Leftrightarrow & (2c_2 x + c_1) - 3\cdot \left( c_2 x^2 + c_1 x + c_0 \right) &= 15x^2 - 5x \\
\Leftrightarrow & (-3 c_2) x^2 + (2c_2 - 3c_1)x + (c_1 - 3c_0)  &= 15x^2 - 5x \\
\end{array}
\end{equation*}

So erhalten wir für $c_2 =5$, $c_1 = -\frac{5}{3}$ und $c_0 = -\frac{5}{9}$ und somit

$$
    y_p(x) = -5x^2 -\frac{5}{3} x - \frac{5}{9}
$$

Wir setzen die Lösungen zusammen und bestimmen $c$ mit $y(0) = 3 \Leftrightarrow c = \frac{32}{9}$

\begin{align*}
    y(x) &= y_h(x) + y_p(x) \\
         &= c \cdot \ct{e}^3x -5x^2 -\frac{5}{3} x - \frac{5}{9} \\
         &= \frac{32}{9} \ct{e}^3x -5x^2 -\frac{5}{3} x - \frac{5}{9} \\
\end{align*}

\subsection{Lösung 2}

Berechnen Sie die allgemeine Lösung der linearen Differentialgleichungen.

\begin{align*}
    y'-3y &= -3\ct{e}^{3x}
\end{align*}



\end{document}
