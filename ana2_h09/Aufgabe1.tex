\documentclass[main.tex]{subfiles}

\begin{document}

\section{Aufgabe 1}
Gesucht ist die allgemeine Lösung der jeweiligen Differentialgleichung.
\begin{enumerate}
    \item $y' + 2y = 8\sin(x) - \cos(x)$
    \item $y' - 4y = -28\sin(3x) + 21\cos(3x)$
\end{enumerate}

\subsection{Lösung 1a}

Löse die homogene Differentialgleichung $y'_h(x) + 2y_h(x) = 0$ durch \textit{Trennung der Variablen}.
\begin{equation*}
\begin{array}{rrl}
                & y'_h(x) + 2y_h(x) &= 0 \\[2mm]
\Leftrightarrow & y_h(x) &= -\frac{1}{2} \frac{\dx{y_h}}{\dx{x}} \\[2mm]
\Leftrightarrow & 1 \dx{x} &= -\frac{1}{2\cdot y_h(x)} \dx{y_h} \\[2mm]
\Leftrightarrow & \int 1 \dx{x} &= -\frac{1}{2} \int \frac{1}{y_h(x)} \dx{y_h} \\[2mm]
\Leftrightarrow & x + c &= -\frac{1}{2} \ln \abs{y_h(x)} \\[2mm]
\Leftrightarrow & -2x + -2c &= \ln \abs{y_h(x)} \\[2mm]
\Leftrightarrow & \ct{e}^{-2x} \cdot \underbrace{\ct{e}^{-2c}}_{:=\tilde{c} \in \mathbb{R^+}} &= \abs{y_h(x)} \\[7mm]
\Leftrightarrow & \ct{e}^{-2x} \cdot \underbrace{c}_{c \in \mathbb{R}} &= y_h(x) \\[2mm]
\end{array}
\end{equation*}

Die homogene Lösung lautet somit $y_h(x) = c \cdot \ct{e}^{-2x}$.

Bestimme die partikuläre Lösung $y_p(x)$ mittels \textit{Ansatz vom Typ der rechten Seite} für die Störfunktion $g(x) = 8\sin(x) - \cos(x)$.
\begin{align*}
    y_p(x)  &= c_0 \sin (ax) + c_1\cos(ax) \\
    y'_p(x) &= ac_0 \cos (ax) - ac_1\sin(ax)
\end{align*}

Wir setzten den Ansatz für in die DGL ein und bereiten den Koeffizientenvergleich vor.
\begin{equation*}
\begin{array}{rrl}
                & 8\sin(x) - \cos(x) &\overset{!}{=} y'_p(x) + 2y_p(x) \\
\Leftrightarrow & 8\sin(x) - \cos(x) &= ac_0 \cos (ax) - ac_1\sin(ax) +  2c_0 \sin(ax) + 2c_1\cos(ax) \\
\Leftrightarrow & 8\sin(x) - \cos(x) &= \sin(ax)\underbrace{(2c_0 - ac_1)}_{\overset{!}{=}8} + \cos (ax)\underbrace{(ac_0 + 2c_1)}_{\overset{!}{=}-1}\\
\end{array}
\end{equation*}

Mit $a=1$ ergibt sich für $c_0 = 3$ und für $c_1 = -2$. Somit erhalten wir die partikuläre Lösung:

\begin{align*}
    y_p(x)  &= 3\sin (x) -2 \cos(ax) \\
\end{align*}

Zusammengesetzt bedeutet das für die allgemeine Lösung der DGL

\begin{align*}
    y(x) &= y_h(x) + y_p(x) \\
         &= c \cdot \ct{e}^{-2x} + 3\sin (x) -2 \cos(ax) \\
\end{align*}

\subsection{Lösung 1b}
Nach dem zuvor beschriebenen Verfahren lösen wir $y'_h(x) - 4y_h(x) = 0$ zu $y_h(x) = c\cdot \ct{e}^{4x}$ 
und bestimmen mit Koeffizientenvergleich:
\begin{equation*}
    \begin{array}{rrl}
                    & -28\sin(3x) + 21\cos(3x) &\overset{!}{=} y'_p(x) - 4y_p(x) \\
    \Leftrightarrow & -28\sin(3x) + 21\cos(3x) &= ac_0 \cos (ax) - ac_1\sin(ax) - 4c_0 \sin (ax) -4c_1\cos(ax) \\
    \Leftrightarrow & -28\sin(3x) + 21\cos(3x) &= \underbrace{-(ac_1 + 4c_0)}_{\overset{!}{=}-28}\sin(ax) + \underbrace{(ac_0  -4c_1)}_{\overset{!}{=}21}\cos(ax) \\
    \end{array}
\end{equation*}

Wir erhalten die Koeffizienten $a=3, c_0=7, c_1=0$ und können somit mit der partikulären Lösung $y_p(x)=7\sin(3x)$ die allgemeine Lösung zusammensetzen:

\begin{align*}
    y(x) &= y_h(x) + y_p(x) \\
         &= c\cdot \ct{e}^{4x} + 7\sin(3x)
\end{align*}

\end{document}
