\documentclass[main.tex]{subfiles}

\begin{document}

\section{Aufgabe 6}
Sei $A_n=(a_{ij})$ mit $a_{ii} = 2, a_{ij} = -1$ für $\abs{i-j} = 1$ und $a_{ij} = 0$ sonst und $D_n = \det(A_n)$.
\begin{enumerate}
    \item Berechnen Sie $D_n$ für $n= 1, 2, 3$ und stellen Sie eine Vermutung für $D_n$ für größere $n$ auf.
    \item Beweisen Sie Ihre Vermutung.
\end{enumerate}

\subsection{Lösung 6a}

$$
    A_n =
            (a_{ij}) \text{ mit } \begin{cases}
                a_{ij} = 2 \text{ für } i = j \\
                a_{ij} = -1 \text{ für } \abs{i -j} = 1 \\
                a_{ij} = 0 \text { sonst}
            \end{cases}
$$

\begin{align*}
    A = \eqmatrix{ccccc}{
              2 &    -1 & 0 & \dots  & 0 \\
             -1 &     2 & -1 & \ddots & \vdots \\
              0 &    -1 & \ddots & -1 & 0 \\
         \vdots & \ddots & -1 & 2 & -1 \\
              0 & \dots & 0 & -1  & 2 \\
    }
\end{align*}

Es ergibt sich $D_1 = 2$, $D_2 = 3$, sowie $D_3 = 4$.
Es ist zu vermuten, dass $D_n = n+1$ gilt.

\subsection{Lösung 6b}
Zeige $D_n = n+1$ mit Laplace.



\end{document}
