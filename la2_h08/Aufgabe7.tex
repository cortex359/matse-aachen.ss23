\documentclass[main.tex]{subfiles}

\begin{document}

\section{Aufgabe 7}
Es seien 3 Punkte mit den Koordinaten $(x_i, y_i)$ gegeben. Es gilt, dass
$$
    \begin{vmatrix}
          x^2 + y^2   & x   & y   & 1 \\
        x_1^2 + y_1^2 & x_1 & y_1 & 1 \\
        x_2^2 + y_2^2 & x_2 & y_2 & 1 \\
        x_3^2 + y_3^2 & x_3 & y_3 & 1 \\
    \end{vmatrix} = 0
$$
die Gleichung des Kreises ist, der durch die 3 Punkte geht (Umkreis), falls
$$
    \begin{vmatrix}
        x_1 & y_1 & 1 \\
        x_2 & y_2 & 1 \\
        x_3 & y_3 & 1 \\
    \end{vmatrix} \neq 0.
$$

\begin{enumerate}
    \item Bestimmen Sie mithilfe dieser Gleichung den Mittelpunkt und den Radius des Kreises, der durch die Punkte $(0, 0), (2, -2)$ und $(3, \sqrt{3})$ geht.
    \item Was bedeutet die Bedingung für die Gültigkeit der oben genannten Kreisgleichung geometrisch?
\end{enumerate}

\subsection{Lösung 7}
Für einen Kreis mit Mittelpunkt $(a, b)$ und dem Radius $r$ gilt:
$$
    (x-a)^2 + (y-b)^2 = r^2.
$$

\subsection{Lösung 7a}
Für die angegebenen Punkte gilt:
\begin{align*}
    0 &=
    \begin{vmatrix}
           x^2 + y^2 & x &        y & 1 \\
               0 + 0 & 0 &        0 & 1 \\
        2^2 + (-2)^2 & 2 &       -2 & 1 \\
             3^2 + 3 & 3 & \sqrt{3} & 1 \\
    \end{vmatrix}\\
    &=
    \begin{vmatrix}
        x^2 + y^2 & x &        y & 1 \\
                0 & 0 &        0 & 1 \\
                8 & 2 &       -2 & 1 \\
               12 & 3 & \sqrt{3} & 1 \\
    \end{vmatrix}\\
    &=
    \begin{vmatrix}
        x^2 + y^2 & x &        y \\
                8 & 2 &       -2 \\
               12 & 3 & \sqrt{3} \\
    \end{vmatrix}\\
    &= 2\left(x^2 + y^2\right)\sqrt{3} -24x +24y - 24y + 6\left(x^2 + y^2\right) - 8x\sqrt{3} \\[2mm]
    &= \left(2\sqrt{3} +6\right)\cdot \left(x^2 + y^2\right) - x\cdot\left(24 + 8\sqrt{3}\right)
\end{align*}

Zur Vereinfachung der Lesbarkeit bestimmen wir $\tilde{a} := 2\sqrt{3} +6$ sowie $\tilde{b} := 24 + 8\sqrt{3}$ und formen um nach $y$.

\begin{align*}
    \begin{array}[pos]{rrl}
                        & 0 &= \tilde{a}\cdot \left(x^2 + y^2\right) - x\cdot\tilde{b}\\[2mm]
        \Leftrightarrow & y &= \pm \sqrt{x\cdot\nicefrac{\tilde{b}}{\tilde{a}} - x^2} \\[2mm]
        \Leftrightarrow & y &= \pm \sqrt{4x - x^2} \\
    \end{array}
\end{align*}

$y$ liefert nun in Abhängigkeit von $x$ zwei Lösungen im Definitionsbereich $x \in [0;4]$. 
Auf Grund der Symmetrie bezüglich der $x$-Achse, lässt sich schließen, dass der Mittelpunkt $M$ auf der $x$-Achse und zwar in der Mitte des Definitionsbereichs liegen muss $M=(2|0)$.

Um den analytischen Nachweis allgemein zu erbringen, betrachten wir die zuvor bestimmte Gleichung, welche uns zu jeder $x$-Koordinate einen oberen sowie einen unteren Punkt auf dem Kreis liefert. Durch Subtraktion erhalten wir eine Gleichung $q(x)$ für die Länge des Kreisquerschnitts an der Stelle $x$:
\begin{align*}
    q(x) &= \sqrt{4x - x^2} - \left(-\sqrt{4x - x^2}\right)\\
         &= 2\sqrt{4x - x^2}
\end{align*}

Da die Länge des Kreisquerschnitts genau an der Stelle maximal ist, wo er durch den Mittelpunkt des Kreises geht, bestimmen wir das Maximum von $q(x)$ über die Nullstellen von $q'(x)=(-2x+4)(4x-x^2)^{-\frac{1}{2}}$ zu $x=2$.

An der Stelle $x=2$ hat die Kreisgleichung die Lösungen $y_1=2 \land y_2=-2$. Der Radius beträgt somit $r=2$ und die $y$-Koordinate des Kreismittelpunktes ist also $M_y = r + y_2 = 0$. 

$$
    M = (2|0)
$$

\subsection{Lösung 7b}
Bedingung verlangt, dass die Determinante ungleich Null sein muss, die Zeilenvektoren der Matrix also linear unabhängig sein müssen. Wären Sie dies nicht, wäre der Kreis nicht eindeutig beschrieben. \\

\textcolor{Cerulean}{
Die Bedingung lässt sich auch als Summennorm des Kreuzprodukts $\norm{x \times y}_1 \neq 0$ schreiben.
\begin{align*}
    \begin{vmatrix}
        x_1 & y_1 & 1 \\
        x_2 & y_2 & 1 \\
        x_3 & y_3 & 1 \\
    \end{vmatrix}
    &= x_1y_2 + y_1x_3 + x_2y_3 - y_2x_3 - x_1y_3 - y_1x_2 \\
    &= x_1\cdot (y_2-y_3) + x_2\cdot (y_3-y_1) + x_3\cdot (y_1-y_2)\\
    &= \norm{x \times y}_1
\end{align*}
}


\end{document}
