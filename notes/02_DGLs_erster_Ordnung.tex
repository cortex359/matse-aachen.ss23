\documentclass[main.tex]{subfiles}

\begin{document}

\section{Lösen von gewöhnlichen Differenzialgleichungen}

%Wir haben eine DGL + Anfangswert gegeben.

Wenn die DGL nicht in expliziter Form gegeben ist, also $y'$ nicht freigestellt ist und wir sie auch nicht in explizite Form bringen können, so können wir sie nicht (ohne Weiteres) lösen.

Ist sie in expliziter Form gegeben, also in der Form $y'(x) = f(x,y(x))$, bestimmen wir als Nächstes die Ordnung der DGL.

1. Ordnung siehe \nameref{erste_ordnung}\\
2. Ordnung siehe \nameref{zweite_ordnung}\\
Höherer Ordnung: Können wir nicht lösen.

\subsection{DGLs 1. Ordnung} \label{erste_ordnung}

Ist die DGL separable? Ja $\rightarrow$ \nameref{trennung_der_variablen}, Nein $\rightarrow$ Weiter mit \nameref{substitution_1}

\subsection{Trennung der Variablen} \label{trennung_der_variablen}
Die DGL ist separable, also können wir durch Trennung der Variablen lösen.\\

\textit{Beispiel:}
\[
	y' = x\cdot y^2
\]
1. Fall: $y = 0 \Rightarrow y' = 0$ ist eine triviale Lösung. \\
2. Fall: $y \neq 0$ Trennung der Variablen.
\begin{equiveqs}[crcl]
	& \int \frac{1}{y^2} \dx{y} &=& \int x \dx{x} \\
\equiv & -\frac{1}{y} &=& \frac{x^2}{2} + c \\
\equiv & y &=& -\frac{2}{x^2} + \tilde{c} \\
\end{equiveqs}

\subsection{Substitution I} \label{substitution_1}
Die DGL ist nicht separable. \\

Ist die DGL in der Form 
\[
	y' = h(ax + by + c)\; ?
\]
Dann können wir durch Substitution mit $z = ax + by + c$ lösen.

\textit{Beispiel:}
\[
	y' = (x - y +3)^2,\quad y(1) = 1
\]
Wir substituieren mit $z(x) := x-y+3$ und leiten nach $x$ ab:
\begin{align*}
	z(x) &= x - y(x) + 3 \\
	z'(x) &= 1 - y'(x) \\
		  &= 1-z^2(x)
\end{align*}

Nun können wir den Anfangswert $y(1)=1$ an der Stell $x=1$ in die Funktion einsetzen.
\begin{equiveqs}[rcllr]
	z(1) &=& 1& -y(1) & + 3 \\
		 &=& 1&    -1 & + 3 \\
		 &=& 3& &
\end{equiveqs}

Wenn die neue DGL, die wir durch die Substitution erhalten haben, separable ist, können wir nach \nameref{trennung_der_variablen} weiterverfahren.
Da unser Beispiel $z' = 1 - z^2$ nicht separable ist, müssen wir stattdessen wie folgt lösen und rücksubstituieren.

\begin{align*}
	\int_{3}^{z} \frac{1}{1-s^2} \dx{s} &= \int_{1}^{x} 1 \dx{t}
\end{align*}

Zum Lösen des Integrals führen wir mit $\frac{1}{1-s^2} = \frac{1}{-s^2+1} = -\frac{1}{s^2-1}$ eine Partialbruchzerlegung durch. 
\begin{equiveqs}[cccl]
		& \frac{1}{1-s^2} &=& \frac{A}{s-1} + \frac{B}{s+1} \\[5mm]
		& -\frac{1}{(s+1)(s-1)} &=& \frac{A}{s-1} + \frac{B}{s+1} \\[5mm]
\equiv 	& -1 &=& \frac{A\cdot \cancel{(s-1)}(s+1)}{\cancel{s-1}} + \frac{B\cdot (s-1)\cancel{(s+1)}}{\cancel{s+1}} \\[5mm]
\equiv 	& -1 &=& A\cdot (s+1) + B\cdot (s-1) \\
\end{equiveqs}

Mit $s_1=1$ erhalten wir $A = -\frac{1}{2}$ und mit $s_2=-1$ erhalten wir $B = \frac{1}{2}$. Somit gilt:
\begin{equiveqs}[ccl]
	\frac{1}{1-s^2} &=& -\frac{1}{2} \cdot \frac{1}{s-1} +\frac{1}{2} \cdot \frac{1}{s+1}\\[5mm]
			        &=& \frac{1}{2} \cdot \left( \frac{1}{s+1} - \frac{1}{s-1} \right)
\end{equiveqs}

Damit können wir unser Integral lösen:
\begin{equiveqs}[cccl]
		& \int_{3}^{z} \frac{1}{1-s^2} \dx{s} &=& \int_{1}^{x} 1 \dx{t} \\[4mm]
\equiv	& \frac{1}{2} \int_{3}^{z} \frac{1}{s+1} - \frac{1}{s-1} \dx{s} &=& \int_{1}^{x} 1 \dx{t} \\[4mm]
\equiv	& \frac{1}{2} \Bigl[ \ln(s+1) - \ln(s-1) \Bigr]_{3}^{z} &=& \bigl[ t \bigr]_{1}^{x} \\[4mm]
\equiv	& \frac{1}{2} \left[ \ln\left(\frac{s+1}{s-1}\right) \right]_{3}^{z} &=& x - 1 \\[4mm]
\equiv	& \ln\left(\frac{z+1}{z-1}\right) - \ln\left(2\right)  &=& 2x - 2 \\[4mm]
\end{equiveqs}

Danach stellen wir nach $z$ um:
\begin{equiveqs}[cccl]
		& \ln\left(\frac{z+1}{z-1}\right) - \ln\left(2\right)  &=& 2x - 2 \\[4mm]
\equiv	& \ln\left(\frac{z+1}{z-1}\right)  &=& 2x - 2 + \ln\left(2\right) \\[4mm]
\equiv	& \frac{z+1}{z-1} &=& \e^{2x - 2} \cdot \e^{\ln\left(2\right)} \\[4mm]
\equiv	& \frac{z+1}{z-1} &=& 2\e^{2x - 2} \\[4mm]
\equiv  & z &=& \frac{2\e^{2x - 2} + 1}{\underbrace{2\e^{2x - 2} - 1}_{\neq 0}}\\
\end{equiveqs}

\textcolor{blue!60}{
Umformungstrick:
\[
	a = \frac{b+1}{b-1} \equiv
	b = \frac{a+1}{a-1}\quad \text{mit } a \neq 1
\]
\[
	a = \frac{b-1}{b+1} \equiv
	b = -\frac{a+1}{a-1} = \frac{a+1}{1-a} \quad \text{mit } a \neq 1
\]
}

Rücksubstitution:
\begin{align*}
	y &= x + 3 - z \\
	  &= x + 3 - \frac{2\e^{2x - 2} + 1}{2\e^{2x - 2} - 1}
\end{align*}



\pagebreak

\subsection{Substitution II} \label{substitution_2}
Die DGL ist nicht separable aber in der Form 
\[
	y' = h\left(\frac{y}{x}\right)\; ?
\]

Dann können wir sie durch Substitution mit $z = \frac{y}{x}$ und $y' = z'\cdot x + z$ lösen.\\

\textit{Beispiel:}
\[
	y' = \frac{x^2 + y^2}{x\cdot y},\quad y(1) = 1
\]

Wir formen um, sodass wir mit $z = \frac{y}{x}$ substituieren können.
\begin{align*}
	y' &= \frac{x^2 + y^2}{x\cdot y} \\
	&= \frac{x^{\cancel{2}}}{\cancel{x}\cdot y} + \frac{y^{\cancel{2}}}{x\cdot \cancel{y}}\\[2mm]
	&= \underbrace{\frac{x}{y}}_{=z^{-1}} + \underbrace{\frac{y}{x}}_{=z}\\
\end{align*}

Wir sehen also, dass wir mit der Substitution $z = \frac{y}{x}$ eine Funktion $h(t^{-1} + t)$ vorliegen haben. 
\begin{equiveqs}[rrclr]
		& y' &=& z^{-1} + z \\
\equiv 	& z'\cdot x + \cancel{z} &=& z^{-1} + \cancel{z}\\
\equiv  & z'\cdot x &=& z^{-1}\\
\end{equiveqs}

Diese DGL ist nun wieder separable und kann wieder leicht durch \nameref{trennung_der_variablen} gelöst werden.

Da wir einen Anfangswert mit $z(\textcolor{orange!80}{1})=\frac{y(\textcolor{orange!80}{1})}{\textcolor{orange!80}{1}}=\textcolor{blue!60}{1}$ gegeben haben, können wir auch mit dem bestimmten Integral rechnen und dann nach $z$ umstellen.
\begin{equiveqs}[rrclr]
	& z'\cdot x &=& z^{-1}\\
\equiv & z' &=& \frac{z^{-1}}{x} \\[5mm]
\equiv & \int_{\textcolor{blue!60}{1}}^{z} s \dx{s} &=& \int_{\textcolor{orange!80}{1}}^{x} \frac{1}{t} \dx{t} \\[3mm]
\equiv & \frac{z^2}{2} - \frac{1}{2} &=& \ln(x) - \overbrace{\ln(1)}^{=0} \\[4mm]
\equiv & z &=& \sqrt{2\ln(x) + 1}
\end{equiveqs}

Wobei wegen des Anfangswertes $2\ln(x) + 1 > 0$ sein muss, also gilt diese Lösung nur für $x > \sqrt{\e^{-1}}$.

Die Rücksubstitution liefert sodann die Lösung des Anfangswertproblemes.
\begin{align*}
	y &= x\cdot z \\
	  &= x\cdot \sqrt{2\ln(x) + 1}
\end{align*}

\subsection{Inhomogene DGL} \label{inhomogen}
Handelt es sich um eine inhomogene DGL, so können wir die folgenden Formen unterscheiden: 

Ist die DGL in der Form
\[
	y' = p(x)\cdot y + q(x)\; ?
\]
Dann können wir sie mithilfe des Superpositionsprinzips wie folgt lösen: 
Wir bestimmen die homogene Lösung $y_h$, für die $q(x)=0$ ist, sowie die partikuläre Lösung $y_p$ und setzen diese zusammen:
\[
	y(x) = y_h(x) + y_p(x)
\]

% Mitschrift Fortsetzung vom 06.06.2023

\subsection{Bernoulli} \label{bernoulli}
Wenn nicht der Form $y'(x) = p(x)\cdot y + g(x)$, dann untersuchen wir, ob die DGL in einer speziellen Form vorliegt.

Liegt eine Bernoulli DGL vor?
\begin{align*}
	y'(x) + f(x)\cdot y(x) &= g(x)\cdot y^\alpha \\
	y'(x) + p(x)\cdot y(x) &= q(x)\cdot \textcolor{red}{y(x)^\alpha}
\end{align*}

Wenn ja: 
\begin{align*}
	z(x)  &= y(x)^{1-\alpha} \\
	z'(x) &= \underbrace{(1-\alpha)\cdot p(x)}_{= \tilde{p}}\cdot z(x) + \underbrace{(1-\alpha)\cdot q(x)}_{= \tilde{q}}	
\end{align*}
Wir erhalten die Form wie im Abschnitt \nameref{inhomogen} und lösen entsprechend.\\

\textit{Beispiel:}
\begin{align*}
	(1+x)\cdot y' + y &= - (1+x)^2\cdot y^4, \quad y(0) = 1
\end{align*}

Um zu erkennen, dass es sich um eine Bernoulli DGL handelt, formen wir wie folgt um: 
\begin{align*}
	(1+x)\cdot y' + y &= - (1+x)^2\cdot y^4 \\
\equiv	y' &= \frac{-y}{1+x} - \frac{(1+x)^2}{1+x} \cdot y^4, \quad x\neq -1 \\
\equiv	y' &= \underbrace{\frac{-1}{1+x}}_{p(x)} \cdot y \underbrace{- \frac{(1+x)^2}{1+x}}_{q(x)} \cdot y^4, \quad \alpha = 4
\end{align*}

Substituiere $z = y^{1-\alpha} = y^{-3}$ 
\begin{align*}
	z' &= (1-\alpha)\cdot p(x) \cdot z + (1-\alpha)\cdot q(x)\\
	   &= -3\cdot \left( \frac{-1}{1+x}\right) z + 3 (1+x)\\
	   &= \underbrace{\frac{3}{1+x}\cdot z + 3 (1+x)}_{P(x)}\\
\end{align*}
% $z(0)=y(0)^{-3}=1$

Mit
\begin{align*}
	P(x) &= \int^x_0 p(\tau) \dx{\tau}\\
	&= \int^x_0 \frac{3}{1+\tau}\dx{\tau}\\
	&= 3\ln(1+x) - 0 \\
	&= 3\ln(1+x)
\end{align*}
bestimmen wir:
\begin{align*}
	z_h(x) &= 1\cdot \e^{P(x)} \\
		   &= 1\cdot \e^{\ln(1+x)^3} \\
		   &= (1+x)^3
\end{align*}

Die Bestimmung der homogenen Lösung $z_h$, weicht am Standort Aachen dadurch ab, dass wir $f(x) = -p(x)$ setzen. Ansonsten sind die Verfahren gleich. 
\begin{align*}
	z_h'(x) + f(x)\cdot z(x) &= 0 \\
	z_h'(x) - \frac{3}{1+x}\cdot z(x) &= 0 \\
	z_h'(x) &= \frac{3}{1+x} z(x) \\
	\int \frac{1}{z}\dx{z} &= \int \frac{3}{1+x} \dx{x} \\
	\ln(z_h) &= \int \frac{3}{1+x} \dx{x}\\
	z_h(x) &= \e^{\int \frac{3}{1+x}\dx{x}} \\
	  &= \e^{3\ln(1+x)+\tilde{c}}\\
	  &= c\cdot \e^{\ln((1+x)^3)}\\
	  &= c\cdot (1+x)^3 \\
	z_h(0) &= c\cdot 1 \equiv c = 1 \\
	z_h(x) &= (1+x)^3
\end{align*}

Zur Bestimung der partikulären Lösung. Dies lässt sich auch mit Ansatz vom Typ der rechten Seite machen. Wir wählen jedoch Variation der Konstanten: 
\begin{align*}
	z_p(x) &= \e^{P(x)} \cdot c(x)\\
		   &= \e^{P(x)} \cdot \int_0^x q(\tau)\e^{-P(\tau)} \dx{\tau}\\
		   &= (1+x)^3 \int_0^x 3(1+\tau) \e^{-\ln((1+\tau)^3)} \dx{\tau} \\
		   &= (1+x)^3 \int_0^x 3(1+\tau) \e^{\ln\frac{1}{(1+\tau)^3}} \dx{\tau} \\
		   &= (1+x)^3 \int_0^x 3(1+\tau) \frac{1}{(1+\tau)^3} \dx{\tau} \\
		   &= (1+x)^3 \int_0^x \frac{3}{(1+\tau)^2} \dx{\tau} \\
		   &= (1+x)^3 \left[ - \frac{1}{1+\tau} \right]_0^x \\
		   &= -3(1+x)^2 + 3(1+x)^3 \\
\end{align*}

\begin{align*}
	z(x) &= z_h(x) + z_p(x) \\
		 &= (1+x)^3 -3(1+x)^2 + 3(1+x)^3 \\
		 &= 4(1+x)^3 -3(1+x)^2 \\
\end{align*}

Rücksubstitution:
\begin{align*}
	y(x) &= \frac{1}{\sqrt[3]{z(x)}} \\
		 &= \frac{1}{\sqrt[3]{4(1+x)^3 -3(1+x)^2}}
\end{align*}

\subsection{Exakte DGLs I} \label{exakte1}
Ist die DGL in der Form
$$
	P(x,y) + Q(x,y)\cdot y' = 0
$$
Wenn ja, ist die Integrabilitätsbedingung erfüllt?
\begin{align*}
	\frac{\partial P}{\partial y} (x,y) \overset{?}{=} \frac{\partial Q}{\partial x}(x,y)
\end{align*}

\textit{Beispiel:}\\
\begin{align*}
	x^2 -y = (x + \sin^2(y))\cdot y'
\end{align*}
Wir stellen um, damit sich diese Form ergibt: 
\[\everymath={\displaystyle}\begin{array}{crl}
				& x^2 -y &= (x + \sin^2(y)) \cdot y' \\
\equiv & \underbrace{(x^2 -y)}_{P(x,y)} + \underbrace{(-(x+\sin^2(y)))}_{Q(x,y)} \cdot y' & = 0 \\
\end{array}\]
Integrabilitätsbedingung erfüllt?
\begin{align*}
	\frac{\partial P}{\partial y}(x,y) &= -1 \\
	\frac{\partial Q}{\partial x}(x,y) &= -1 
\end{align*}
$\Rightarrow$ Ja, Integrabilitätsbedingung erfüllt.

\begin{align*}
	F(x,y) &= \int P(x,y) \dx{x} = \int x^2 -y \dx{x} \\
		&= \frac{x^3}{3} \textcolor{blue}{-y x} + \textcolor{magenta}{c_1(y)} \\
	F(x,y) &= \int Q(x,y) \dx{y} = \int -x - \sin^2(y) \dx{y} \\
		&= \textcolor{blue}{-y x} \textcolor{magenta}{-\frac{1}{2}y + \frac{1}{y} \sin(2 y)} + c_2(x)\\
\end{align*}
Koeffizientenvergleich gibt uns die Potenzialfunktion $F(x,y)$
\begin{align*}
	F(x,y) = \frac{x^3}{3} -xy -\frac{1}{2}y + \frac{1}{y} \sin(2 y) = c
\end{align*}

In Aachen setzen wir üblicherweise $\frac{\partial F(x,y)}{\partial y} \overset{!}{=} Q(x,y)$ gleich und lösen dann die Gleichung nach $c_1 \cdot y'$ auf.

\subsection{Exakte DGLs II} \label{exakte2}
Oder in der From? 
$$
	P(x,y) + Q(x,y)\cdot y' = f(x)
$$

\textit{Beispiel:}
\begin{align*}
	\underbrace{y}_{P(x,y)} \underbrace{- (2x +y)}_{Q(x,y)} \cdot y' = 0 \\
\end{align*}

Integrabilitätsbedingung:
\begin{align*}
	\frac{\partial P}{\partial y}(x,y) &= 1 \\
	\frac{\partial Q}{\partial x}(x,y) &= -2 
\end{align*}
Nein, ist nicht erfüllt. $\rightarrow$ Finde den integrierenden Faktor:
\begin{align*}
	\frac{\partial P}{\partial y} - \frac{\partial Q}{\partial x} = 3
\end{align*}

Nun müssen wir überprüfen, ob 
\[
	\frac{\frac{\partial P}{\partial y} - \frac{\partial Q}{\partial x}}{P} 
\]
von $y$ abhängig ist, oder ob 
\[
	\frac{\frac{\partial P}{\partial y} - \frac{\partial Q}{\partial y}}{P} 
\]


Versuche: 
\begin{align*}
	\frac{\frac{\partial P}{\partial y} - \frac{\partial Q}{\partial y}}{P} 
	&= \frac{3}{y} =: \mu(y)
\end{align*}

Suche nach einer Funktion in Abhängigkeit von $y$. 
\begin{align*}
	\mu(y) = \e^{-\int g(y)\dx{y}} = \e^{-3\ln y} = \frac{1}{y^3} \\
	y-(2x +y) \cdot y' = 0 \\
	\Rightarrow \underbrace{\frac{1}{y^2}}_{P(x,y)} \underbrace{ -\left(\frac{2x}{y^3} + \frac{1}{y^2} \right)}_{Q(x,y)} = 0
\end{align*}

Neue Integrabilitätsbedingung:
\begin{align*}
	\frac{\partial P}{\partial y}(x,y) &= - \frac{2}{y^3} \\
	\frac{\partial Q}{\partial x}(x,y) &= - \frac{2}{y^3} 
\end{align*}
IB ist nun erfüllt. 

% TODO farbig markieren
\begin{align*}
	F(x,y) = \int \frac{1}{y^2} \dx{x} = \frac{x}{y^2} + c_1(y) + 0\\
	F(x,y) = \int - \frac{2}{y^3} - \frac{1}{y^2} \dx{y} = \frac{x}{y^2} + \frac{1}{y} + c_2(x)
\end{align*}

Koeffizientenvergleich liefert

\begin{align*}
	F(x,y) = \frac{x}{y^2} + \frac{1}{y} = c
\end{align*}

Können wir nach $y$ freistellen? Nein, also fertig. 

\end{document}