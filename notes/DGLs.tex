\documentclass[main.tex]{subfiles}

\begin{document}

Wir haben eine DGL + Anfangswert gegeben. \\

Ist die DGL nicht in expliziter From gegeben, also ist $y'$ nicht freigestellt, so können wir sie nicht ohne Weiteres lösen.

Ist sie in expliziter Form gegeben, also in der Form $y'(x) = f(x,y(x))$ bestimmen wir als nächstes die Ordnung der DGL.

1. Ordnung siehe [1.A]
2. Ordnung siehe [2.A]
Höherer Ordnung: Können wir nicht lösen.

\section{DGLs 1. Ordnung}

\subsection*{[1.A]}
Ist die DGL separabel? Ja -> [1.B], Nein -> [1.C]

\subsection*{[1.B]}
Die DGL ist separabel, also können wir durch Trennung der Variablen lösen.\\

\textit{Beispiel:}
\[
	y' = x\cdot y^2
\]
1. Fall: $y(x) = 0$ ist eine triviale Lösung. \\
2. Fall: $y(x) \neq 0$ Trennung der Variablen.
\begin{equiveqs}[crcl]
	& \int \frac{1}{y^2} \dx{y} &=& \int x \dx{x} \\
\equiv & -\frac{1}{y} &=& \frac{x^2}{2} + c \\
\equiv & y &=& -\frac{2}{x^2} + \tilde{c} \\
\end{equiveqs}

\subsection*{[1.C]}
Die DGL ist nicht separabel. \\

Ist die DGL in der Form 
\[
	y' = h(ax + by + c)\; ?
\]
Dann können wir durch Substitution lösen.

Ist die DGL in der Form
\[
	y' = h\left(\frac{y}{x}\right)
\]


\section{Mitschrift Fortsetzung vom 06.06.2023}

\subsection{[D]}
Wenn nicht der Form $y'(x) = p(x)\cdot y + g(x)$, dann untersuchen wir, ob die DGL in einer speziellen Form vorliegt.

Liegt eine Bernoulli DGL vor?
\begin{align*}
	y'(x) + f(x)\cdot y(x) &= g(x)\cdot y^\alpha \\
	y'(x) + p(x)\cdot y(x) &= q(x)\cdot \textcolor{red}{y(x)^\alpha}
\end{align*}

Wenn ja: 
\begin{align*}
	z(x)  &= y(x)^{1-\alpha} \\
	z'(x) &= \underbrace{(1-\alpha)\cdot p(x)}_{= \tilde{p}}\cdot z(x) + \underbrace{(1-\alpha)\cdot q(x)}_{= \tilde{q}}	
\end{align*}
Wir erhalten die Form wie [C.1] und lösen entsprechend.\\

\textit{Beispiel:}
\begin{align*}
	(1+x)\cdot y' + y &= - (1+x)^2\cdot y^4, \quad y(0) = 1
\end{align*}

Um zu erkennen, dass es sich um eine Bernoulli DGL handelt, formen wir wir folgt um: 
\begin{align*}
	(1+x)\cdot y' + y &= - (1+x)^2\cdot y^4 \\
\Leftrightarrow	y' &= \frac{-y}{1+x} - \frac{(1+x)^2}{1+x} \cdot y^4, \quad x\neq -1 \\
\Leftrightarrow	y' &= \underbrace{\frac{-1}{1+x}}_{p(x)} \cdot y \underbrace{- \frac{(1+x)^2}{1+x}}_{q(x)} \cdot y^4, \quad \alpha = 4
\end{align*}

Substituiere: $z(x) = y(x)^{1-\alpha} = y(x)^{-3}$
\begin{align*}
	z'(x) &= (1-\alpha)\cdot p(x) \cdot z(x) + (1-\alpha)\cdot q(x)\\
		  &= -3(\frac{-1}{1+x} z(x) +3 (1+x)), \quad z(0)=y(0)^{-3}=1
\end{align*}

Mit
\begin{align*}
	P(x) &= \int^x_0 p(\tau) \dx{\tau}\\
	&= \int^x_0 \frac{3}{1+\tau}\dx{\tau}\\
	&= 3\ln(1+x) - 0 \\
	&= 3\ln(1+x)
\end{align*}
bestimmen wir:
\begin{align*}
	z_h(x) &= 1\cdot e^{P(x)} \\
		   &= 1\cdot \ct{e}^{\ln(1+x)^3} \\
		   &= (1+x)^3
\end{align*}

Die Bestimmung der homogenen Lösung $z_h$, weicht am Standort Aachen dadurch ab, dass wir $f(x) = -p(x)$ setzen. Ansonsten sind die Verfahren gleich. 
\begin{align*}
	z_h'(x) + f(x)\cdot z(x) &= 0 \\
	z_h'(x) - \frac{3}{1+x}\cdot z(x) &= 0 \\
	z_h'(x) &= \frac{3}{1+x} z(x) \\
	\int \frac{1}{z}\dx{z} &= \int \frac{3}{1+x} \dx{x} \\
	\ln(z_h) &= \int \frac{3}{1+x} \dx{x}\\
	z_h(x) &= \ct{e}^{\int \frac{3}{1+x}\dx{x}} \\
	  &= \ct{e}^{3\ln(1+x)+\tilde{c}}\\
	  &= c\cdot \ct{e}^{\ln((1+x)^3)}\\
	  &= c\cdot (1+x)^3 \\
	z_h(0) &= c\cdot 1 \Leftrightarrow c = 1 \\
	z_h(x) &= (1+x)^3
\end{align*}

Zur Bestimung der partikulären Lösung. Dies lässt sich auch mit Ansatz vom Typ der rechten Seite machen. Wir wählen jedoch Variation der Konstanten: 
\begin{align*}
	z_p(x) &= \ct{e}^{P(x)} \cdot c(x)\\
		   &= \ct{e}^{P(x)} \cdot \int_0^x q(\tau)\ct{e}^{-P(\tau)} \dx{\tau}\\
		   &= (1+x)^3 \int_0^x 3(1+\tau) \ct{e}^{-\ln((1+\tau)^3)} \dx{\tau} \\
		   &= (1+x)^3 \int_0^x 3(1+\tau) \ct{e}^{\ln\frac{1}{(1+\tau)^3}} \dx{\tau} \\
		   &= (1+x)^3 \int_0^x 3(1+\tau) \frac{1}{(1+\tau)^3} \dx{\tau} \\
		   &= (1+x)^3 \int_0^x \frac{3}{(1+\tau)^2} \dx{\tau} \\
		   &= (1+x)^3 \left[ - \frac{1}{1+\tau} \right]_0^x \\
		   &= -3(1+x)^2 + 3(1+x)^3 \\
\end{align*}

\begin{align*}
	z(x) &= z_h(x) + z_p(x) \\
		 &= (1+x)^3 -3(1+x)^2 + 3(1+x)^3 \\
		 &= 4(1+x)^3 -3(1+x)^2 \\
\end{align*}

Rücksubstitution:
\begin{align*}
	y(x) &= \frac{1}{\sqrt[3]{z(x)}} \\
		 &= \frac{1}{\sqrt[3]{4(1+x)^3 -3(1+x)^2}}
\end{align*}

\subsection{[E]}
Ist die DGL in der Form?
$$
	P(x,y) + Q(x,y)\cdot y' = 0
$$
Wenn ja, ist die Integrabilitätsbedingung erfüllt?
\begin{align*}
	\frac{\partial P}{\partial y} (x,y) \overset{?}{=} \frac{\partial Q}{\partial x}(x,y)
\end{align*}

\textit{Beispiel:}\\
\begin{align*}
	x^2 -y = (x + \sin^2(y))\cdot y'
\end{align*}
Wir stellen um, damit sich diese Form ergibt: 
\[\everymath={\displaystyle}\begin{array}{crl}
				& x^2 -y &= (x + \sin^2(y)) \cdot y' \\
\Leftrightarrow & \underbrace{(x^2 -y)}_{P(x,y)} + \underbrace{(-(x+\sin^2(y)))}_{Q(x,y)} \cdot y' & = 0 \\
\end{array}\]
Integrabilitätsbedingung erfüllt?
\begin{align*}
	\frac{\partial P}{\partial y}(x,y) &= -1 \\
	\frac{\partial Q}{\partial x}(x,y) &= -1 
\end{align*}
$\Rightarrow$ Ja, Integrabilitätsbedingung erfüllt.

\begin{align*}
	F(x,y) &= \int P(x,y) \dx{x} = \int x^2 -y \dx{x} \\
		&= \frac{x^3}{3} \textcolor{blue}{-y x} + \textcolor{magenta}{c_1(y)} \\
	F(x,y) &= \int Q(x,y) \dx{y} = \int -x - \sin^2(y) \dx{y} \\
		&= \textcolor{blue}{-y x} \textcolor{magenta}{-\frac{1}{2}y + \frac{1}{y} \sin(2 y)} + c_2(x)\\
\end{align*}
Koeffizientenvergleich gibt uns die Potentialfunktion $F(x,y)$
\begin{align*}
	F(x,y) = \frac{x^3}{3} -xy -\frac{1}{2}y + \frac{1}{y} \sin(2 y) = c
\end{align*}

In Aachen setzen wir üblicherweise $\frac{\partial F(x,y)}{\partial y} \overset{!}{=} Q(x,y)$ gleich und lösen dann die Gleichung nach $c_1 * y'$ auf.

\subsection{[F]}
\vspace{6cm}
Oder in der From? 
$$
	P(x,y) + Q(x,y)\cdot y' = f(x)
$$

\textit{Beispiel:}
\begin{align*}
	\underbrace{y}_{P(x,y)} \underbrace{- (2x +y)}_{Q(x,y)} \cdot y' = 0 \\
\end{align*}

Integrabilitätsbedingung:
\begin{align*}
	\frac{\partial P}{\partial y}(x,y) &= 1 \\
	\frac{\partial Q}{\partial x}(x,y) &= -2 
\end{align*}
Nein, ist nicht erfüllt. 
-> Finde den integrierenden Faktor:
\begin{align*}
	\frac{\partial P}{\partial y} - \frac{\partial Q}{\partial y} = 3
\end{align*}
Versuche: 
\begin{align*}
	\frac{\frac{\partial P}{\partial y} - \frac{\partial Q}{\partial y}}{P} 
	&= \frac{3}{y} =: g(y)
\end{align*}

Suche nach einer Funktion in Abhängigkeit von $y$. 
\begin{align*}
	\mu(y) = \ct{e}^{-\int g(y)\dx{y}} = \ct{e}^{-3\ln y} = \frac{1}{y^3} \\
	y-(2x +y) \cdot y' = 0 \\
	\Rightarrow \underbrace{\frac{1}{y^2}}_{P(x,y)} \underbrace{ -\left(\frac{2x}{y^3} + \frac{1}{y^2} \right)}_{Q(x,y)} = 0
\end{align*}

Neue Integrabilitätsbedingung:
\begin{align*}
	\frac{\partial P}{\partial y}(x,y) &= - \frac{2}{y^3} \\
	\frac{\partial Q}{\partial x}(x,y) &= - \frac{2}{y^3} 
\end{align*}
IB ist nun erfüllt. 

% TODO farbig markieren
\begin{align*}
	F(x,y) = \int \frac{1}{y^2} \dx{x} = \frac{x}{y^2} + c_1(y) + 0\\
	F(x,y) = \int - \frac{2}{y^3} - \frac{1}{y^2} \dx{y} = \frac{x}{y^2} + \frac{1}{y} + c_2(x)
\end{align*}

Koeffizientenvergleich liefert

\begin{align*}
	F(x,y) = \frac{x}{y^2} + \frac{1}{y} = c
\end{align*}

Können wir nach $y$ freistellen? Nein, also fertig. 

\section{DGLs 2. Ordnung}

\subsection*{[2.A]}

Homogene DGL 2. Ordnung. (für inhomogene DGLs 2. Ordnung ist auch eine Lösung mit Variation der Kontanten möglich).
\begin{align*}
	y'' + a y' + by = 0 \qquad y(\zeta) = \eta_1,\ y'(\zeta) = \eta_2
\end{align*}

Ansatz: $y(x) = \ct{e}^{\lambda x}$. 
\begin{align*}
	\lambda^2 \ct{e}^{\lambda x} + a\lambda \ct{e}^{\lambda x} + b\ct{e}^{\lambda x} = 0 \\
	\Leftrightarrow \ct{e}^{\lambda x} \cdot \underbrace{\left( \textcolor{blue}{\lambda^2 +a\lambda + b} \right)}_{= 0} = 0
\end{align*}

\textcolor{blue}{charakteristische Gleichung}

Entscheidung anhand der Diskriminanten der Charakteristischen Gleichgung.

\begin{align}
	D &= \frac{a^2}{4} -b \\
\end{align}

Wenn $D > 0$ haben wir zwei reelle Nullstellen, $\lambda_1, \lambda_2$.
$$
	y(x) = c_1 \ct{e}^{\lambda_1 x} + c_2 \ct{e}^{\lambda_2 x} 
$$

Wenn $D = 0$ haben wir eine reelle (doppelte) Nullstelle $\lambda$.
$$
	y(x) = c_1 \ct{e}^{\lambda x} + c_2 \cdot x \cdot \ct{e}^{\lambda x} 
$$

Wenn $D < 0$ liegen zwei komplexe Nullstellen vor. $\lambda_1 = w + \ct{i}v,\ \lambda_2 = w - \ct{i}v$
$$
	y(x) = \ct{e}^{w x} \cdot \left( c_1 \cos(v x) + c_2 \sin(v x) \right) 
$$

\end{document}