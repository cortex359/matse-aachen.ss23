\documentclass[main.tex]{subfiles}

\begin{document}

\renewcommand*{\phi}{\varphi}

\section{Extremwertprobleme mit Nebenbedingung}
\subsection{Lagrange-Multiplikator}

\section{Extremstellen von Funktionen mit mehreren Veränderlichen}

\section{Stetigkeit}

\section{Taylorpolynom}

\section{Vektorfelder}
% grad, rot, div 

\section{Tangentialebenen, Richtungsableitungen}
\subsubsection{Gradient}
\[
    \grad \left(f(x, y)\right) = \nabla f(x, y) = \vector{ f_x(x, y) \\ f_y(x, y)}
\]

\subsubsection{Tangentialebene}
\newcommand{\fxxnyn}[2]{
    \textcolor{#1}{
        f_{#2}(x_0, y_0)
    }
}
\[
    T(x, y) = \fxxnyn{blue!40}{}
    + \fxxnyn{red!70}{x}\cdot (x-x_0) 
    + \fxxnyn{orange!70}{y}\cdot (y-y_0)
\]
\[
    T(x, y) = \vector{x_0 \\ y_0 \\ \fxxnyn{blue!40}{}} + \lambda \vector{1 \\ 0 \\ \fxxnyn{red!70}{x}} + \mu \vector{0 \\ 1 \\ \fxxnyn{orange!70}{y}}
\]

\subsubsection{Richtungsableitung}
\[
    D_{\vec{v}}\left(f\left(x_0, y_0\right)\right) = \frac{\partial f}{\partial \vec{v}} = \scalarprod*{\nabla f, \vec{v}} \quad \text{mit} \quad \norm*{\vec{v}} = 1
\]
\subsubsection{Normieren}
\[
    \vec{v} = \frac{\vec{a}}{\norm*{\vec{a}}}
\]
\subsubsection{Richtung des steilsten Anstiegs}
\[
    \vec{v}_{\text{max}} = \frac{\nabla f}{\norm*{\nabla f}}
\]
\subsubsection{Wert der maximalen Steigung}
\[
    D_{\vec{v}_{\text{max}}}\left(f\left(x_0, y_0\right)\right) 
    = \scalarprod*{\nabla f, \vec{v}_{\text{max}}}
    %= \scalarprod*{\nabla f, \frac{\nabla f}{\norm*{\nabla f}}} 
    %= \norm*{\nabla f}^{-1} \scalarprod*{\nabla f, \nabla f} 
    = \norm*{\nabla f\left(x_0, y_0\right)}
\]


\section{Kurvenintegral, Potentialfunktion}

\section{Mehrdimensionale Integration}
\subsubsection{Polarkoordinaten}
\[
    \iint f(\textcolor{blue!70}{x}, \textcolor{OrangeRed}{y}) \dx{(x, y)} = \iint f(
        \textcolor{blue!70}{r\cos\phi}, 
        \textcolor{OrangeRed}{r\sin\phi}
    ) \cdot r \dx{(\phi, r)}
\]
Kreisfläche: $A = \pi \cdot r^2$

\subsubsection{Kugelkoordinaten}
\[
    \iiint f(\textcolor{blue!70}{x}, \textcolor{OrangeRed}{y}, \textcolor{OliveGreen}{z}) \dx{(x, y, z)} = \iiint f\left(
        \textcolor{blue!70}{r\cos \phi \sin\theta},
        \textcolor{OrangeRed}{r\sin \phi \sin\theta}, 
        \textcolor{OliveGreen}{r\cos\theta} \right) 
        \abs{r^2 \sin\theta}
        \dx{(\theta, \phi, r)}
\]

Kugelvolumen: $V = \frac{4}{3} \pi\cdot r^3$

\subsection{Schwerpunkte}

\section{Implizite Funktionen}

\section{Differenzialgleichungen und Anfangswertprobleme}

\end{document}
