\documentclass[main.tex]{subfiles}

\begin{document}

\section{Aufgabe 1}

Sei $\Sigma = \{o, p, t, r\}$. Bestimmen Sie die Mächtigkeit der folgenden Sprachen:

\begin{enumerate}
    \item $L_1 = \left\{ w\in \Sigma^*\ \middle|\ w \text{ enthält } otto \text{ als Teilwort und } |w| = 7 \right\}$
    \item $L_2 = \left\{ w\in \Sigma^*\ \middle|\ w \text{ enthält } otto \text{ als Teilwort und } |w|\leq 7 \right\}$
    \item $L_3 = \left\{ w\in \Sigma^*\ \middle|\ |w|=5 \lor |w|=6 \right\}$
    \item $L_4 = \left\{ w\in \Sigma^*\ \middle|\ |w|_o = 2 \land |w|_t = 2 \land |w| = 5 \right\}$
    \item $L_5 = \left\{ w\in \Sigma^*\ \middle|\ |w| = 7 \land \left( w\in L_2 \lor |w|_r = 2\right) \right\}$
\end{enumerate}


\subsection{Lösung 1a}
Es gibt vier Positionen, an denen das Wort $otto$ stehen kann, sowie jeweils drei weitere Stellen, an denen ein beliebiger Buchstabe stehen kann.
Außerdem gibt es noch eine Dopplung im Fall $ottotto$, welche nicht doppelt gezählt werden darf.
$$
    |L_1| = 4\cdot 4^3 - 1 = 255
$$

\subsection{Lösung 1b}
Für $|w|\leq 4$ gibt es genau ein Wort, nämlich das Teilwort selbst.
Für $|w|=5$ gibt es zwei mögliche Positionen für das Teilwort und jeweils eine weitere Stelle $2\cdot 4^1$.
Für $|w|=6$ gibt es drei mögliche Positionen und jeweils zwei weitere Stellen $3\cdot 4^2$ und für $|w|=7$ entsprechend $|L_1|$ viele Möglichkeiten.
$$
    |L_2| = 1 + 2\cdot 4^1 + 3\cdot 4^2 + 4\cdot 4^3 - 1 
          = 1 + 8 + 48 + 255 = 312
$$

\subsection{Lösung 1c}
$$
    |L_3| = 4^5 + 4^6 = 5120
$$

\subsection{Lösung 1d}
Für den ersten Buchstaben gibt es 2 von 5 möglichen Positionen, für den zweiten Buchstaben gibt es 2 von 3 verbliebenen Positionen.
Der letzte Buchstabe kann entweder $p$ oder $r$ sein.
\begin{align*}
    |L_4| &= \vektor{5\\ 2} \cdot \vektor{3\\ 2} \cdot 2^1 \\[1mm]
        &= \frac{5!}{(5-2)!\cdot 2!} \cdot \frac{3!}{(3-2)!\cdot 2!} \cdot 2^1 \\[2mm]
        &= 10 \cdot 3 \cdot 2 \\[2mm]
        &= 60
\end{align*}

\subsection{Lösung 1e}

Es gilt $L_1 = \left\{w \in L_2 \middle| |w|=7\right\}$, jedoch muss beachtet werden, dass die Anzahl der Konfigurationen in $L_1$,
in denen bereits zweimal der Buchstabe $r$ auftritt $\left\{ w\in \Sigma^*\ \middle|\ w\in L_1 \land |w|_r = 2 \right\}$, also 
$$
    \vektor{4\\1}\cdot \vektor{3\\2}\cdot 3^1 = 4\cdot 3\cdot 3 = 36,
$$
nicht doppelt gezählt werden.

Betrachtet man die Ziehung ohne Zurücklegen von 2 aus 7 Positionen, multipliziert mit den jeweils 3 möglichen Belegungen der übrigen 5 Positionen, so ergeben sich
insgesamt
$$
    \vektor{7\\ 2} \cdot 3^5 = \frac{7!}{(7-2)!\cdot 2!} \cdot 3^5 = 21 \cdot 243 = 5103
$$
Möglichkeiten den Buchstaben $r$ in einem Wort der Länge 7 unterzubringen.\\

Für die Mächtigkeit von $L_5$ ergibt sich sodann
\begin{align*}
    |L_5| &= |L_1| - 36 + 5103 \\[1mm]
          &= 5322.
\end{align*}

\end{document}
