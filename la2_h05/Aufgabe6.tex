\documentclass[main.tex]{subfiles}

\begin{document}

\section{Aufgabe 6}
Ein Architekt plant, auf dem Dach eines Hauses eine Antenne anzubringen (siehe Skizze). 
Von seinem Bezugspunkt $A$ aus gesehen soll sie senkrecht über der Stelle, die auf der Grundfläche des Hauses 5m nach rechts ($x$-Richtung) und 2m nach
hinten ($y$-Richtung) liegt, auf dem Dach angebracht werden.

\begin{enumerate}
    \item Berechnen Sie den Anfangspunkt der Antenne auf dem Dach vom Bezugspunkt $A$ aus gesehen.
    \item  Der Dachdecker, der an dieser Stelle Dachziegel weglassen muss, nimmt als Bezugssystem die rechte untere Ecke des Daches $B$ und als Basisvektoren die eingezeichneten Richtungsvektoren $x'$ , $y'$ und $z'$ (der Länge 1). 
    Berechnen Sie die Koordinaten des Punktes $C$ bzgl. seines Koordinatensystems.
    \item Wie lauten die Koordinaten des Bezugspunktes A des Architekten im Koordinatensystem des Dachdeckers?
    \item Wie muss die Transformation (Matrix und Verschiebungsvektor) aussehen, die einen beliebigen Punkt des Hauses aus dem Koordinatensystem des Dachdeckers in das des Architekten umrechnet? 
    Überprüfen Sie ihr Ergebnis, indem Sie das Ergebnis von (b) in das Ergebnis von (a) umrechnen.
    \item Wie muss die Transformation (Matrix und Verschiebungsvektor) aussehen, die einen beliebigen Punkt des Hauses aus dem Koordinatensystem des Architekten in das des Dachdeckers umrechnet?
\end{enumerate}

\subsection{Lösung 6}

\end{document}
