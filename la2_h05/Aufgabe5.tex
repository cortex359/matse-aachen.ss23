\documentclass[main.tex]{subfiles}

\begin{document}

\section{Aufgabe 5}
Es sind folgende Abbildungsmatrizen gegeben:

\begin{align*}
    A_\phi &= \begin{pmatrix}
        \cos (\phi) & -\sin (\phi) \\
        \sin (\phi) & \cos (\phi)
    \end{pmatrix}
    &\text{und} & &
    B &= \begin{pmatrix}
        1 &  0 \\
        0 & -1
    \end{pmatrix}
\end{align*}

Durch Matrix $A_\phi$ wird ein Vektor im $\mathbb{R}^2$ um den Winkel $\phi$ gedreht, die Matrix $B$ spiegelt selbigen an der $x$-Achse.

\begin{enumerate}
    \item Veranschaulichen Sie die Behauptungen am Beispiel des Vektors $x_0 = \begin{pmatrix}2 \\ 1 \end{pmatrix}$ und $\phi = \frac{\pi}{2}$, wobei $A=A_{\left(\frac{\pi}{2}\right)}$, in dem Sie den Vektor selber und dessen Abbildungen $f_A(x_0) = A\cdot x_0$ und $f_B(x_0) = B\cdot x_0$ in ein Koordinatensystem einzeichnen.
    \item Zeichnen Sie auch die hintereinander geschalteten Abbildungen $f_{AB}(x_0) = A\cdot B\cdot x_0$ und $f_{BA}(x_0) = B\cdot A\cdot x_0$ von $x_0$.
    \item Wie sehen die Umkehrabbildungen zu $f_A(X)$ und $f_B(x)$ aus? Stellen Sie dazu die Abbildungsmatrizen $A^{-1}$ und $B^{-1}$ auf.
    \item Bestimmen Sie die zugehörigen Abbildungsmatrizen zu den Umkehrabbildungen $f^{-1}_{AB}$ und $f^{-1}_{BA}$.
    \item Verifizieren Sie die Ergebnisse aus b) und d), indem Sie die Vektoren $f_{AB}(x_0)$ und $f_{BA}(x_0)$, die Sie zeichnerisch bei b) erhalten haben, mit den Matrizen aus d) multiplizieren.
\end{enumerate}

\subsection{Lösung 5a}
\begin{center}
\begin{tikzpicture}[scale=1]
    \begin{axis}[
            width=\textwidth,
            unit vector ratio*=1 1 1,
            axis lines = middle,
            ymin=-1,
            ymax=2,
            xmin=-2,
            xmax=3,
            xlabel = $x$,
            ylabel = $y$,
            xtick distance=1,
            ytick distance=1,
            disabledatascaling,
            legend cell align={left},
        ]

        \draw[black,->] (axis cs:0,0) -- (axis cs:2,1) node [midway, above left] {$x_0$};
        \addlegendimage{black,->}
        \addlegendentry{$x_0 = (2, 1)^T$}

        \draw[td-orange,->] (axis cs:0,0) -- (axis cs:-1,2) node [midway, below left] {$f_A(x_0)$};
        \addlegendimage{td-orange,->}
        \addlegendentry{$f_A(x_0) = (-1,2)^T$}

        \draw[blue,->] (axis cs:0,0) -- (axis cs:2,-1) node [midway, below left] {$f_B(x_0)$};;
        \addlegendimage{blue,->}
        \addlegendentry{$f_B(x_0) = (2,-1)^T$}

        \draw[thin, blue, dotted] (axis cs:2,1) -- (axis cs:2,-1);

        \draw [thin, td-orange, dotted, ->] (axis cs:0.5,0.25) arc [radius=0.55,start angle=26.56,end angle=116.56];
    \end{axis}
\end{tikzpicture}
\end{center}

\subsection{Lösung 5b}
\begin{center}
\begin{tikzpicture}[scale=1]
    \begin{axis}[
            width=0.8\textwidth,
            unit vector ratio*=1 1 1,
            axis lines = middle,
            ymin=-2,
            ymax=2,
            xmin=-2,
            xmax=2,
            xlabel = $x$,
            ylabel = $y$,
            xtick distance=1,
            ytick distance=1,
            disabledatascaling,
            legend cell align={left},
            legend style={at={(1.2,0)},anchor=south east}
        ]

        \draw[black,->] (axis cs:0,0) -- (axis cs:2,1) node [midway, above left] {$x_0$};
        \addlegendimage{black,->}
        \addlegendentry{$x_0 = (2, 1)^T$}

        \draw[td-orange,->] (axis cs:0,0) -- (axis cs:-1,-2) node [midway, above left] {$f_{BA}(x_0)$};
        \addlegendimage{td-orange,->}
        \addlegendentry{$f_{BA}(x_0) = (-1,-2)^T$}

        \draw[thin, td-orange, dotted, ->] (axis cs:0,0) -- (axis cs:-1,2);

        \draw[thin, td-orange, dotted] (axis cs:-1,2) -- (axis cs:-1,-2);


        \draw[blue,->] (axis cs:0,0) -- (axis cs:1,2) node [midway, below right] {$f_{AB}(x_0)$};;
        \addlegendimage{blue,->}
        \addlegendentry{$f_{AB}(x_0) = (1,2)^T$}

        \draw[thin, blue, dotted, ->] (axis cs:0,0) -- (axis cs:2,-1);
    \end{axis}
\end{tikzpicture}
\end{center}

\subsection{Lösung 5c}
Bestimmung der Inversen von $A$ für $\cos (\phi) \neq 0$ und $\sin (\phi) \neq 0$, also $\phi \neq \pi \cdot n \wedge  \phi \neq \pi \cdot n - \frac{\pi}{2}\ \forall\ n \in \mathbb{Z}.$
\begin{align*}
    & \eqmatrix{cc|cc}{
        \cos (\phi) & -\sin (\phi) &  1 & 0 \\
        \sin (\phi) &  \cos (\phi) &  0 & 1 \\
    }\\
    \leadsto & \eqmatrix{cc|cc}{
        1 & -\frac{\sin (\phi)}{\cos (\phi)} &  \cos^{-1} (\phi) & 0 \\
        1 &  \frac{\cos (\phi)}{\sin (\phi)} &  0 & \sin^{-1} (\phi) \\
    }\\
    \leadsto & \eqmatrix{cc|cc}{
        1 & -\frac{\sin (\phi)}{\cos (\phi)}                                 &  \cos^{-1} (\phi)   & 0 \\
        0 & \frac{\cos (\phi)}{\sin (\phi)}-\frac{-\sin (\phi)}{\cos (\phi)} &  - \cos^{-1} (\phi) & \sin^{-1} (\phi) \\
    }\\
    \leadsto & \eqmatrix{cc|cc}{
        1 & -\frac{\sin (\phi)}{\cos (\phi)} &  \cos^{-1} (\phi) & 0 \\
        0 & \frac{\cos^2 (\phi) + \sin^2 (\phi)}{\sin (\phi) \cdot \cos (\phi)} & -\frac{1}{\cos (\phi)} & \frac{1}{\sin (\phi)} \\
    }\\
    \leadsto & \eqmatrix{cc|cc}{
        1 & 0 & \cos^{-1} (\phi) - \frac{\sin^2 (\phi)}{\cos (\phi)} & \sin (\phi) \\
        0 & 1 & -\sin (\phi)            & \cos (\phi) \\
    }\\
    \leadsto & \eqmatrix{cc|cc}{
        1 & 0 & \frac{1 - \sin^2 (\phi)}{\cos (\phi)} & \sin (\phi) \\
        0 & 1 & -\sin (\phi)                          & \cos (\phi) \\
    }\\
    \leadsto & \eqmatrix{cc|cc}{
        1 & 0 &  \cos (\phi) & \sin (\phi) \\
        0 & 1 & -\sin (\phi) & \cos (\phi) \\
    }\\
\end{align*}

Für $\phi = \pi \cdot n$ gilt $\cos (\phi) = \pm 1 \wedge \sin (\phi) = 0$:
\begin{align*}
    & \eqmatrix{cc|cc}{
        \cos (\phi) & 0 &  1 & 0 \\
        0 &  \cos (\phi) &  0 & 1 \\
    }\\
    \leadsto & \eqmatrix{cc|cc}{
        1 & 0 & \cos^{-1}(\phi) & 0 \\
        0 & 1 &               0 & \cos^{-1}(\phi) \\
    }
\end{align*}

Für $\phi = \pi \cdot n - \frac{\pi}{2}$ gilt $\cos (\phi) = 0 \wedge \sin (\phi) = \pm 1$:
\begin{align*}
    & \eqmatrix{cc|cc}{
        0           & -\sin (\phi) & 1 & 0 \\
        \sin (\phi) & 0            & 0 & 1 \\
    }\\
    \leadsto & \eqmatrix{cc|cc}{
        0 & 1 & -\sin^{-1} (\phi) & 0 \\
        1 & 0 & 0 & \sin^{-1}(\phi) \\
    }\\
    \leadsto & \eqmatrix{cc|cc}{
        1 & 2 & -2\sin^{-1} (\phi) & \sin^{-1}(\phi) \\
        0 & 1 & -\sin^{-1} (\phi) & 0 \\
    }\\
    \leadsto & \eqmatrix{cc|cc}{
        1 & 0 & 0                 & \sin^{-1}(\phi) \\
        0 & 1 & -\sin^{-1} (\phi) & 0 \\
    }\\
\end{align*}
\end{document}
