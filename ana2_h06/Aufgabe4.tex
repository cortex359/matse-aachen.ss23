\documentclass[main.tex]{subfiles}

\begin{document}

\section{Aufgabe 4}
Sei die Funktion
$f(x,y) = x \cdot y$
sowie das Integrationsgebiet
\[
G = \left\{ (x,y) \in \mathbb{R}^2 \
| \ x \geq 0, \ y \geq 0, \ x^2 + y^2 \leq 2, \ y \leq x^2 \right\}
\]
gegeben. Berechnen Sie das zugehörige Volumen.

\subsection{Lösung 4}


\begin{center}
  \begin{tikzpicture}[scale=1]
      \begin{axis}[
              width=\textwidth,
              unit vector ratio*=1 1 1,
              axis lines = middle,
              ymin=0, ymax=1.6,
              ytick distance=0.2,
              minor y tick num = 1,
              ytick={0.2,0.4,0.6,0.8,1.0,1.2,1.41421356237,1.6},
              yticklabels={ 0.2,0.4,0.6,0.8,1.0,1.2, $\sqrt{2}$, 1.6 },
              xmin=0, xmax=1.6,
              xtick distance=0.2,
              minor x tick num = 1,
              xtick={0.2,0.4,0.6,0.8,1.0,1.2,1.41421356237,1.6},
              xticklabels={ 0.2,0.4,0.6,0.8,1.0,1.2, $\sqrt{2}$, 1.6 },
              xlabel = $x$,
              ylabel = $y$,
              legend cell align={left},
              legend pos={outer north east},
              samples=100,
              domain=0:1.6, y domain=0:1.6,
          ]

          %\addplot [yellow] {x>0, y>=0, x^2 + y^2 <= 2, y<=x^2};
          \addplot [name path=f, red, samples=400, samples at={0.0,0.01,0.02,...,1.3,1.3001,1.3002,...,1.4142,1.41421356237}] {sqrt(2-x^2)};
          \addplot [name path=g, blue] {x^2};

          \path[name path=axis1] (axis cs:1,0) -- (axis cs:1.5,0);
          \path[name path=axis2] (axis cs:0,0) -- (axis cs:1,0);

          \addplot [
              thick,
              color=blue,
              fill=blue,
              fill opacity=0.10
          ]
          fill between[
              of=g and axis2,
              soft clip={domain=0:1},
          ];

          \addplot [
              thick,
              color=red,
              fill=red,
              fill opacity=0.15
          ]
          fill between[
              of=f and axis1,
              soft clip={domain=1:1.41421356237},
          ];

          \legend{$\sqrt{2-x^2}$, $x^2$, $F_1$, $F_2$}
      \end{axis}
  \end{tikzpicture}
\end{center}


Wir formen das Integrationsgebiet $G$ in geeigneter Weise um und erhalten somit
\begin{align*}
    G &= \left\{ (x,y) \in \mathbb{R}^2 \middle| x  \geq 0, y \geq 0, x^2 + y^2 \leq 2, y \leq x^2 \right\} \\
      &= \left\{ (x,y) \in \mathbb{R}^2 \middle|
      \left(0 < x \leq 1, 0 \leq y \leq x^2\right) \wedge \left( 1 < x < \sqrt{2}, 0 \leq y \leq \sqrt{2-x^2} \right)
      \right\}.
\end{align*}

Entsprechend kann über das Gebiet $F_1$ wie folgt integriert werden:
\begin{align*}
    F_1 &= \int_{x=0}^{1} \int_{y=0}^{x^2} x\cdot y \dx{(y,x)} \\
        &= \int_{0}^{1} \left[ \frac{x}{2} y^2 \right]_{y=0}^{x^2} \dx{x} \\
        &= \int_{0}^{1} \frac{1}{2} x^5 \dx{x} \\
        &= \left[ \frac{1}{12} x^6 \right]_{0}^{1} \\
        &= \frac{1}{12} \\
\end{align*}

Für $F_2$ gilt:
\begin{align*}
  F_2 &= \int_{x=1}^{\sqrt{2}} \int_{y=0}^{\sqrt{2-x^2}} x\cdot y \dx{(y,x)} \\
      &= \int_{x=1}^{\sqrt{2}} \left[ \frac{x}{2} y^2 \right]_{y=0}^{\sqrt{2-x^2}} \dx{x} \\
      &= \int_{1}^{\sqrt{2}} \frac{x}{2}\cdot (2-x^2) \dx{x} \\
      &= \int_{1}^{\sqrt{2}} x - \frac{x^3}{2} \dx{x} \\
      &= \left[ \frac{x^2}{2} - \frac{x^4}{8} \right]_{1}^{\sqrt{2}} \\
      &= \frac{2}{2} - \frac{4}{8} - \left( \frac{1}{2} - \frac{1}{8} \right) \\
      &= \frac{1}{8}
\end{align*}

Das Volumen der Funktion $f$ im Gebiet $G$ beträgt somit $F = F_1 + F_2 = \frac{1}{12} + \frac{1}{8} = \frac{5}{24}$ VE.

\end{document}
