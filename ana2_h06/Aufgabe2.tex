\documentclass[main.tex]{subfiles}

\begin{document}

\section{Aufgabe 2}
Gegeben seien die Funktionen 
\[
    f(x) = 5 - \frac{5}{\pi^2} x^2 
    \qquad \mbox{und} \qquad 
    g(x) = 4 \cos \left( \frac{x}{2} \right)
\]
\begin{enumerate}
\item Berechnen Sie die von den beiden Funktionen begrenzte Fläche.
\item Bestimmen Sie anschließend den Schwerpunkt der eingeschlossenen Fläche.
\end{enumerate}
\textit{Hinweise:} Die Schnittstellen der beiden Funktionen sind die Nullstellen. Nutzen Sie zur Berechnung der Fläche ggfls. die Symmetrie der Funktionen. 

\subsection{Lösung 2a}

Berechne die Nullstellen der Funktionen.
\begin{equation*}
    \begin{array}[pos]{lrl}
                         & f(x)    &= 0\\[2mm]
        \Leftrightarrow  & 0       &= 5 - \frac{5}{\pi^2} x^2 \\[2mm]
        \Leftrightarrow  & x_{1,2} &= \pm \pi \\
    \end{array}
\end{equation*}


Der Flächeninhalt $F$ einer Grundfläch $A$ ergibt sich durch Integration mit $f(x,y) = 1$, also
\begin{align*}
    F &= \int_{A} 1 \dx{A}\\
      &= \int_{x=-\pi}^{\pi} \int_{y=g(x)}^{f(x)} 1 \dx{y} \dx{x} \\
      &= \int_{x=-\pi}^{\pi} \int_{y=4 \cos \left( \frac{x}{2} \right)}^{5 - \frac{5}{\pi^2} x^2} 1 \dx{y} \dx{x} \\
      &= \int_{-\pi}^{\pi} \left(5 - \frac{5}{\pi^2} x^2\right) - \left(4 \cos \left(\frac{x}{2} \right)\right) \dx{x} \\
      &= \left[5x - \frac{5}{\pi^2\cdot 3} x^3 - 8 \sin \left(\frac{x}{2} \right)\right]_{-\pi}^{\pi} \\
      &= \left(5\pi - \frac{5}{\pi^2\cdot 3} \pi^3 - 8 \sin \left(\frac{\pi}{2} \right)\right) - \left(-5\pi + \frac{5}{\pi^2\cdot 3} \pi^3 - 8 \sin \left(\frac{-\pi}{2} \right)\right) \\
      &= \frac{20}{3} \pi - 16 \\
\end{align*}

\subsection{Lösung 2b}
Der Schwerpunkt mit den Koordinaten $(x_s, y_s)$ einer solchen Fläche mit homogener Dichte ergibt sich mit der zuvor berechneten Flächenmaßzahl $F$ durch
\begin{align*}
    x_s = \frac{1}{F} \int_{A}x \dx{A} \quad \wedge \quad y_s = \frac{1}{F} \int_{A} y \dx{A}.
\end{align*}

Für $x_s$ also
\begin{align*}
    x_s &= \frac{1}{F} \int_{A}x \dx{A} \\
        &= \frac{1}{F}\int_{x=-\pi}^{\pi} \int_{y=g(x)}^{f(x)} x \dx{y} \dx{x} \\
        &= \frac{1}{F}\int_{x=-\pi}^{\pi} x \int_{y=g(x)}^{f(x)} 1 \dx{y} \dx{x} \\
        &= \frac{1}{F}\int_{x=-\pi}^{\pi} x \left(5 - \frac{5}{\pi^2} x^2 - 4 \cos \left(\frac{x}{2} \right)\right) \dx{x} \\
        %&= \frac{1}{F} \int_{x=-\pi}^{\pi} 5x - \frac{5}{\pi^2} x^3 - 4x \cos \left(\frac{x}{2} \right) \dx{x} \\
        &= \frac{1}{F} \left[ \frac{5}{2} x^2 - \frac{5}{\pi^2\cdot 4} x^4 - 8x \sin \left(\frac{x}{2}\right) - 16 \cos \left(\frac{x}{2} \right) \right]_{x=-\pi}^{\pi} \\
        &= \frac{1}{F} \left( \frac{5}{2} \pi^2 - \frac{5}{4} \pi^2 - 8\pi - \frac{5}{2} \pi^2 + \frac{5}{4} \pi^2 + 8\pi \right)\\
        &= 0 \\
\end{align*}

und entsprechend für $y_s$
\begin{align*}
    y_s &= \frac{1}{F} \int_{A}y \dx{A} \\
        &= \frac{1}{F} \int_{x=-\pi}^{\pi} \int_{y=g(x)}^{f(x)} y \dx{y} \dx{x} \\
        &= \frac{1}{F} \int_{-\pi}^{\pi} \left[ \frac{y^2}{2} \right]_{g(x)}^{f(x)} \dx{x} \\
        &= \frac{1}{F} \int_{-\pi}^{\pi} \frac{1}{2} \left(5 - \frac{5}{\pi^2} x^2 \right)^2 - \frac{1}{2}\left(4 \cos \left( \frac{x}{2} \right)\right)^2 \dx{x} \\
        &= \frac{1}{F} \int_{-\pi}^{\pi} \left(\frac{5^2}{2} - \frac{5^2\cdot x^2}{\pi^2} + \frac{5^2\cdot x^4}{\pi^4\cdot 2} \right) - 8 \cos^2 \left( \frac{x}{2} \right) \dx{x} \\
        &= \frac{1}{F} \int_{-\pi}^{\pi} 5^2\cdot \left(\frac{1}{2} - \frac{x^2}{\pi^2} + \frac{x^4}{2\pi^4} \right) - 8 \cos^2 \left( \frac{x}{2} \right) \dx{x} \\
        &= \frac{1}{F} \left[ 5^2\cdot \left(\frac{1}{2} x - \frac{x^3}{3\pi^2} + \frac{x^5}{10\pi^4} \right) - 4x +\sin(x) \right]_{-\pi}^{\pi} \\
        &= \frac{1}{F} \left( 5^2\cdot \left(\frac{1}{2} \pi - \frac{\pi}{3} + \frac{\pi}{10} \right) - 4\pi \right) 
            - \left( 5^2\cdot \left(\frac{-\pi}{2} + \frac{\pi}{3} - \frac{\pi}{10} \right) + 4\pi \right) \\
        &= \frac{1}{F} \left( 5^2 \pi \cdot \left(\frac{1}{2} - \frac{1}{3} + \frac{1}{10} \right) 
            - 5^2\cdot \pi \left(\frac{-1}{2} + \frac{1}{3} - \frac{1}{10} \right) - 8\pi \right)\\
        &= \frac{1}{F} \left( 5^2 \pi \cdot \frac{8}{15} - 8\pi \right) \\
        &= \frac{1}{\frac{20}{3} \pi - 16} \pi \cdot \frac{16}{3} \\
\end{align*} n*}



\end{document}
