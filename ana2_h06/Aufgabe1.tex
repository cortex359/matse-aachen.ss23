\documentclass[main.tex]{subfiles}

\begin{document}

\section{Aufgabe 1}
Bestimmen Sie die Fläche, die begrenzt ist durch die Parabeln
\[
    y^2 = 4 - x 
    \qquad \mbox{und} \qquad
    y^2 = 4 - 4x
\]

\subsection{Lösung 1}

Die beiden Parabeln in Blau, lassen sich durch Multiplikation mit einer Rotationsmatrix $D_\alpha = \eqmatrix{cc}{\cos (\alpha) & -\sin(\alpha)\\ \sin(\alpha) & \cos(\alpha)}$
mit $\alpha = \frac{\pi}{2}$ um $90^\circ$ rotieren (Parabeln in Rot).

\begin{center}
    \begin{tikzpicture}[scale=1]
        \begin{axis}[
                width=\textwidth,
                unit vector ratio*=1 1 1,
                axis lines = middle,
                no markers,
                ymin=-4, ymax=4,
                xmin=-4, xmax=4,
                xtick distance=1, ytick distance=1,
                xlabel = $x$, ylabel = $y$,
                legend cell align={left},
                samples=100,
                domain=-4:4, y domain=-4:4,
            ]
    
            %\draw[black,->] (axis cs:0,0) -- (axis cs:2,1) node [midway, above left] {$x_0$};
            %\addlegendimage{black,->}
            %\addlegendentry{$x_0 = (2, 1)^T$}
    
            \addplot +[blue, samples at={4,3.999,3.995,3.99,...,3.8,3.7,3.6,...,-4}] {sqrt(4-x)};
            \addplot +[blue, samples at={4,3.999,3.995,3.99,...,3.8,3.7,3.6,...,-4}] {-sqrt(4-x)};
            \addplot +[blue, samples at={1,0.999,0.995,0.99,...,0.8,0.7,0.6,...,-4}] {sqrt(4-4*x)};
            \addplot +[blue, samples at={1,0.999,0.995,0.99,...,0.8,0.7,0.6,...,-4}] {-sqrt(4-4*x)};

            \addplot [name path=a, red] {(1/4)*x^2-1};
            \addplot [name path=b, red] {x^2-4};

            \addplot [
                thick,
                color=red,
                fill=red, 
                fill opacity=0.05
            ]
            fill between[
                of=a and b,
                soft clip={domain=-2:2},
            ];

        \end{axis}
    \end{tikzpicture}
\end{center}

Aus $D_{\frac{\pi}{2}} \times \vektor{x \\ y} = \vektor{-y \\ x}$ ergeben sich die Gleichungen
\begin{align*}
    x^2 = 4+x    &\Leftrightarrow y = x^2 -4 \\
    x^2 = 4 + 4y &\Leftrightarrow y = \frac{1}{4} x^2 -1.
\end{align*}

Die Fläche zwischen den blauen Graphen ist gleich der Fläche zwischen den roten Graphen. 

\begin{align*}
    A &= \left| \int_{-2}^{2} x^2-4 \dx{x} \right| - \left| \int_{-2}^{2} \frac{1}{4} x^2 -1 \dx{x} \right| \\[2mm]
      &= \left| \left[ \frac{1}{3} x^3 - 4x \right]_{-2}^{2}\right| - \left| \left[ \frac{1}{12} x^3 - x \right]_{-2}^{2}\right| \\[2mm]
      &= \frac{32}{3} - \frac{8}{3} \\[2mm]
      &= 8
\end{align*}

Zur Überprüfung vergleichen wir mit
\begin{align*}
    \frac{1}{2} A = \int_{0}^{4} \sqrt{4-x} \dx{x} - \int_{0}^{1} \sqrt{4-4x} \dx{x}
\end{align*}
und erhalt ebenfalls $A=8.\ \checkmark$



\end{document}
