\documentclass[main.tex]{subfiles}

\begin{document}

\section{Aufgabe 2}
Lösen Sie das Anfangswertproblem $\quad y' = -2x \cdot y + 10x, \quad  y(0) = 2$

\subsection{Lösung 2}
\begin{align*}
    y' &= -2xy + 10x \\
    &= \underbrace{x}_{f(x)}\cdot \underbrace{(10-2y)}_{g(y)}
\end{align*}

Wir trennen die Variablen und erhalten
\begin{equation*}
    \begin{array}{lrl}
        & \int \frac{1}{10-2y} \dx{y(x)} &= \int x \dx{x} \\
        \Leftrightarrow & -\frac{1}{2} \ln (5 - y(x)) &= \frac{x^2}{2} + C \\
        \Leftrightarrow & \ln (5 - y(x))  &= -x^2 - 2C \\
        \Leftrightarrow & 5 - y(x)  &= \ct{e}^{-x^2 - 2C} \\
        \Leftrightarrow & y(x)  &= 5 - \ct{e}^{-x^2 - 2C} \\
    \end{array}
\end{equation*}

Da $y(0) = 5-\ct{e}^{-2C} = 2 \Leftrightarrow C = -\frac{\ln(3)}{2}$ folgt somit
$$
    y(x) = 5 - 3\ct{e}^{- x^2}
$$

\end{document}
