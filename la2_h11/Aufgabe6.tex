\documentclass[main.tex]{subfiles}

\begin{document}

\section{Aufgabe 6}
Bestimmen Sie die Lösung mit der kleinsten Norm des folgenden unterbestimmten Gleichungssystems.
\begin{equation*}
\begin{array}{rrrr}
     x &+ 2y &+ 3z &= 4 \\
    3x &+ 2y &+ z  &= 2 \\
\end{array}
\end{equation*}

\subsection{Lösung 6}
\begin{align*}
    A &= \eqmatrix{rrr}{1 & 2 & 3 \\ 3 & 2 & 1} \\
    b &= \eqmatrix{r}{4 \\ 2} \\[4mm]
    A^T A &= \eqmatrix{rrr}{10 & 8 & 6 \\ 8 & 8 & 8 \\ 6 & 8 & 10} \\[2mm]
    A^T b &= \eqmatrix{r}{10 \\ 12 \\ 14} \\
\end{align*}

Aus $A^TA x = A^Tb$ ergibt sich die erweiterte Koeffizientenmatrix, welche mit dem freien Parameter $x_3 = \lambda \in \mathbb{R}$ folgende Lösungsmenge für $x \in \mathbb{R}^3$ liefert.
\begin{align*}
    \eqmatrix{rrr|r}{
        10 & 8 &  6 & 10 \\
         8 & 8 &  8 & 12 \\
         6 & 8 & 10 & 14
    } \leadsto
    \eqmatrix{rrr|r}{
        1 & 0,8 & 0,6 & 1 \\
        0 &   1 &   2 & 2,5 \\
        0 &   0 &   0 & 0
    }\\ \Rightarrow \mathcal{L} = \left\{
        \lambda \in \mathbb{R}
        \middle|
        x = \lambda \eqmatrix{r}{1 \\ -2 \\ 1} + \eqmatrix{r}{-1 \\ \sfrac{5}{2} \\ 0} \right\}
\end{align*}


\end{document}
