\documentclass[main.tex]{subfiles}

\begin{document}

\section{Aufgabe 5}
Matse Maik plant eine Cafeteria und unterhält sich mit verschiedenen Freunden bzgl. angemessener Preise.
Es werden Kuchen und Kaffee verkauft. Er hat in Erinnerung, letztens für \textbf{zwei Kaffee und ein Stück Kuchen 3 Euro} bezahlt zu haben.
Zwei Freundinnen erzählen ihm, für \textbf{einen Kaffee und zwei Stücke Kuchen 3,50 Euro}, 
bzw. für \textbf{zwei Kaffee und drei Stücke Kuchen 4 Euro} bezahlt zu haben.\\
Berechnen Sie Preise, die möglichst wenig von den erfragten Preisen abweichen.

\subsection{Lösung 5}
Aus dem gegebenen Text ergibt sich das überbestimmte Gleichungssystem $Ax=b$ wie folgt.
$$
    A = \eqmatrix{rr}{
        2 & 1 \\
        1 & 2 \\
        2 & 3 \\
    }, \qquad
    x = \vektor{
        \texttwemoji{coffee} \\
        \texttwemoji{cake}
    }, \qquad
    b = \vektor{
        3 \\
        3,5 \\
        4
    }
$$

% Voraussetzung für die \textit{Methode der kleinsten Quadrate} ist, dass der Rang von $A$ gleich der Anzahl der Spalten von $A$ ist. \\
Für ein Gleichungssystem $Ax=b$ mit
$$
    A\in \mathbb{R}^{m\times n},\; b\in\mathbb{R}^m,\; m\geq n, \; \rg(A) = n
$$
lässt sich die Näherungslösung $x_s$ nach der \textit{Methode der kleinsten Quadrate} mit
$$
    x_s = \left( A^T A \right)^{-1} A^T b
$$
oder in Form der Normalgleichung
$$
    A^TAx = A^Tb
$$
bestimmen.

\begin{align*}
    A^T A &= \eqmatrix{rr}{9 & 10 \\ 10 & 14} \\
    A^T b &= \eqmatrix{r}{17,5 \\ 22} \\
\end{align*}
Mit dem Gauß-Jordan-Algorithmus lösen wir das Normalgleichungssystem
$$
    \eqmatrix{rr}{9 & 10 \\ 10 & 14} \vektor{
        \texttwemoji{coffee} \\
        \texttwemoji{cake}
    } = \vektor{17,5 \\ 12}
$$

und erhalten $\texttwemoji{coffee} = \frac{25}{26}$ und $\texttwemoji{cake} =\frac{23}{26}$


\end{document}
