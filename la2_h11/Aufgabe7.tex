\documentclass[main.tex]{subfiles}

\begin{document}

\section{Aufgabe 7}
Mit der Wassertiefe ändert sich der Druck, der auf einen im Wasser befindlichen Körper wirkt. 
Es wird ein Experiment durchgeführt, um den vermuteten Zusammenhang 
$$
    P = \alpha+\beta\cdot d
$$
zwischen Wassertiefe $d$ und Druck $P$ zu überprüfen. Es wurden folgende Messwerte aufgenommen:

\begin{center}
	\begin{tabular}{l|ccccc}
		Wassertiefe & 1 & 3 & 5 & 7 & 9\\
		\hline
		Druck       & 2 & 4 & 5,5 & 8,5 & 10
	\end{tabular}	
\end{center}
\begin{enumerate}
	\item Bestimmen Sie die Parameter $\alpha$ und $\beta$ nach der Methode der kleinsten Quadrate.
	\item Ermitteln Sie mit diesen Werten den Druck in einer Tiefe von 15 Metern.
\end{enumerate}

\subsection{Lösung 7}

\end{document}
