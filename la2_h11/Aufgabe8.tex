\documentclass[main.tex]{subfiles}

\begin{document}

\section{Aufgabe 8}
Die Punkte $A=(6|0|0)$, $B=(2|1|3)$ und $C=(-2|-2|2)$ liegen in einer Ebene $E$.
\begin{enumerate}
	\item Stellen Sie die Hessesche Normalform der Ebene auf. Wie groß ist der Abstand der Ebene zum Ursprung?
	\item Welcher Punkt in der Ebene hat den kleinsten Abstand zum Ursprung? Stellen Sie dazu das zugehörige unterbestimmte LGS auf und finden Sie die Lösung mithilfe der verallgemeinerten Inverse.
\end{enumerate}

\subsection{Lösung 8a}
Um die Hessesche Normalform der Ebene zu bestimmen, können wir den Normalenvektor der Ebene finden. Der Normalenvektor ist orthogonal zu jedem Vektor, der in der Ebene liegt. 

Wir können zwei Vektoren in der Ebene verwenden, um den Normalenvektor zu bestimmen. Zum Beispiel können wir die Vektoren $\overrightarrow{AB}$ und $\overrightarrow{AC}$ verwenden.

Der Vektor $\overrightarrow{AB}$ ist gegeben durch:

$$\overrightarrow{AB} = \begin{pmatrix} 2-6 \\ 1-0 \\ 3-0 \end{pmatrix} = \begin{pmatrix} -4 \\ 1 \\ 3 \end{pmatrix}$$

Der Vektor $\overrightarrow{AC}$ ist gegeben durch:

$$\overrightarrow{AC} = \begin{pmatrix} -2-6 \\ -2-0 \\ 2-0 \end{pmatrix} = \begin{pmatrix} -8 \\ -2 \\ 2 \end{pmatrix}$$

Um den Normalenvektor zu bestimmen, berechnen wir das Kreuzprodukt von $\overrightarrow{AB}$ und $\overrightarrow{AC}$:

$$\overrightarrow{n} = \overrightarrow{AB} \times \overrightarrow{AC} = \begin{pmatrix} -4 \\ 1 \\ 3 \end{pmatrix} \times \begin{pmatrix} -8 \\ -2 \\ 2 \end{pmatrix} = \vektor{8\\ -16 \\ 16}$$

Wir normieren zunächst noch den Normalenvektor
\begin{align*}
	n_0 &= \frac{\overrightarrow{n}}{{\|\overrightarrow{n}\|}} \\
	n_0 &= \frac{1}{3} \vektor{1\\ -2 \\ 2}
\end{align*}

und stellendie Hessesche Normalform der Ebene dann mit de Normalenvektor $\overrightarrow{n_0}$ und einem Punkt in der $A=(6|0|0)$ als Stützvektor auf:

$$
	E: \frac{1}{3} \vektor{1\\ -2 \\ 2} \cdot \left[ \overrightarrow{x} - \vektor{6 \\ 0 \\ 0} \right] = 0
$$

\end{document}
