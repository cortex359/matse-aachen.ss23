\documentclass[main.tex]{subfiles}

\begin{document}

\section{Aufgabe 8}
Die Punkte $A=(6|0|0)$, $B=(2|1|3)$ und $C=(-2|-2|2)$ liegen in einer Ebene $E$.
\begin{enumerate}
	\item Stellen Sie die Hessesche Normalform der Ebene auf. Wie groß ist der Abstand der Ebene zum Ursprung?
	\item Welcher Punkt in der Ebene hat den kleinsten Abstand zum Ursprung? Stellen Sie dazu das zugehörige unterbestimmte LGS auf und fnden Sie die Lösung mit Hilfe der verallgemeinerten Inverse.
\end{enumerate}

\subsection{Lösung 8}

\end{document}
