\documentclass[main.tex]{subfiles}

\begin{document}

\section{Aufgabe 2}
Berechnen Sie alle relativen Extrema und Sattelpunkte der Funktionen

\[
    f(x, y) = x^2\cdot y^2
\]

\subsection{Lösung 2}

Bestimme $\nabla f$

\begin{align*}
    \nabla f &= \vektor{f_x \\ f_y} \\
    &= \vektor{
        2x\cdot y^2 \\
        2x^2\cdot y
    }
\end{align*}

Wir setzten jede Zeile des Vektors $\nabla f$ null und lösen das Gleichungssystem. Wir erkennen leicht, dass nur $(x, y) = (0, 0)$ eine Lösung ist, also haben wir eine Extremstelle in $(0, 0)$.

Um zusätzlich die Art des Extremums zu bestimmen brauchen wir die Hesse-Matrix der Funktion, und dafür wiederum die zweiten partiellen Ableitung: 

\begin{align*}
    f_{xx} &= 2y^2 \\
    f_{yy} &= 2x^2 \\
    f_{xy} &= 4xy
\end{align*}

Dies fassen wir in der Hesse-Matrix zusammen:

\begin{align*}
    H_f(x,y) &= \eqmatrix{lr}{
        f_{xx} & f_{xy} \\
        f_{yx} & f_{yy} \\
    } \\
    &= \eqmatrix{lr}{
        2y^2 & 4xy \\
         4xy & 2x^2 \\
    } \\
\end{align*}

Nun setzten wir die zuvor gefundene Extremstelle $(0, 0)$ in die Hesse-Matrix ein:
\begin{align*}
    H_f(0, 0) &= \eqmatrix{lr}{
        0 & 0 \\
        0 & 0 \\
    } \\
\end{align*}


\end{document}
