\documentclass[main.tex]{subfiles}

\begin{document}

\arraycolsep=1pt\def\arraystretch{1}
\section{Aufgabe 7}
Über eine lineare Abbildung $f$ sei folgendes bekannt.
$$
    f\left(\begin{pmatrix}
        1 \\ 1 \\ 0
    \end{pmatrix}\right) = \begin{pmatrix}
        1 \\ 1
    \end{pmatrix},\,
    f\left(\begin{pmatrix}
        0 \\ 1 \\ 1
    \end{pmatrix}\right) = \begin{pmatrix}
        0 \\ 1
    \end{pmatrix},\,
    f\left(\begin{pmatrix}
        1 \\ 0 \\ \alpha
    \end{pmatrix}\right) = \begin{pmatrix}
        1 \\ -1
    \end{pmatrix}\\
    $$

\begin{enumerate}
    \item Für welche Werte von $\alpha$ ist die Abbildungsmatrix $A_f$ von $f$ eindeutig bestimmt?
    \item Bestimmen Sie die Matrix $A_f$ in Abhängigkeit von $\alpha$.
    \item Bestimmen Sie den Kern und das Bild der Abbildung in Abhängigkeit von $\alpha$.
\end{enumerate}

\subsection{Lösung 7}
Wir betrachten die lineare Abbildung $f : V \to W$ mit $V \subseteq \mathbb{R}^3$ und $W \subseteq \mathbb{R}^2$.

Wenn $\mathcal{B}_V = \left(\begin{pmatrix}
    1 \\ 1 \\ 0
\end{pmatrix}, \begin{pmatrix}
    0 \\ 1 \\ 1
\end{pmatrix}, \begin{pmatrix}
    1 \\ 0 \\ \alpha
\end{pmatrix}\right)$
eine Basis von $V$ ist, dann ist die Abbildungsmatrix eindeutig bestimmt. Also prüfen wir die Vektoren auf lineare Unabhängigkeit und stellen fest, dass für $\alpha = - 1$ der dritte Vektor zu einer Linearkombination der ersten beiden wird. In einem solchen Fall ist die Abbildungsmatrix nicht eindeutig bestimmt.

Wenn $\alpha 

\end{document}
