\documentclass[main.tex]{subfiles}

\begin{document}

\section{Aufgabe 5}
\textit{Hinweis: Umbenennung des Parameters von $x$ nach $a$.}\\
Berechnen Sie in Abhängigkeit von $a$

\begin{enumerate}
    \item den Kern
    \item die Dimension des Kerns
    \item den Rang
    \item das Bild
\end{enumerate}
der zu der folgenden Matrix gehörenden linearen Abbildung:
$$
    \arraycolsep=0.6em
    A = \begin{pmatrix}[1]
        2 & 1 & 2 & 3 \\
        2 & 5 & 4 & 3 \\
        1 & 2 & 5 & a \\
        2 & 1 & 3 & 5
    \end{pmatrix}
$$

\subsection{Lösung 5}
\textit{Dimensionsformel für lineare Abbildungen:}\\
Für eine lineare Abbildung $f: V \to W$ gilt
$$
    \dim (\ker (f)) + \operatorname{rg}(f) = \dim (V).
$$

\subsection{Lösung 5a}

Der Kern einer linearen Abbildung $A$ ist die Menge aller Vektoren $x \in \mathbb{R}^4$, die von $A$ auf den Nullvektor abgebildet werden, d.h.
$$
    \ker (A) = \left\{ x\in \mathbb{R}^4 \middle| Ax=0 \right\}.
$$

Durch elementare Zeilenumformungen erhalten wir die Zeilenstufenform der Koeffizientenmatrix

\begin{gather*}
    \arraycolsep=0.6em
    \begin{array}{ r l }
        & \begin{pmatrix}[1]
            2 & 1 & 2 & 3 \\
            2 & 5 & 4 & 3 \\
            1 & 2 & 5 & a \\
            2 & 1 & 3 & 5
        \end{pmatrix}\\
        \rightsquigarrow &
        \begin{pmatrix}[1]
            1 & \nicefrac{1}{2} & 1 & \nicefrac{3}{2} \\
            0 &   4 & 2 &   0 \\
            0 & \nicefrac{3}{2} & 4 & a - \nicefrac{3}{2} \\
            0 &   0 & 1 &   2
        \end{pmatrix}\\
        \rightsquigarrow &
        \begin{pmatrix}[1]
            1 & \nicefrac{1}{2} & 1 & \nicefrac{3}{2} \\
            0 & 1 & \nicefrac{1}{2} & 0 \\
            0 & 0 & \frac{13}{6} & \frac{2a}{3} - 1 \\
            0 & 0 & 1 & 2
        \end{pmatrix}\\
        \rightsquigarrow &
        \begin{pmatrix}[1]
            1 & \nicefrac{1}{2} & 1 & \nicefrac{3}{2} \\
            0 & 1 & \nicefrac{1}{2} & 0 \\
            0 & 0 & 1 & 4a - 6 \\
            0 & 0 & 0 & 20 - 4a
        \end{pmatrix}
    \end{array}
\end{gather*}

% Die Determinante ist eine normierte, alternierende Multiliniarform auf der Menge der quadtratsichen Matrizen einer festen größe
% \det{A} = 64 - 8x

Es gilt $A x = f(x) = 0$ und wir erhalten

$x_4 \cdot (20-4a) = 0 \Leftrightarrow x_4 = 0 \vee  a = 5$

$x_3 + x_4 \cdot (4a-6)  = 0$
$x_3 = - 14 x_4$



\subsection{Lösung 5}


\end{document}
