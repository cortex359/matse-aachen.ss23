\documentclass[main.tex]{subfiles}

\begin{document}

\section{Aufgabe 5}
\textit{Hinweis: Umbenennung des Parameters von $x$ nach $a$.}\\
Berechnen Sie in Abhängigkeit von $a$

\begin{enumerate}
    \item den Kern
    \item die Dimension des Kerns
    \item den Rang
    \item das Bild
\end{enumerate}
der zu der folgenden Matrix gehörenden linearen Abbildung:
$$
    \arraycolsep=0.6em
    A = \begin{pmatrix}
        2 & 1 & 2 & 3 \\
        2 & 5 & 4 & 3 \\
        1 & 2 & 5 & a \\
        2 & 1 & 3 & 5
    \end{pmatrix}
$$

\subsection{Lösung 5}
\textit{Dimensionsformel für lineare Abbildungen:}\\
Für eine lineare Abbildung $f: V \to W$ gilt
$$
    \dim (\ker (f)) + \operatorname{rg}(f) = \dim (V).
$$
Statt $\dim (\ker (f))$ kann man auch $\operatorname{def}(f)$ schreiben.


Der Kern einer linearen Abbildung $A$ ist die Menge aller Vektoren $x \in \mathbb{R}^4$, die von $A$ auf den Nullvektor abgebildet werden. Ganz allgemein bedeutet das, dass der Kern die Menge der Lösungen des Gleichungssystems $Ax=0$ ist:
$$
    \ker (A) = \left\{ x\in \mathbb{R}^4 \middle| Ax=0 \right\}.
$$

Um diese Menge konkret zu bestimmen, formen wir durch elementare Zeilenoperationen die Abbildungsmatrix so lange um, bis wir die Zeilenstufenform erhalten und dabei den Parameter $a$ möglichst frei stehend haben. 

\begin{gather*}
    \arraycolsep=0.6em
    \begin{array}{ r l }
        & \begin{pmatrix}
            2 & 1 & 2 & 3 \\
            2 & 5 & 4 & 3 \\
            1 & 2 & 5 & a \\
            2 & 1 & 3 & 5
        \end{pmatrix}\\[10mm]
    \overset{\substack{Z_2 - Z_1 \\ Z_4 - Z_1}}{\rightsquigarrow} &
        \begin{pmatrix}
            2 & 1 & 2 & 3 \\
            0 & 4 & 2 & 0 \\
            1 & 2 & 5 & a \\
            0 & 0 & 1 & 2
        \end{pmatrix}\\[10mm]
    \overset{\substack{Z_1 : 2 \\ Z_2 : 2 \\ Z_3 - Z_1}}{\rightsquigarrow} &
        \begin{pmatrix}
            1 & \sfrac{1}{2} & 1 & \sfrac{3}{2} \\
            0 & 1 & \sfrac{1}{2} & 0 \\
            0 & \sfrac{3}{2} & 4 & a - \sfrac{3}{2} \\
            0 & 0 & 1 & 2
        \end{pmatrix}\\[10mm]
    \overset{\substack{Z_3 - \sfrac{3}{2} Z_2\\Z_3\cdot \sfrac{4}{13}}}{\rightsquigarrow} &
        \begin{pmatrix}
            1 & \sfrac{1}{2} & 1 & \sfrac{3}{2} \\
            0 & 1 & \sfrac{1}{2} & 0 \\
            0 & 0 & 1 & \sfrac{4a - 6}{13} \\
            0 & 0 & 1 & 2
        \end{pmatrix}\\[10mm]
    \overset{\substack{Z_4 - Z_3\\Z_{3,4}\cdot \sfrac{13}{4}\\Z_3 + Z_4}}{\rightsquigarrow} &
        \begin{pmatrix}
            1 & \sfrac{1}{2} & 1 & \sfrac{3}{2} \\
            0 & 1 & \sfrac{1}{2} & 0 \\
            0 & 0 & \sfrac{13}{4} & \sfrac{13}{2} \\
            0 & 0 & 0 & 8 - a
        \end{pmatrix}\\[10mm]
    \overset{\substack{Z_1\cdot4\\Z_1 -2\cdot Z_2\\Z_1 -3\cdot Z_3\\Z_1:4}}{\rightsquigarrow} &
        \begin{pmatrix}
            1 & 0 & 0 & 0 \\
            0 & 1 & \sfrac{1}{2} & 0 \\
            0 & 0 & 1 & 2 \\
            0 & 0 & 0 & 8 - a
        \end{pmatrix}\\[10mm]
    \end{array}
\end{gather*}

% Die Determinante ist eine normierte, alternierende Multilinearform auf der Menge der quadtratsichen Matrizen einer festen größe
% \det{A} = 64 - 8x

Aus der letzten Zeile entnehmen wir, dass $(8-a)\cdot x_4 =0$ ist. Dadurch erhalten wir zwei Fälle, die wir unterscheiden müssen: \\

\underline{1. Fall, $a = 8$:}\\
Wenn $a = 8$ ist, so kann $x_4$ einen beliebigen Wert annehmen, daher wählen wir $x_4 = \lambda \in \R$ \textit{beliebig} und setzen ein um die übrigen Koordinaten von $x$ zu bestimmen:
$x_3 + 2\lambda = 0 \Leftrightarrow x_3 = -2\lambda$, 
$x_2 + \frac{1}{2} x_3 \Leftrightarrow x_2 = \lambda$ und $x_1 = 0$. 
Daraus folgt:
$$
\ker(A_{a=8}) = \left\{ \lambda\cdot \vektor{0\\1\\-2\\1} \middle| \lambda \in \R \right\}
$$
Die Dimension des Kerns (oder auch Defekt $\operatorname{def}$ genannt) ist $\dim\left(\ker(A_{a=8})\right) = \operatorname{def} (A_{a=8}) = 1$.

Mit der Rangformel ergibt sich für den Rang
$$
\rg(A_{a=8}) = \dim(V) - \operatorname{def} (A_{a=8}) = 4 - 1 = 3.
$$

Das Bild der Matrix ist die Menge aller Linearkombinationen der Spaltenvektoren, sodass wir diese einfach wie folgt angeben können:
$$
    \Bild (A) = \left\{
        \lambda_1 \vektor{1\\0\\0\\0} + 
        \lambda_2 \vektor{0\\1\\0\\0} + 
        \lambda_3 \vektor{0\\0\\1\\0}
        \middle| \lambda_{1,2,3} \in \R
      \right\}
$$

\underline{2. Fall, $a \neq 8$:}\\
Wenn $a \neq 8$ ist, dann muss $x_4 = 0$ sein. Für die übrigen Koordinaten ergibt sich so ebenfalls $x_4 = x_3 = x_2 = x_1 = 0$, was bedeutet, dass lediglich der Nullvektor eine Lösung des Gleichungssystems $Ax=0$ ist und $\ker(A_{a\neq 8})=\{0\}$ ist.

Der Defekt ist $\operatorname{def}(A_{a\neq 8}) = 0$.

Mit der Rangformel ergibt sich für den Rang
$$
\rg(A_{a\neq 8}) = \dim(V) - \operatorname{def} (A_{a\neq 8}) = 4 - 0 = 4.
$$

Da der Rang der Matrix gleich der Anzahl der Spaltenvektoren ist, können wir das Bild mit den Spalten der Ausgangsmatrix beschreiben.
$$
    \Bild (A) = \left\{
        \lambda_1 \vektor{2\\2\\1\\2} + 
        \lambda_2 \vektor{1\\5\\2\\1} + 
        \lambda_3 \vektor{2\\4\\5\\3}
        \lambda_4 \vektor{3\\3\\a\\5}
        \middle| \lambda_{1,2,3,4} \in \R
      \right\}
$$

\end{document}
