\documentclass[main.tex]{subfiles}

\begin{document}

\section{Aufgabe 6}
Stellen Sie zu folgenden Abbildungen die zugehörigen Abbildungsmatritzen auf.

\begin{enumerate}
    \item $f: \mathbb{P}^2 \to \mathbb{P}^3, \ f(p(x)) = \int_{C=0} p(x) \dx$
    \item $g: \mathbb{P}^3 \to \mathbb{P}^2, \ g(p(x)) = p'(x).$
    \item Können Sie den Wertebereich von $f$ so einschränken, dass $f$ bijektiv ist? Falls ja, wie lautet die Umkehrabbildung von $f$?
\end{enumerate}

\subsection{Lösung 6a}
Die Abbildung der Polynome maximal zweiten Grades auf ihre Stammfunktion aus den Polynomen maximal dritten Grades.
$$
    f : \begin{cases}
        \mathbb{P}^2 \to \mathbb{P}^3\\
        p(x) \mapsto \int_{C=0} p(x) \dx
    \end{cases}
$$

Zur Bestimmung der Abbildungsmatrix nehmen wir die Monombasis $(1, x, x^2) \in \mathbb{P}^2$ und wenden die Abbildung $f$ auf ihre Vektoren an. 
\begin{align*}
    f\left(1\right) &= x                 & f\left(\vektor{1\\0\\0}\right) &= \vektor{0\\1\\0\\0} \\
    f\left(x\right) &= \frac{1}{2} x^2   & f\left(\vektor{0\\1\\0}\right) &= \vektor{0\\0\\\nicefrac{1}{2}\\0} \\
    f\left(x^2\right) &= \frac{1}{3} x^3 & f\left(\vektor{0\\0\\1}\right) &= \vektor{0\\0\\0\\\nicefrac{1}{3}} \\
\end{align*}

Die Bilder der Basisvektoren sind die Spalten der Abbildungsmatrix. Somit erhalten wir:

$$
    A_f = \begin{pmatrix}
        0 & 0 & 0 \\
        1 & 0 & 0 \\
        0 & \nicefrac{1}{2} & 0 \\
        0 & 0 & \nicefrac{1}{3}
    \end{pmatrix}
$$

\subsection{Lösung 6b}
Nach dem gleichen Verfahren erhalten wir für die Abbildung
$$
    g : \begin{cases}
        \mathbb{P}^3 \to \mathbb{P}^2\\
        p(x) \mapsto p'(x)
    \end{cases}
$$

die Abbildungsmatrix

$$
    A_g = \begin{pmatrix}
        0 & 1 & 0 & 0 \\
        0 & 0 & 2 & 0 \\
        0 & 0 & 0 & 3
    \end{pmatrix}.
$$

\subsection{Lösung 6c}

Begrenzt man den Wertebereich der injektiven Abbildung $f$ auf die Menge aller Polynome maximal vierten Grades ohne die Polynome vom Grad Null
$\tilde{f}:\mathbb{P}^2 \to \mathbb{P}^3\setminus \mathbb{P}^0$,
so wird die Abbildung surjektiv und damit bijektiv.\\

Die Umkehrabbildung $f^{-1}$ kann nun so angegeben werden:

$$
    f^{-1} : \begin{cases}
        \mathbb{P}^3 \setminus \mathbb{P}^0 \to \mathbb{P}^2\\
        p(x) \mapsto p'(x)
    \end{cases}
$$

Die Abbildungsmatrix von $A_{f^{-1}}$ ist entsprechend $A_g$ aus dem Aufgabenteil b. 


\end{document}
