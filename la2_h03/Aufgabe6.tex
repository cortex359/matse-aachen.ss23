\documentclass[main.tex]{subfiles}

\begin{document}

\section{Aufgabe 6}
Stellen Sie zu folgenden Abbildungen die zugehörigen Abbildungsmatritzen auf.

\begin{enumerate}
    \item $f: \mathbb{P}^2 \to \mathbb{P}^3, \ f(p(x)) = \int_{C=0} p(x) \dx$
    \item $g: \mathbb{P}^3 \to \mathbb{P}^2, \ g(p(x)) = p'(x).$
    \item Können Sie den Wertebereich von $f$ so einschränken, dass $f$ bijektiv ist? Falls ja, wie lautet die Umkehrabbildung von $f$?
\end{enumerate}

\subsection{Lösung 6a}
Die Abbildung der Polynome maximal zweiten Gerades auf ihre Stammfunktion aus den Polynomen maximal dritten Gerades
$$
    f : \begin{cases}
        \mathbb{P}^2 \to \mathbb{P}^3\\
        p(x) \mapsto \int_{C=0} p(x) \dx
    \end{cases}
$$

lässt sich ebenfalls als $f(p(x)) = A\cdot p(x)$ schreiben, mit

$$
    A = \begin{pmatrix}[1]
        0 & 0 & 0 \\
        1 & 0 & 0 \\
        0 & \nicefrac{1}{2} & 0 \\
        0 & 0 & \nicefrac{1}{3}
    \end{pmatrix}.
$$

\subsection{Lösung 6b}
Umgekehrt lässt sich
$$
    g : \begin{cases}
        \mathbb{P}^3 \to \mathbb{P}^2\\
        p(x) \mapsto p'(x)
    \end{cases}
$$

als $f(p(x)) = B\cdot p(x)$ mit

$$
    B = \begin{pmatrix}[1]
        0 & 0 & 0 & 0 \\
        1 & 0 & 0 & 0 \\
        0 & 2 & 0 & 0 \\
        0 & 0 & 3 & 0
    \end{pmatrix}
$$

schreiben.

\subsection{Lösung 6c}

Begrenzt man den Wertebereich der injektiven Abbildung $f$ auf die Menge aller Polynome
maximal vierten Gerades ohne die Polynome vom Grad Null
$\tilde{f}:\mathbb{P}^2 \to \mathbb{P}^3\setminus \mathbb{P}^0$,
so wird die Abbildung surjektiv und damit bijektiv.\\

Die Umkehrabbildung $f^{-1}$ kann nun so angegeben werden:

$$
    f^{-1} : \begin{cases}
        \mathbb{P}^3 \setminus \mathbb{P}^0 \to \mathbb{P}^2\\
        p(x) \mapsto p'(x)
    \end{cases}
$$

Die Abbildungsmatrix braucht dabei ebenfalls nur invertiert zu werden.

$$
    A^{-1} = \begin{pmatrix}[1]
        0 & 1 & 0 & 0 \\
        0 & 0 & 0 & 0 \\
        0 & 0 & \nicefrac{1}{2} & 0 \\
        0 & 0 & 0& \nicefrac{1}{3}
    \end{pmatrix}.
$$



\end{document}
