\documentclass[main.tex]{subfiles}

\begin{document}

\section{Aufgabe 4}
Die Abbildung $f_A$ dreht einen Vektor im $\R^3$ innerhalb der $x$-$z$-Ebene um einen Winkel $\phi$. Die Abbildung $f_B$ spiegelt einen Vektor im $\R^3$ an der $x$-Achse.
\begin{enumerate}
	\item Stellen Sie die zugehörigen Abbildungsmatrizen $A$ und $B$ auf.
	\item Stellen Sie die zugehörige Abbildungsmatrix der hintereinander geschalteten Abbildungen $f_B\circ f_A$ auf.
	\item Bestimmen Sie auch die zugehörige Abbildungsmatrix der Umkehrabbildung $(f_B\circ f_A)^{-1}$.
\end{enumerate}

\subsection{Lösung 4}
\begin{enumerate}
\item 
\begin{align*}
    \begin{array}{rl}
        A & = \begin{pmatrix}
            \cos\phi & 0 & -\sin\phi\\
            0 & 1 & 0\\
            \sin\phi & 0 & \cos\phi				
        \end{pmatrix}\\[7mm]
        B & = \begin{pmatrix}
            1 & 0 & 0\\
            0 & -1 & 0\\
            0 & 0 & -1
        \end{pmatrix}
    \end{array}
\end{align*}

\item
\begin{align*}
    \begin{array}{rl}
        f_B \circ f_A & = B\cdot A\\ 
        & = \begin{pmatrix}
            1 & 0 & 0\\
            0 & -1 & 0\\
            0 & 0 & -1
        \end{pmatrix} \cdot
        \begin{pmatrix}
            \cos\phi & 0 & -\sin\phi\\
            0 & 1 & 0\\
            \sin\phi & 0 & \cos\phi				
        \end{pmatrix}\\[7mm]
        & = \begin{pmatrix}
            \cos\phi & 0 & -\sin\phi\\
            0 & -1 & 0\\
            -\sin\phi & 0 & -\cos\phi	
        \end{pmatrix}
    \end{array}
\end{align*}

\item
Da die Vektoren der Abbildungsmatrix von $f_B\circ f_A$ eine Orthonormalbasis bilden, liegt somit eine Orthogonalmatrix vor, die zu ihrer Hauptdiagonalen symmetrisch ist und somit ist sie ihre eigene Inverse, also gilt $f_B\circ f_A = (f_B\circ f_A)^{-1}$
\end{enumerate}

\end{document}