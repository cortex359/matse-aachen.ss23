\documentclass[main.tex]{subfiles}

\begin{document}

\section{Aufgabe 6}
Eine Lineare Abbildung $\upPhi : \R^5\rightarrow\R^3$ hat bzgl. der kanonischen Basen $E$ die Abbildungsmatrix
\[
	F_E^E = \eqmatrix{rrrrr}{
		1 &  2 & -3 & 0 &  1 \\
		2 & -1 & -1 & 0 & -1 \\
		1 &  1 &  1 & 1 &  0
	}
\]

\begin{enumerate}
\item % a)
Stellen Sie die Transformationsmatrizen auf, die Vektoren der Basis
\[
    D = \left\{ \renewcommand{\sfrac}[2]{#1/#2}
        \vektor{\sfrac{1}{\sqrt{2}}\\0\\0\\\sfrac{1}{\sqrt{2}}\\0},
        \vektor{0\\\sfrac{2}{3}\\-\sfrac{1}{3}\\0\\\sfrac{2}{3}},
        \vektor{0\\\sfrac{2}{3}\\\sfrac{2}{3}\\0\\-\sfrac{1}{3}},
        \vektor{-\sfrac{1}{\sqrt{2}}\\0\\0\\\sfrac{1}{\sqrt{2}}\\0},
        \vektor{0\\-\sfrac{1}{3}\\\sfrac{2}{3}\\0\\\sfrac{2}{3}}
    \right\}
\]
bzw.
\[
    Z = \left\{
        \dfrac{1}{\sqrt{3}}\cdot \vektor{1\\1\\1},
        \dfrac{1}{\sqrt{6}}\cdot \vektor{1\\1\\-2},
        \dfrac{1}{\sqrt{2}}\cdot \vektor{1\\-1\\0}
    \right\}
\]
in die kanonische Basis transformiert.

\item % b)
Bestimmen Sie ohne aufwendige Rechnung die Transformationsmatrizen, die Vektoren aus der kanonischen Basis in die Basis $D$ bzw. $Z$ transformiert.

\item % c)
Beschreiben Sie, wie man die Abbildungsmatrix $F_Z^D$ zu $\upPhi$ bzgl. der Basen $D$ und $Z$ berechnen kann.
\end{enumerate}

\subsection{Lösung 6}
\subsubsection{a)}
\begin{align*}
    \begin{array}{rl}
        T^D_E = A_D & = \begin{pmatrix}
            \sfrac{1}{\sqrt{2}} & 0 & 0 & -\sfrac{1}{\sqrt{2}} & 0\\
            0 & \sfrac{2}{3} & \sfrac{2}{3} & 0 & -\sfrac{1}{3}\\
            0 & -\sfrac{1}{3} & \sfrac{2}{3} & 0 & \sfrac{2}{3}\\
            \sfrac{1}{\sqrt{2}} & 0 & 0 & \sfrac{1}{\sqrt{2}} & 0\\
            0 & \sfrac{2}{3} & -\sfrac{1}{3} & 0 & \sfrac{2}{3}\\
        \end{pmatrix}\\[1.4cm]
        T^Z_E = A_Z & = \begin{pmatrix}
            \sfrac{1}{\sqrt{3}} & \sfrac{1}{\sqrt{6}} & \sfrac{1}{\sqrt{2}}\\ 
            \sfrac{1}{\sqrt{3}} & \sfrac{1}{\sqrt{6}} & -\sfrac{1}{\sqrt{2}}\\
            \sfrac{1}{\sqrt{3}} & -\sfrac{2}{\sqrt{6}} & 0
        \end{pmatrix}		
    \end{array}
\end{align*}

\subsubsection{b)}
\begin{align*}
    \begin{array}{rl}
        T^E_D = (A_D)^{-1} = (A_D)^T & = \begin{pmatrix}
            \sfrac{1}{\sqrt{2}} & 0 & 0 & \sfrac{1}{\sqrt{2}} & 0\\
            0 & \sfrac{2}{3} & \sfrac{2}{3} & 0 & \sfrac{2}{3}\\
            0 & -\sfrac{1}{3} & \sfrac{2}{3} & 0 & -\sfrac{1}{3}\\
            -\sfrac{1}{\sqrt{2}} & 0 & 0 & \sfrac{1}{\sqrt{2}} & 0\\
            0 & -\sfrac{1}{3} & \sfrac{2}{3} & 0 & \sfrac{2}{3}\\
        \end{pmatrix}\\[1.4cm]
        T^E_Z = (A_Z)^{-1} = (A_Z)^T & = \begin{pmatrix}
            \sfrac{1}{\sqrt{3}} & \sfrac{1}{\sqrt{3}} & \sfrac{1}{\sqrt{3}}\\
            \sfrac{1}{\sqrt{6}} & \sfrac{1}{\sqrt{6}} & -\sfrac{2}{\sqrt{6}}\\
            \sfrac{1}{\sqrt{2}} & -\sfrac{1}{\sqrt{2}} & 0
        \end{pmatrix}
    \end{array}
\end{align*}

\subsubsection{c)}
\renewcommand{\dfrac}[2]{\cfrac{#1}{#2}}
\begin{align*}
    \begin{array}{rl}
        F^D_Z & = T^E_Z\cdot F^E_E\cdot T^D_E\\
        & = \eqmatrix{rrr}{
            \dfrac{1}{\sqrt{3}} & \dfrac{1}{\sqrt{3}} & \dfrac{1}{\sqrt{3}}\\[4mm]
            \dfrac{1}{\sqrt{6}} & \dfrac{1}{\sqrt{6}} & -\dfrac{2}{\sqrt{6}}\\[4mm]
            \dfrac{1}{\sqrt{2}} & -\dfrac{1}{\sqrt{2}} & 0
        } \cdot
        \eqmatrix{rrrrr}{
            1 & 2 & -3 & 0 & 1\\
            2 & -1 & -1 & 0 & -1\\
            1 & 1 & 1 & 1 & 0
        } \cdot
        \eqmatrix{rrrrr}{
            \dfrac{1}{\sqrt{2}} & 0 & 0 & -\dfrac{1}{\sqrt{2}} & 0\\[3mm]
            0 & \dfrac{2}{3} & \dfrac{2}{3} & 0 & -\dfrac{1}{3}   \\[3mm]
            0 & -\dfrac{1}{3} & \dfrac{2}{3} & 0 & \dfrac{2}{3}   \\[3mm]
            \dfrac{1}{\sqrt{2}} & 0 & 0 & \dfrac{1}{\sqrt{2}} & 0 \\[3mm]
            0 & \dfrac{2}{3} & -\dfrac{1}{3} & 0 & \dfrac{2}{3}   \\[3mm]
        }\\[30mm]
        &= \eqmatrix{rrrrr}{
     5/\sqrt{6} & 7/(3 \sqrt{3})   &   -2/(3 \sqrt{3}) & -\sqrt{3/2} & -8/(3 \sqrt{3}) \\
-1/(2 \sqrt{3}) & (2 \sqrt{2/3})/3 & -(7 \sqrt{2/3})/3 & -\sqrt{3}/2 & -11/(3 \sqrt{6}) \\
           -1/2 & 2 \sqrt{2}       &                 0 &         1/2 & -1/\sqrt{2}
        }
    \end{array}
\end{align*}

\end{document}

% {{1/sqrt(3),1/sqrt(3),(1/sqrt(3))},{1/sqrt(6),1/sqrt(6),-(2/sqrt(6))},{1/sqrt(2),-1/sqrt(2),0}}
% *
%{{1,2,-3,0,1},{2,-1,-1,0,-1},{1,1,1,1,0}}
% *
% {{1/sqrt(2),0,0,-1/sqrt(2),0},{0,2/3,2/3,0,-1/3},{0,-1/3,2/3,0,2/3},{1/sqrt(2),0,0,1/sqrt(2),0},{0,2/3,-1/3,0,2/3}}
% =
% {{1/Sqrt[2], 3, -1, -(1/Sqrt[2]), -2}, {Sqrt[2], -1, -1, -Sqrt[2], -1}, {Sqrt[2], 1/3, 4/3, 0, 1/3}}
