\documentclass[main.tex]{subfiles}

\begin{document}

\section{Aufgabe 2}
Bestimmen Sie die Potentialfunktion von
\begin{enumerate}
	\item $\begin{aligned}
		\vec{f}(x,y) 
		= \begin{pmatrix}
			y^2 + 2\ct{e}^{2x - y} - 1 \\[2mm]
			2x \cdot y - \ct{e}^{2x - y} + 4y^3
		\end{pmatrix}
		\end{aligned}$
%
	\item $\begin{aligned}
		\vec{g}(x,y) 
		= \begin{pmatrix}
			\ct{e}^x + 2x \cdot y \cdot \cos(x^2 + y) \\[2mm]
			\sin(x^2 + y) + y \cdot \cos(x^2 + y) + 6y
		\end{pmatrix}
		\end{aligned}$
\end{enumerate}

\subsection{Lösung 2}

$\vec{f}$ heißt Gradientenfeld, wenn es eine skalare Funktion $V$ gibt, mit der gilt
$\nabla (V) = \vec{f}$. Die Funktion $V$ heißt dann Potentialfunktion oder mehrdimensionale Stammfunktion von $\vec{f}$.

Im $\mathbb{R}^2$ existiert die Potentialfunktion nur, wenn gilt
$$
	\frac{\partial f_1}{\partial y} = \frac{\partial f_2}{\partial x}.
$$

\subsection{Lösung 2a}
Wir prüfen die Existenz der Potentialfunktion 
\begin{align*}
	\frac{\partial f_1}{\partial y} &= 2y - 2\ct{e}^{2x - y} \\
	\frac{\partial f_2}{\partial x} &= 2y - 2\ct{e}^{2x - y} \\
\end{align*}

und rechnen die Stammfunktion mit der ersten Komponenten aus:
\begin{align*}
	V(x,y) &= \int f_1(x,y) \dx \\
		   &= \int y^2 + 2\ct{e}^{2x - y} - 1 \dx \\
		   &= xy^2 + \ct{e}^{2x - y} - x + c(y) \\
\end{align*}

Die Konstante $c(y)$ wird nun berechnet durch partielle Ableitung nach $y$.

\begin{align*}
	\frac{\partial V}{\partial y} &= 2x\cdot y - \ct{e}^{2x - y} + c'(y) \\
								  &= f_2(x,y) \\
	     		 			      &= 2x \cdot y - \ct{e}^{2x - y} + 4y^3
\end{align*}

Daraus folgt für $c(y)$:
\begin{align*}
	c'(y) &= 4y^3 \\
	c(y)  &= \int 4y^3 \dy \\
		  &= y^4 + c \\
\end{align*}

und damit für $V(x,y)$
\begin{align*}
	V(x,y) &= xy^2 + \ct{e}^{2x - y} - x + y^4 + c \\
\end{align*}

\end{document}
