\documentclass[main.tex]{subfiles}

\begin{document}

\section{Aufgabe 1}
Gegeben sei die Kurve 
$\vec{X}(t) = \begin{pmatrix} t \\ t^2 \\ t^3 \end{pmatrix}$.
Berechnen Sie das Kurvenintegral im Vektorfeldes
\[
\vec{F}(x,y,z) 
= \begin{pmatrix} x + yz \\ y + x \cdot z \\ z + x \cdot y \end{pmatrix}
\]
entlang der Kurve.

\subsection{Lösung 1}

Das Kurvenintegral $W$ im Vektorfeld $\vec{F}(x,y,z)$ ist gegeben mit
\begin{equation*}
    W = \int \vec{F}(\vec{X}(t)) \cdot \vec{X}'(t) \dt
\end{equation*}

Wir berechnen:
\begin{align*}
    W &= \int \vec{F}(\vec{X}(t)) \cdot \vec{X}'(t) \dt \\
      &= \int \begin{pmatrix} t + t^2t^3 \\ t^2 + t \cdot t^3 \\ t^3 + t \cdot t^2 \end{pmatrix}
          \cdot \begin{pmatrix} 1 \\ 2t \\ 3t^2 \end{pmatrix} \dt \\
      &= \int \begin{pmatrix} t + t^5 \\ t^2 + t^4 \\ 2t^3 \end{pmatrix}
          \cdot \begin{pmatrix} 1 \\ 2t \\ 3t^2 \end{pmatrix} \dt \\
      &= \int t + t^5 + 2t \cdot ( t^2 + t^4 ) + 3t^2 \cdot 2t^3 \dt \\
      &= \int 9t^5 + 2t^3 + t \dt \\
      &= \frac{3}{2} t^6 + \frac{1}{2} t^4 + 1 + C \\
\end{align*}

\end{document}
