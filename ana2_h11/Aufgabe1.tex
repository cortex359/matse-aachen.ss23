\documentclass[main.tex]{subfiles}

\begin{document}

\section{Aufgabe 1}
Lösen Sie das Differentialgleichungssystem
\begin{align*}
    y' &= 5y -  z \\
    z' &= 2y + 8z
\end{align*}

\subsection{Lösung 1}
Wir stellen die erste Gleichung nach $z$ um und leiten ab. 
\begin{align*}
    z  &= 5y - y' \\
    z' &= 5y' - y''
\end{align*}

Nun setzen wir in die zweite Gleichung ein.
\begin{align*}
\begin{array}{rrl}
                & z'        &= 2y + 8z \\
\Leftrightarrow & 5y' - y'' &= 2y + 40y - 8y' \\
\Leftrightarrow & 0 &= 42y - 13y' + y'' \\
\end{array}
\end{align*}

Wir lösen die homogene DGL mit Hilfe der charakteristischen Gleichung.
\begin{align*}
\begin{array}{rrl}
                & 0 &= \lambda^2 -13\lambda + 42 \\
\Leftrightarrow & \lambda &= \frac{13}{2} \pm \sqrt{\left(\frac{13}{2}\right)^2 - 42} \\
\Rightarrow & \multicolumn{2}{l}{\lambda_1 = 7 \;\land\; \lambda_2 = 6}
\end{array}
\end{align*}

Und erhalten die allgemeine Lösung der ersten DGL, welche wir dann ableiten können.
\begin{align*}
    y(x)  &= c_0 \ct{e}^{7x} + c_1 \ct{e}^{6x} \\
    y'(x) &= 7c_0 \ct{e}^{7x} + 6c_1 \ct{e}^{6x} \\
\end{align*}

Mit ersten nach $z$ umgestellten DGL können wir durch Einsetzen sodann auch $z(x)$ bestimmen:
\begin{align*}
    z(x) &= 5y - y' \\
         &= 5c_0 \ct{e}^{7x} + 5c_1 \ct{e}^{6x} - 7c_0 \ct{e}^{7x} - 6c_1 \ct{e}^{6x} \\
         &= -2c_0 \ct{e}^{7x} - c_1 \ct{e}^{6x}
\end{align*}

Die allgemeine Lösung des Differentialgleichungssystems lautet somit:
\begin{align*}
    y &= c_0 \ct{e}^{7x} + c_1 \ct{e}^{6x} \\
    z &= -2c_0 \ct{e}^{7x} - c_1 \ct{e}^{6x}
\end{align*}

\end{document}
