\documentclass[main.tex]{subfiles}

\begin{document}

\section{Aufgabe 4}
Bestimmen Sie die allgemeine Lösung der Differentialgleichung 2.Ordnung und lösen Sie das gegebene Anfangswertproblem
\begin{equation*}
    y'' - 2y' + y = 0 \qquad y(0)=1, \quad y'(0)=0
\end{equation*}

\subsection{Lösung 4}
Die charakteristische Gleichung der homogenen DGL 2. Ordnung ist 
$$
    \lambda^2 - 2\lambda + 1 = 0.
$$
Mit $\lambda = 1$ liegt eine doppelte Nullstelle vor, weshalb der Ansatz
$$
    y(x) = c_1 \ct{e}^{x} + c_2 x \ct{e}^{x}
$$
zur Lösung gewählt werden muss.
Wir bestimmen die Ableitung und setzen die beiden Anfangswerte ein um die Parameter zu bestimmen.
\begin{align*}
    y(x)  &= c_1 \ct{e}^{x} + c_2 x \ct{e}^{x} \\
    y'(x) &= c_1 \ct{e}^{x} + c_2 \cdot \left( \ct{e}^{x} + \ct{e}^{x}x \right) \\
          &= c_1 \ct{e}^{x} + c_2\ct{e}^{x} + c_2\ct{e}^{x}x
\end{align*}

\begin{equation*}
\begin{array}{rrl}
                & y(0) & = 1 \\
\Leftrightarrow & c_1  &= 1 \\[2mm]
                & y'(0) &= 0 \\
\Leftrightarrow & 0 &= c_1 + c_2 \\
\Leftrightarrow & c_2 &= -1 \\
\end{array}
\end{equation*}

Die spezielle Lösung des Anfangswertproblems lautet somit
$$
    y(x) = \ct{e}^{x} - x \ct{e}^{x}.
$$

\end{document}
