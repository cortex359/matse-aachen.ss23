\documentclass[main.tex]{subfiles}

\begin{document}

\section{Aufgabe 5}
Gegeben seien die folgenden Differentialgleichungen 2. Ordnung
\begin{enumerate}
    \item $y'' - 6y' + 9y = -17 + 21x + 18x^2$
    \item $y'' - 8y' + 16y = -72e^{-2x}$
    \item $y'' - 7y' + 6y = 82\sin(2x) + 26\cos(2x)$
\end{enumerate}
Bestimmen Sie jeweils die Lösung $y(x)$.

\subsection{Lösung 5a}
\begin{align*}
    y_h'' - 6y_h' + 9y_h = 0 \\[5mm]
    \lambda^2 -6 \lambda + 9 = 0 \\
    \Leftrightarrow \lambda_{1,2} = 3
\end{align*}
Doppelte Nullstelle, daher ist die Lösung der homogenen DGL
$$
    y_h(x) = c_0 \ct{e}^{3x} + c_1 x \ct{e}^{3x}. \\
$$

Auf Grund der Störfunktion wählen wir für die partikuläre Gleichung den folgenden Ansatz:
\begin{align*}
    y_p(x)   &= c_0 + c_1 x + c_2 x^2\\
    y_p'(x)  &= c_1 + 2 c_2 x\\
    y_p''(x) &= 2 c_2\\
\end{align*}

\begin{equation*}
\begin{array}{rll}
                & y_p'' - 6y_p' + 9y_p &= -17 + 21x + 18x^2 \\
\Leftrightarrow & 2 c_2 - 6c_1 - 12 c_2 x + 9c_0 + 9c_1 x + 9c_2 x^2 &= -17 + 21x + 18x^2\\
\Leftrightarrow & \left(2 c_2 - 6c_1 + 9c_0\right)
    + \left(9c_1 - 12 c_2\right) x 
    + 9c_2 x^2 &= -17 + 21x + 18x^2\\
\end{array}
\end{equation*}
Der Koeffizientenvergleich liefert $c_2 =2$, $c_1 = 5$ und 
$c_0 = 1$, also erhalten wir folgende partikuläre Gleichung.
$$
    y_p(x) = 1 + 5x + 2x^2\\
$$

Die allgemeine Lösung lautet somit:
% y(x) = c_2 e^(3 x) x + c_1 e^(3 x) + 2 x^2 + 5 x + 1
$$
    y(x) =  c_0 \ct{e}^{3x} + c_1 x \ct{e}^{3x} + 2x^2 + 5x + 1
$$

\subsection{Lösung 5b}
Die homogene DGL, ihre charakteristische Gleichung und das vorlegen einer doppelten Nullstelle in der charakterischen Gleichung
\begin{align*}    
    y_h'' - 8y_h' + 16y_h = 0 \\
    \lambda^2 - 8\lambda + 16 = 0 \\
    \Leftrightarrow \lambda = 4
\end{align*}

liefert uns diese Lösung.
$$
    y_h(x) = c_0 \ct{e}^{4x} + c_1 x \ct{e}^{4x} 
$$

Setzt man den Ansatz für die rechte Seite
\begin{align*}
    y_p(x) &= c_0 \ct{e}^{\alpha x} \\
    y_p'(x) &= c_0 \alpha \ct{e}^{\alpha x} \\
    y_p''(x) &= c_0 \alpha^2 \ct{e}^{\alpha x}
\end{align*}
ein, so lassen sich die Parameter bestimmen.
\begin{equation*}
\begin{array}{cll}
                & y_p'' - 8y_p' + 16y_p &= -72e^{-2x} \\[2mm]
\Leftrightarrow &  c_0 \alpha^2 \ct{e}^{\alpha x}
                - 8  c_0 \alpha \ct{e}^{\alpha x}
                + 16 c_0 \ct{e}^{\alpha x}
                &= -72e^{-2x} \\[2mm]
\overset{\alpha = -2}{\Leftrightarrow} &  4 c_0 \ct{e}^{-2x}
                \cancel{+ 16  c_0 \ct{e}^{-2x}}
                \cancel{- 16 c_0 \ct{e}^{-2x}}
                &= -72e^{-2x} \\[2mm]
\Rightarrow & \multicolumn{2}{l}{ c_0 = -18 }
\end{array}
\end{equation*}
Wir erhalten $y_p(x) = 18 \ct{e}^{-2x}$ und damit lautet die allgemeine Lösung der DGL
$$
    y(x) = c_0 \ct{e}^{4x} + c_1 x \ct{e}^{4x} - 18 \ct{e}^{-2x}.
$$

\subsection{Lösung 5c}
Wir lösen die homogene DGL nach dem beschriebenen Verfahren.
\begin{align*}
    y_h'' - 7y_h' + 6y_h = 0 \\
    \lambda^2 - 7\lambda + 6 = 0 \\
    \Leftrightarrow \lambda_{1,2} = \frac{7}{2} \pm \sqrt{\frac{49-24}{4}}\\
    \Rightarrow  \lambda_1 = 6 \;\land\; \lambda_2 = 1
\end{align*}

$$
    y_h(x) = c_0 \ct{e}^{6x} + c_1 \ct{e}^{x}
$$

Zur Bestimmung der partikulären Lösung wählen wir den Ansatz
\begin{align*}
    y_p(x)   &= c_0 \sin(\alpha x) + c_1 \cos(\alpha x) \\
    y_p'(x)  &= \alpha c_0 \cos(\alpha x) - \alpha c_1 \sin(\alpha x) \\
    y_p''(x) &= - \alpha^2 c_0 \sin(\alpha x) - \alpha^2 c_1 \cos(\alpha x) \\
\end{align*}
und setzten ein:

\begin{equation*}
\begin{array}{crl}
                & 82\sin(2x) + 26\cos(2x) &= y_p'' - 7y_p' + 6y_p \\
\Leftrightarrow & 82\sin(2x) + 26\cos(2x) &=
    - \alpha^2 c_0 \sin(\alpha x) - \alpha^2 c_1 \cos(\alpha x) 
    - 7\alpha c_0 \cos(\alpha x) + 7\alpha c_1 \sin(\alpha x)\\
    && \quad +6c_0 \sin(\alpha x) + 6c_1 \cos(\alpha x) \\
\overset{\alpha = 2}{\Leftrightarrow} & 82\sin(2x) + 26\cos(2x) &=
    \underbrace{\left(2c_0 + 14c_1\right)}_{\overset{!}{=} 82} \sin(2x)
    + \underbrace{\left(2c_1 - 14c_0\right)}_{\overset{!}{=} 26}\cos(2x) \\
\end{array}
\end{equation*}

Der Koeffizientenvergleich ergibt $c_0 = -1$ und $c_1 = 6$ und wir erhalten die partikuläre Lösung
$$
    y_p(x) = -\sin(2x) + 6\cos(2x),
$$
sowie die allgemein Lösung der DGL:
$$
    y(x) = c_0 \ct{e}^{6x} + c_1 \ct{e}^{x} -\sin(2x) + 6\cos(2x)
$$

\end{document}
