\documentclass[main.tex]{subfiles}

\begin{document}

\section{Aufgabe 2}
Lösen Sie das folgende Anfangswertproblem
\begin{equation*}
    y'' - 3y' + 2y = \ct{e}^x \qquad y(0)=3, \quad y'(0)=4
\end{equation*}

\subsection{Lösung 2}
Es handelt sich um eine lineare inhomogene DGL 2. Ordnung in expliziter Form.
Bestimme die homogene Lösung mit dem Ansatz
\begin{align*}
    y_h(x) &= \ct{e}^{\lambda x} \\
    y_h'(x) &= \lambda\cdot \ct{e}^{\lambda x} \\
    y_h''(x) &= \lambda^2 \cdot \ct{e}^{\lambda x} \\
\end{align*}

zur charakteristischen Gleichung
\begin{equation*}
\begin{array}[pos]{rrl}
       & y_h'' - 3y_h' + 2y_h & = 0 \\
\Leftrightarrow & \lambda^2 \cdot \ct{e}^{\lambda x} - 3\lambda\cdot \ct{e}^{\lambda x} + 2\ct{e}^{\lambda x} &= 0\\
\Leftrightarrow & \lambda^2 - 3\lambda + 2 &= 0\\
\end{array}
\end{equation*}
welche die Lösungen $\lambda_1 = 2 \land \lambda_2 = 1$ hat.

$$
    y_h(x) = c_1 \ct{e}^{2x} + c_2 \ct{e}^{x}
$$

Für die partikuläre Lösung wähle Ansatz vom Typ der rechten Seite mit
\begin{align*}
    y_p(x) = 
\end{align*}


% Lösung durch WolframAlpha bestätigt
$$
    y(x) = \ct{e}^x \cdot \left(-x + 2 \ct{e}^x + 1\right)
$$

\end{document}
