\documentclass[main.tex]{subfiles}

\begin{document}

\section{Aufgabe 1}
Gegeben seien die folgenden Differentialgleichungen 2. Ordnung.
\begin{enumerate}
    \item $y'' - 6y' + 8y = -24 -56x + 48x^2$
    \item $y'' + 2y' = 8 + 36x$
\end{enumerate}
Bestimmen Sie deren Lösung $y(x)$.

\subsection{Lösung 1a}
Die lineare inhomogene DGL 2. Ordnung ist in expliziter Form angegeben.
Wir bestimmen homogene Lösung der DGL
\begin{align*}
    y_h'' - 6y_h' + 8y_h &= 0 \\
\end{align*}
und verwenden dazu den folgenden Ansatz:
\begin{align*}
    y_h(x) &= \ct{e}^{\lambda x} \\
    y_h'(x) &= \lambda\cdot \ct{e}^{\lambda x} \\
    y_h''(x) &= \lambda^2 \cdot \ct{e}^{\lambda x} \\
\end{align*}

Damit lässt sich $\lambda$ bestimmen.
\begin{equation*}
\begin{array}[pos]{lrl}
                & y_h'' - 6y_h' + 8y_h &= 0 \\[2mm]
\Leftrightarrow & \lambda^2 \cdot \ct{e}^{\lambda x} - 6\lambda\cdot \ct{e}^{\lambda x} + 8\ct{e}^{\lambda x} &= 0 \\[2mm]
\Leftrightarrow & \lambda^2 -6\lambda +8 &= 0\\[2mm]
\Rightarrow & \multicolumn{2}{l}{\lambda_1 = 4\; \land\; \lambda_2 = 2} \\
\end{array}
\end{equation*}

$0 = \lambda^2 -6\lambda +8$ wird auch als \textit{charakteristische Gleichung} bezeichnet. 

Da wir zwei mögliche Lösungen für $\lambda$ erhalten, lautet die Lösung
$$
    y_h(x) = c_1\ct{e}^{4x} + c_2\ct{e}^{2x}.
$$

Die partikuläre Lösung bestimmen wir mit dem Ansatz vom Typ der rechten Seite für die Störfunktion $g(x) = -24 -56x + 48x^2$

\begin{align*}
    y_p(x) &= c_0 + c_1x + c_2x^2 \\
    y_p'(x) &= c_1 + 2c_2x \\
    y_p''(x) &= 2c_2 \\
\end{align*}

und setzten in die linke Seite ein:

\renewcommand{\equiv}{\Leftrightarrow}

\begin{equation*}
\begin{array}{rrl}
       & y_p''(x) - 6y_p'(x) + 8y_p(x) &= -24 -56x + 48x^2 \\[2mm]
\equiv & 2c_2 - 6\left( c_1 + 2c_2x \right) + 8\left( c_0 + c_1x + c_2x^2 \right) &= -24 -56x + 48x^2 \\[2mm]
\equiv & \left(2c_2 - 6c_1 + 8c_0\right) - x \left( 12c_2 - 8c_1\right) + 8c_2x^2 &= -24 -56x + 48x^2 \\[2mm]
\end{array}
\end{equation*}

Der Koeffizientenvergleich zeigt, dass $c_2 = 6,\; c_1 = 2,\; c_0 = -3$. Somit erhalten wir
$$
    y_p(x) = -3 +2x +6x^2
$$

und mit $y(x) = y_h(x) + y_p(x)$

% Lösung durch WolframAlpha bestätigt
$$
    y(x) = c_1 \ct{e}^{2 x} + c_2 \ct{e}^{4 x} + 6 x^2 + 2 x - 3
$$

\subsection{Lösung 1b}

Die lineare inhomogene DGL 2. Ordnung ist in expliziter Form angegeben. Wir bestimmen die homogene Lösung der DGL mit dem gleichen Ansatz wie zuvor.
Die charakteristische Gleichung lautet $\lambda^2 + 2\lambda =0$, die Lösungen sind $\lambda_1 = 0 \land \lambda_2 = -2$ und somit erhalten wir
\begin{align*}
    y_h(x) = c_1\ct{e}^{-2x} + c_2.\\
\end{align*}

Die partikuläre Lösung lässt sich mit gleichem Ansatz wie in 1a) durch einsetzen in die DGL so bestimmen:
\begin{equation*}
\begin{array}{rrl}
       & y_p''(x) + 2y_p'(x) &= 8 + 36x \\[2mm]
\equiv & 2c_2 + 2\cdot \left(c_1 + 2c_2x\right) &= 8 + 36x \\
\equiv & \left(2c_2 + 2c_1\right) + 4c_2x &= 8 + 36x \\
\end{array}
\end{equation*}

Der Koeffizientenvergleich zeigt, dass $c_2 = 9,\; c_1 = -5$. Somit erhalten wir
$$
    y_p(x) = c_0 -5x + 9x^2.
$$

Wir fassen $c_2 + c_0 := c$ zusammen wir erhalten die allgemeine Lösung der DGL.
% Lösung durch WolframAlpha bestätigt
$$
    y(x) = c_1 \ct{e}^{-2 x} + c + 9x^2 - 5x
$$
\pagebreak

\end{document}
