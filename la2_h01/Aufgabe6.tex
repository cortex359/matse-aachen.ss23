\documentclass[main.tex]{subfiles}

\begin{document}

\section{Aufgabe 6}

\begin{enumerate}
    \item Sei $0\neq a\in \mathbb{R}^{n}$ und $f:\mathbb{R}^{n}\rightarrow \mathbb{R} ,\ f( x) =a\cdotp x$. Zeigen Sie: $f$ ist zwar surjektiv, aber nicht injektiv.
    \item Wir betrachten $\mathcal{C}[ 0,1]$, die Menge der stetigen Funktionen auf $[ 0,1]$. Man zeige: Die Auswertung einer stetigen Funktion in einem festen Punkt $a\in [ 0,1]$, d.h. die Abbildung
        \begin{equation*}
            \varphi _{a} :\mathcal{C}[ 0,1]\rightarrow \mathbb{R} ,\ \varphi _{a}( f) =f( a)
        \end{equation*}
        ist linear. Ist $\varphi _{a}$ bijektiv?
    \item Der Vektorraum $\mathcal{C}^{1}[ a,b]$ der stetig differenzierbaren Funktionen ist in den Vektorraum der stetigen Funktionen $\mathcal{C}[ a,b] ,\ a< b\in \mathbb{R}$, eingebettet durch die Einbettung $\varphi :\mathcal{C}^{1}[ a,b]\rightarrow \mathcal{C}[ a,b] ,\ \varphi ( f) =f.$ Man schreibt $\mathcal{C}^{1}[ a,b] \hookrightarrow \mathcal{C}[ a,b] .$ Man zeige: Die Einbettung $\varphi $ ist linear, injektiv, aber nicht surjektiv.
\end{enumerate}

\subsection{Lösung 6a}
Gemeint scheint hier die Skalarmultiplikation zu sein.
$$
f:\begin{cases}
    \mathbb{R}^{n}\to \mathbb{R}\\
    x \mapsto \scalarprod{a,x}
\end{cases}
$$

Die Abbildung ist surjektiv, da $\Bild(f) =\mathbb{R}$ gilt und daher auch $\text{rg}(f) =\dim(\Bild(f)) =\dim(\mathbb{R}) =1$.\\

Sie ist jedoch nicht injektiv, da $\ker( f) \neq \{0\}$.\\

Notiz: $\Bild\left(f\right) =\left\{f( x)\middle| x\in \tilde{X}\right\} \subseteq \mathbb{R} ,\ \tilde{X} \subseteq \mathbb{R}^{n}$


\subsection{Lösung 6b}
Die Abbildung $\varphi _{a}$ ist genau dann linear, wenn sie additiv und homogen ist.

Additivität:
\begin{equation*}
    \begin{array}{ r r l }
     & \varphi _{a}( f+y) & =f( a) +y\\
     & \varphi _{a}( f) +y & =f( a) +y
    \end{array}
\end{equation*}
Homogenität:
\begin{equation*}
    \begin{array}{ r r l }
     & \varphi _{a}( \lambda f) & =\lambda f( a)\\
     & \lambda \varphi _{a}( f) & =\lambda f( a)
    \end{array}
\end{equation*}
Die Abbildung ist linear.\\

$\varphi _{a}$ ist nicht injektiv, da es mehrere stetige Funktionen gibt, welche an einem Punkt $a$ den gleichen Wert haben können. Beispiel: Mit $a=1$ und $f( x) =x,\ g( x) =-x+2$ also $f,g\in \mathcal{C}[ 0,1]$ gilt $\varphi _{1}( f) =\varphi _{1}( g) =1.$\\

Daher kann $\varphi _{a}$ auch nicht bijektiv sein.


\subsection{Lösung 6c}
$$
\varphi : \begin{cases}
    \mathcal{C}^1 [a,b] \to \mathcal{C} [a,b] & a < b \in \mathbb{R}\\
    f \mapsto f
\end{cases}
$$

Es handelt sich um eine lineare Abbildung, da sie homogen
$$
    \varphi (\lambda \cdot f) = \lambda \varphi (f),\quad \lambda \in \mathbb{R} \quad \checkmark
$$
und additiv ist.
$$
    \varphi (f_1 + f_2) = \varphi (f_1) + \varphi (f_2),\quad f_{1,2} \in \mathcal{C}^1 [a,b] \quad \checkmark
$$

Da aus $\varphi(f_1) = \varphi (f_2) \Rightarrow f_1 = f_2$ folgt, ist die Einbettung injektiv. 
Da es jedoch stetige Funktionen gibt, welche nicht stetig differenzierbar sind, ist die Einbettung nicht surjektiv. $\mathcal{C}^1 [a,b] \subsetneq \mathcal{C} [a,b]$ Beispiel: $f(x) = \abs{x} \in \mathcal{C} [-1, 1]$, aber $f(x) \notin \mathcal{C}^1 [-1, 1]$.


\end{document}
