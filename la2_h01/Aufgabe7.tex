\documentclass[main.tex]{subfiles}

\begin{document}

\section{Aufgabe 7}

Die Abbildung $f:A\rightarrow B$, wobei $A=\mathbb{Z}$ und $B=\mathbb{Z}$ sind, sei folgendermaßen definiert:
\begin{equation*}
    f( a) =a\bmod 17.
\end{equation*}

\begin{enumerate}
    \item Handelt es sich um eine lineare Abbildung?
    \item Wie lautet der Kern von $f$?
    \item Zeigen Sie, dass die Abbildung weder surjektiv noch injektiv ist.
    \item Welche Möglichkeiten gibt es, $A$ und $B$ zu wählen, damit die Abbildung bijektiv ist?
\end{enumerate}

\subsection{Lösung 7a}

Die Abbildung $f:\mathbb{Z}\rightarrow \mathbb{Z}$ ist genau dann linear, wenn sie additiv und homogen ist.

Additivität:
\begin{equation*}
\begin{array}{ r r l }
 f(a+b) & = & (a+b) \bmod 17\\
    & \neq & a \bmod 17 + b \bmod 17\\
    & = & f(a) + f(b)
\end{array}
\end{equation*}
Die Abbildung ist nicht linear, da sie nicht additiv ist.

\subsection{Lösung 7b}
Der Kern von $f$ ist die Menge aller Vielfachen von 17:
\begin{equation*}
    \ker( f) =f^{-1}( 0) =\{n\cdotp 17 | n\in \mathbb{Z}\}
\end{equation*}

\subsection{Lösung 7c}

Die Abbildung ist nicht injektiv, da nicht jedes Element der Zielmenge maximal einmal als Funktionswert angenommen wird. Beispiel: $f(1) = f(18) = 1$.\\

Die Abbildung ist nicht surjektiv, weil nicht jedes Element der Zielmenge mindestens einmal als Funktionswert angenommen wird. Beispiel:
$\mathbb{Z} = B \ni  17 \notin \Bild (f)$.

\subsection{Lösung 7d}

Man könnte den Definitionsbereich auf das Intervall $A=[n;n+16],\ n\in \mathbb{Z}$ für ein beliebiges
$n\geq 0$ einschränken um die Abbildung injektiv zu machen und den Wertebereich auf das Intervall
$B=[0;16]$ zuschneiden, damit die Abbildung surjektiv wird.\\

Die Abbildung $f:[ n;n+16]\rightarrow [ 0;16] ,n\in \mathbb{Z}$ wäre bijektiv.

\end{document}
